We can proceed exactly as for the case a. We can first analyze the four pure strategies and we already find two Nash equilibria :

- $\mathbf{\{x_1\}}\times\mathbf{\{x_2\}}$. Indeed step 3 implies :
\begin{align*}
    3 =w_1&\ge u_1(x_1,x_2)=2 \\
    7=w_2&\ge u_2(x_1,y_2)=6 
\end{align*}
And so all the step are respected (note that step 1 and 2 are trivial for the pure strategies) 

- $\mathbf{\{y_1\}}\times\mathbf{\{y_2\}}$. Indeed step 3 implies :
\begin{align*}
    7 =w_1&\ge u_1(x_1,y_2)=6 \\
    3=w_2&\ge u_2(y_1,x_2)=2
\end{align*}
Again all the step are respected.


\begin{itemize}
  \item[$\bullet$] Randomized vs pure strategy supports
\end{itemize}

One can develop Payoff equation (step 2) as in Exercise "a" and see that all support does not lead to a Nash equilibrium.

\begin{itemize}
  \item[$\bullet$] Fully randomized strategy 
\end{itemize}
- $\mathbf{\{x_1,y_1\}}\times\mathbf{\{x_2,y_2\}}$.

Again we first check the payoff equations (step 2), for the player 1 ($i=1$) we have : 
\begin{align*}
 \sigma_2(x_2)u_1(x_1,x_2) + \sigma_2(y_2)u_1(x_1,y_2) &= \sigma_2(x_2)u_1(y_1,x_2) + \sigma_2(y_2)u_1(y_1,y_2) \\
 3\sigma_2(x_2)+6\sigma_2(y_2)&= 2\sigma_2(x_2) + 7\sigma_2(y_2) \\
 \sigma_2(x_2)&=\sigma_2(y_2)
\end{align*}

Moreover we know that $\sigma_2(x_2) + \sigma_2(y_2) = 1$ and so we find $\sigma_2(x_2)=\frac{1}{2}$ and $\sigma_2(x_2)=\frac{1}{2}$

Now for the player 2 ($i=2$) we have : 
\begin{align*}
 \sigma_1(x_1)u_2(x_1,x_2) + \sigma_1(y_1)u_2(y_1,x_2) &= \sigma_1(x_1)u_2(x_1,y_2) + \sigma_1(y_1)u_2(y_1,y_2) \\
 7\sigma_1(x_1)+2\sigma_1(y_1)&= 6\sigma_1(x_1) + 3\sigma_1(y_1) \\
 \sigma_1(x_1)&=\sigma_1(y_1)
\end{align*}

We also know that $\sigma_1(x_1) + \sigma_1(y_1) = 1$ and so we find $\sigma_1(x_1)=\frac{1}{2}$ and $\sigma_1(y_1)=\frac{1}{2}$

To conclude we have 3 Nash equilibrium :
\begin{itemize}
  \item[$\bullet$]  $\sigma_1 = \left(x_1,x_2\right)$ with $w_1= \left(3,7\right)$
  \item[$\bullet$]  $\sigma_2 = \left(y_1,y_2\right)$ with $w_2= \left(7,3\right)$
  \item[$\bullet$]  $\sigma_3 = \left( \frac{1}{2}x_1 + \frac{1}{2}y_1, \frac{1}{2}x_2 + \frac{1}{2}y_2\right)$ with $w_3= \left(\frac{9}{2},\frac{9}{2}\right)$
\end{itemize}
