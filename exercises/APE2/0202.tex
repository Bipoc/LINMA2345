\begin{enumerate}[label=\alph*.]
	
	%%%%%%%%%%%%%%%%%%%%%%
	%%%%%%%%% a %%%%%%%%%%
	%%%%%%%%%%%%%%%%%%%%%%
	\item On Figure \ref{fig:stratForm45} we can find the strategic form.
	      
	      \begin{center}
\centering

\begin{tikzpicture}
\node[noeud-std] (noeud0) {}
   [sibling distance=8.5cm]
      child {node[noeud-std] (noeud1) {}
      [sibling distance=2.5cm]
      child[level distance=1cm]{node[noeud-std] (noeud2){} 
         	[sibling distance=2.5cm]
         	child[level distance=1.5cm]{node[noeud-std,fill=white] (noeud3){} }
         	child[level distance=1.5cm]{node[noeud-std,fill=white] (noeud4){} }}
      child[level distance=1cm]{node[noeud-std,fill=white] (noeud15){} 
         	[sibling distance=2cm]
         	}
      child[level distance=1cm]{node[noeud-std] (noeud5){} 
         	[sibling distance=2.5cm]
         	child[level distance=1.5cm]{node[noeud-std,fill=white] (noeud6){} }
         	child[level distance=1.5cm]{node[noeud-std,fill=white] (noeud7){} }}
   }
   child {node[noeud-std] (noeud8) {}
       [sibling distance=2.5cm]
       child[level distance=1cm]{node[noeud-std] (noeud9){} 
         	[sibling distance=2.5cm]
         	child[level distance=1.5cm]{node[noeud-std,fill=white] (noeud10){} }
         	child[level distance=1.5cm]{node[noeud-std,fill=white] (noeud11){} }}
       child[level distance=1cm]{node[noeud-std,fill=white] (noeud16){} 
         	[sibling distance=2cm]
         	}
       child[level distance=1cm]{node[noeud-std] (noeud12){} 
         	[sibling distance=2.5cm]
         	child[level distance=1.5cm]{node[noeud-std,fill=white] (noeud13){} }
         	child[level distance=1.5cm]{node[noeud-std,fill=white] (noeud14){} }}   
   }
;

\node[above=5pt] at (noeud0) {$0$}; 
\node[above=-25pt] at (noeud0) {}; 
\node[above=5pt] at (noeud1) {$1.a$};
\node[above=-25pt] at (noeud1) {};   
\node[above=5pt] at (noeud2) {$2.c$};  
\node[above=-25pt] at (noeud2) {};
\node[above=-20pt] at (noeud3) {$15 /  1$};     
\node[above=-35pt] at (noeud3) {}; 
\node[above=-20pt] at (noeud4) {$12  / 4$};   
\node[above=-35pt] at (noeud4) {}; 
\node[above=5pt] at (noeud5) {$2.d$}; 
\node[above=-25pt] at (noeud5) {};
\node[above=-20pt] at (noeud6) {$16  /  0$};  
\node[above=-35pt] at (noeud6) {};   
 \node[above=-20pt] at (noeud7) {$ 12  /  4$};  
\node[above=-35pt] at (noeud7) {};
\node[above=5pt] at (noeud8) {$1.b$}; 
\node[above=-25pt] at (noeud8) {};
\node[above=5pt] at (noeud9) {$2.c$};
\node[above=-25pt] at (noeud9) {};
\node[above=-35pt] at (noeud10) {}; 
\node[above=-20pt] at (noeud10) {$1 / 15$};   
\node[above=-35pt] at (noeud11) {}; 
\node[above=-20pt] at (noeud11) {$12  /  4$}; 
\node[above=5pt] at (noeud12) {$2.d$}; 
\node[above=-25pt] at (noeud12) {};
\node[above=-35pt] at (noeud13) {};   
\node[above=-20pt] at (noeud13) {$0 / 16$};
\node[above=-35pt] at (noeud14) {}; 
\node[above=-20pt] at (noeud14) {$12  /  4$}; 
\node[above=-35pt] at (noeud15) {}; 
\node[above=-20pt] at (noeud15) {$12  /  4$}; 
\node[above=-35pt] at (noeud16) {}; 
\node[above=-20pt] at (noeud16) {$4  /  12$}; 

  
\node[above left] at ($(noeud0)!{0.25}!(noeud1)$) {red : 0.5};
\node[above right] at ($(noeud0)!{0.25}!(noeud8)$) { black : 0.5};
\node[above right] at ($(noeud1)!{0.25}!(noeud5)$) {R4};
\node[above left] at ($(noeud1)!{0.3}!(noeud2)$) { R3};
\node[below left] at ($(noeud1)!{0.3}!(noeud15)$) {F};
\node[above right] at ($(noeud8)!{0.3}!(noeud12)$) { r4};
\node[above left] at ($(noeud8)!{0.3}!(noeud9)$) { r3};
\node[below left] at ($(noeud8)!{0.3}!(noeud16)$) { f};
\node[above right] at ($(noeud2)!{0.3}!(noeud4)$) { P};
\node[above left] at ($(noeud2)!{0.3}!(noeud3)$) { M};
\node[above right] at ($(noeud5)!{0.3}!(noeud7)$) { p};
\node[above left] at ($(noeud5)!{0.3}!(noeud6)$) { m};
\node[above right] at ($(noeud9)!{0.3}!(noeud11)$) { P};
\node[above left] at ($(noeud9)!{0.3}!(noeud10)$) { M};
\node[above right] at ($(noeud12)!{0.3}!(noeud14)$) { p};
\node[above left] at ($(noeud12)!{0.3}!(noeud13)$) { m};
\end{tikzpicture}



\captionof{figure}{A nice strategic form}
\label{fig:stratForm45}
\end{center}
	      
	      
	      Let us explain some notations. First, the information states
	      \begin{itemize}
	      	\item Information state $1.a$: Matthew knows that card is red
	      	\item Information state $1.b$: Matthew knows that card is black
	      	\item Information state $2.c$: Beno�t knows that Matthew has raised 3 EUR but Beno�t does not know the color of the card. Notice that we use this information state two times!
	      	\item Information state $2.d$: Beno�t knows that Matthew has raised 4 EUR but Beno�t does not know the color of the card. Notice that we use this information state two times!
	      \end{itemize}
	      
	      Secondly, let us explain the abbreviations
	      \begin{itemize}
	      	\item F: Matthew decides to fold 
	      	\item R3: Matthew decides to raise 3 EUR 
	      	\item R4: Matthew decides to raise 4 EUR 
	      	\item M: Beno�t decides to meet Matthew's raise
	      	\item P: Beno�t decides to pass
	      \end{itemize}
	      
	      Finally, let us explain why we use capital letters next to small letters. It is important to be able to differentiate the action of a player depending on which information state he is in. Indeed, two actions may look identical, while they actually take place in two different regions of the strategic form! For example, if Matthew decides to raise 3 EUR with a red card, we denote that choice with R3; Matthew decides to raise 3 EUR with a black card, we denote that choice with r3. The other conventions are
	      \begin{itemize}
	      	\item For Matthew: capital letters when we have a red card, small letters when we have a black card
	      	\item For Beno�t: capital letters when Matthew has raised 3 EUR, small letters when Matthew has raised 4 EUR
	      \end{itemize}
	      
	      
	      
	      
	      
	      The normal representation can be found on Table \ref{tab:normRep45}. 
	      
	      \begin{table}[h]
	      	\centering
	      	\begin{tabular}{c|cccccccc}
	      		       && Mm           && Mp           && Pm           && Pp     \\ \hline
	      		R3r3   && $8 /  8$     && $8  /   8$   && $12  /  4$   && $12 /   4$ \\
	      		R3r4   && $7.5 / 8.5$  && $ 13.5/ 2.5$ && $6  /  10$   && $12  /  4$ \\
	      		R3f    && $9.5 / 6.5 $ && $9.5 / 6.5$  && $8  /   8$   && $8  /   8$ \\
	      		R4r3   && $8.5 /7.5$   && $ 6.5 / 9.5$ && $14  /   2$  && $12 /   4$ \\
	      		R4r4   && $8 /    8$   && $12  /    4$ && $8  /   8$   && $12 /   4$ \\
	      		R4f    && $10 /     6$ && $8  /    8$  && $10 /   6$   && $8  /   8$ \\
	      		Fr3    && $6.5 / 9.5$  && $6.5 / 9.5$  && $12  /  4$   && $12  /  4$ \\
	      		Fr4    && $6 /    10$  && $12  /    4$ && $6  /  10$   && $12  /  4$ \\
	      		Ff     && $8  /   8$   && $8  /   8$   && $8  /   8$   && $8  /   8$
	      	\end{tabular}
	      	\caption{A nice normal representation}
	      	\label{tab:normRep45}
	      \end{table}
	      
	      
	      
	\item Let us analyze the normal representation. We can eliminate the weakly dominated strategies. Strategy Fr3 is weakly dominated by the pure strategy R3r3, strategy Fr4 is weakly dominated by the pure strategy R3r4 and strategy Ff is weakly dominated by the pure strategy R3f. Hence the table becomes
	
	\begin{table}[h]
	      	\centering
	      	\begin{tabular}{c|cccccccc}
	      		       && Mm           && Mp           && Pm           && Pp     \\ \hline
	      		R3r3   && $8 /  8$     && $8  /   8$   && $12  /  4$   && $12 /   4$ \\
	      		R3r4   && $7.5 / 8.5$  && $ 13.5/ 2.5$ && $6  /  10$   && $12  /  4$ \\
	      		R3f    && $9.5 / 6.5 $ && $9.5 / 6.5$  && $8  /   8$   && $8  /   8$ \\
	      		R4r3   && $8.5 /7.5$   && $ 6.5 / 9.5$ && $14  /   2$  && $12 /   4$ \\
	      		R4r4   && $8 /    8$   && $12  /    4$ && $8  /   8$   && $12 /   4$ \\
	      		R4f    && $10 /     6$ && $8  /    8$  && $10 /   6$   && $8  /   8$
	      	\end{tabular}
	      \end{table}
	      
	      Matthew's position is preferable, because in most of the cases, his payoff is bigger than Beno�t's.
	      
	      
	      
	      
\end{enumerate}