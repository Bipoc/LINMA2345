\documentclass{../ape}

\usepackage{../../linma2345}

\newcommand{\pawn}[5]{\path (#1.5,#2.5) node {\includegraphics[scale=.7]{images/#3-#4.#5}};}
\newcommand{\gameone}{
% draw black squares
\foreach \x in {0,...,3}
{
	\foreach \y in {0,...,3}
	{
		\fill[black!40] (2*\x,2*\y) rectangle (2*\x+1,2*\y+1);
		\fill[black!40] (2*\x+1,2*\y+1) rectangle (2*\x+2,2*\y+2);
	}
}
\draw[line width=1mm,black!80,rounded corners] (0,0) rectangle (8,8);

% add the pawns
\pawn{1}{2}{W}{bishop}{png}
\pawn{1}{3}{W}{pawn}{png}
\pawn{2}{4}{W}{bishop}{png}
\pawn{4}{4}{W}{pawn}{png}
\pawn{5}{1}{W}{king}{png}
\pawn{6}{2}{W}{pawn}{png}
\pawn{7}{3}{W}{tower}{png}

\pawn{0}{6}{B}{pawn}{png}
\pawn{0}{7}{B}{tower}{png}
\pawn{1}{4}{B}{king}{jpg}
\pawn{5}{4}{B}{pawn}{png}
\pawn{6}{6}{B}{pawn}{png}
\pawn{7}{5}{B}{pawn}{png}
}

\begin{document}

\session{1}{Introduction to Game Theory}

\section{}
It is W. White's turn and he can win for sure. Can you see how?

\begin{center}
\scalebox{.7}{
\begin{tikzpicture}
  \gameone{}
  \pawn{3}{0}{W}{tower}{png}
  \pawn{2}{7}{B}{bishop}{jpg}
  \pawn{2}{2}{W}{pawn}{png}
  \pawn{5}{7}{B}{tower}{png}
\end{tikzpicture}
}
\end{center}

\begin{solution}
  \url{chess.com} reports Mate in 3 (Stockfish.jl | depth 31)
  \begin{enumerate}
      \item
        Rd6 Be6
\begin{center}
\scalebox{.5}{
\begin{tikzpicture}
  \gameone{}
  \pawn{3}{5}{W}{tower}{png}
  \pawn{4}{5}{B}{bishop}{jpg}
  \pawn{2}{2}{W}{pawn}{png}
  \pawn{5}{7}{B}{tower}{png}
\end{tikzpicture}
}
\end{center}
      \item
        Rxe6 Rf7
\begin{center}
\scalebox{.5}{
\begin{tikzpicture}
  \gameone{}
  \pawn{4}{5}{W}{tower}{png}
  \pawn{2}{2}{W}{pawn}{png}
  \pawn{5}{6}{B}{tower}{png}
\end{tikzpicture}
}
\end{center}
      \item
        c4\#
\begin{center}
\scalebox{.5}{
\begin{tikzpicture}
  \gameone{}
  \pawn{4}{5}{W}{tower}{png}
  \pawn{2}{3}{W}{pawn}{png}
  \pawn{5}{6}{B}{tower}{png}
\end{tikzpicture}
}
\end{center}
  \end{enumerate}
\end{solution}

\section{}
It is again W. White's turn, and he is again sure to win but he is in a hurry and wants to finish the game as quickly as possible. How should he proceed, depending on whether he is facing his son (who did not really understand how to play with the knight yet) or a grand chess master?

\begin{center}
\scalebox{.7}{
\begin{tikzpicture}

% draw black squares
\foreach \x in {0,...,3}
{
	\foreach \y in {0,...,3}
	{
		\fill[black!40] (2*\x,2*\y) rectangle (2*\x+1,2*\y+1);
		\fill[black!40] (2*\x+1,2*\y+1) rectangle (2*\x+2,2*\y+2);
	}
}
\draw[line width=1mm,black!80,rounded corners] (0,0) rectangle (8,8);

% add the pawns
\pawn{1}{2}{W}{bishop}{png}
\pawn{1}{3}{W}{pawn}{png}
\pawn{2}{2}{W}{pawn}{png}
\pawn{2}{4}{W}{bishop}{png}
\pawn{3}{0}{W}{tower}{png}
\pawn{4}{2}{W}{king}{png}
\pawn{6}{3}{W}{tower}{png}

\pawn{0}{6}{B}{pawn}{png}
\pawn{0}{7}{B}{tower}{png}
\pawn{1}{4}{B}{king}{jpg}
\pawn{2}{7}{B}{bishop}{jpg}
\pawn{6}{2}{B}{knight}{png}
\end{tikzpicture}
}
\end{center}

\begin{solution}
  If we use the same strategy as for the previous exercise, starting with Rd6, the Black player can move the knight from g3 to f5.
  This is a double attack against the rook and the king. By saving the king, we then loose the rook!
  We can assume that our son won't thing about moving the knight but if we play with a grand chess master, we need 6 moves as shown below.

  \url{chess.com} reports Mate in 6 (Stockfish.jl | depth 75).
  It went deeper than in the previous exercise for the same time limit (1 min) because less pieces are left.

\begin{enumerate}
\item Bc4+ Kc6
\item Rg6+ Kb7
\item Rg7+ Bd7
\item Rdxd7+ Kc8
\item Ba6+ Kb8
\item Bd6\#
\end{enumerate}
\end{solution}

\section{}
Your pirate crew just stole a purse containing 10 gold pieces. It must now be decided how the gold will be divided between you. Your crew consists of five pirates, in hierarchical order: the Captain, the Ship Master, the Navigator, the Mate and the Cabin Boy. Luckily, to avoid arguments, there exists a longstanding traditional pirate code on how treasure should be divided.

First, the most highly ranked pirate proposes a particular division of the treasure. Then, all pirates (including the one who proposed the split) vote on whether to accept or reject this division. If a strict majority votes in favor, then the treasure is divided as suggested. Otherwise, the pirate who proposed the division is thrown overboard and the next most senior pirate gets a chance to propose an alternative. This process continues, until either a split is accepted or only the cabin boy remains.

Now, you know the crew. They are a highly intelligent and ferociously logical bunch. If their life is not threatened, they always act to maximize their gains. And of course, since they are pirates, if they get the chance to throw one of their superior overboard, they will favor this option provided it does not affect their gold revenue.
Moreover, if they have the choice between two divisions of gold with the same amount for them, they will choose the one that gives the highest amount to the highest ranked pirates.

According to pirate's traditions, how will the gold be divided?

\textbf{Challenge:} What would happen if there were $50$ pirates, again with a strict hierarchy? And with $n$ pirates?

\begin{solution}
\input{0103}
\end{solution}

\section{}
You are playing \emph{Rock, Paper, Scissors}. What is your strategy to win the game?

\begin{solution}
With some intuition, we can guess that the optimal strategy when playing \textit{Rock, Paper, Scissors} is to play \textit{rock}, \textit{paper} or \textit{scissors} with probability 1/3 each. But, since we are in a Game Theory course, we are going to formalize this intuition in terms of Nash equilibrium.

At first, we are going to model the \textit{Rock, Paper, Scissors} game in strategic form. We consider a payoff of $1$ when winning the game, a payoff of $0$ in case of a tie and a payoff of $-1$ when we loose the game.
\begin{center}
\begin{tabular}[h!]{l|ccccccc}
	&& \Large{Rock} && \Large{Paper} && \Large{Scissors} & \\
	\hline
	\Large{rock} && \Large{0,0} && \Large{-1,1} && \Large{1,-1} & \\
	\Large{paper} && \Large{1,-1} && \Large{0,0} && \Large{-1,1} & \\
	\Large{scissors} && \Large{-1,1} && \Large{1,-1} && \Large{0,0} & \\
\end{tabular}
\end{center}

Then, in order to find the optimal strategy of the game, we are going to find the (Nash) equilibria of the game.

The first questions we can ask us is whether there exist some pure equilibria (using pure strategies) ? This question is relatively easy to answer. Indeed, if the opponent plays \textit{rock} the optimal strategy for me is to play \textit{paper}. But if I play \textit{paper}, the optimal strategy for my opponent is to play \textit{scissors} and if he plays \textit{scissors}, my optimal strategy is to play \textit{rock} and so one... (we can see that \textit{rock} is best response to \textit{scissors} while \textit{scissors} is best response to \textit{paper} and \textit{paper} is best response to \textit{rock}). So we can conclude that there isn't some pure equilibria.

In a similar way, we can prove that there does not exist any randomized equilibria for the following cases : 
\begin{enumerate}
\item pure strategy vs. randomized strategy between two moves
\item pure strategy vs. randomized strategy between three moves
\item randomized strategy between two moves vs. randomized strategy between two moves
\item randomized strategy between two moves vs. randomized strategy between three moves
\end{enumerate}

Finally, the last possibility is that both players play a randomized strategy between the three moves (\textit{rock}, \textit{paper} and \textit{scissors}). Since the game is symmetric, we expect that the strategy will be the same for both player. We suppose that a player play rock with probability $\alpha$, paper with probability $\beta$ and scissors with probability $\gamma$. 
We need 
\[
\alpha + \beta + \gamma = 1.
\]
Moreover we expect
\[
-\beta + \gamma = \alpha - \gamma = - \alpha + \beta.
\]
We therefore find 
\[
\beta = \gamma = \alpha = \dfrac{1}{3}
\]
and so the only one Nash equilibrium of this game is 
\[
\left(\dfrac{1}{3}[\text{rock}] + \dfrac{1}{3}[\text{paper}] + \dfrac{1}{3}[\text{scissors}], \dfrac{1}{3}[\text{Rock}] + \dfrac{1}{3}[\text{Paper}] + \dfrac{1}{3}[\text{Scissors}]\right).
\]

\end{solution}

\section{}
You are playing the following Prisoner's Dilemma.
\begin{center}
	\begin{tabular}[h!]{l|ccccc}
		&& \Large{Cooperate} && \Large{Defect} & \\
		\hline
		\Large{Cooperate} && \Large{6,6} && \Large{1,7} & \\
		\Large{Defect} && \Large{7,1} && \Large{2,2} &
	\end{tabular}
\end{center}
Would you take the ``Cooperate'' option? Would it help to communicate to improve your payoff?

\begin{solution}
\paragraph{1.} No. For both players (the game is symmetric), strategy \textit{Cooperate} is strongly dominated by strategy \textit{Defect}. Indeed, if the other player plays \textit{Cooperate}, then I can either win 6 by playing \textit{Cooperate} or 7 by playing \textit{Defect}. If the other player plays \textit{Defect}, I can either win 1 by playing \textit{Cooperate} or 2 by playing \textit{Defect}. Therefore, I will always win more by playing \textit{Defect} and so it is the optimal strategy.

\paragraph{2.} We can improve our payoff if we communicate with the other player. Indeed, for example, you could say to the other player :
\emph{Whatever you do, I will play "Defect" but if you play "Cooperate" I will give you 3.}
\end{solution}

\section{}
How many equilibria\footnote{An equilibrium is a strategy profile for all players such that neither of them would gain from deviating alone from their strategy.} can you find in the following game?
\begin{center}
\begin{tabular}[h!]{l|ccccc}
	&& \Large{$x_2$} && \Large{$y_2$} & \\
	\hline
	\Large{$x_1$} && \Large{2,2} && \Large{4,1} & \\
	\Large{$y_1$} && \Large{1,4} && \Large{5,5} &
\end{tabular}
\end{center}

\begin{solution}
In order to answer this question, we are going to decompose our reasoning into three steps.
First of all, we will consider that both players have a pure strategy. Then, that one player has a pure strategy and the other one has a randomized strategy. Finally, we will assume that both players have a randomized strategy.

\subsection*{Pure strategy vs. pure strategy}

We can directly see that $(x_1, y_2)$ is not an equilibrium. Indeed, both players can increase their payoff from deviating alone from their strategy. Player 2 will gain 2 instead of 1 if he deviates alone to $x_2$ and player 1 will gain 5 instead of 4 if he deviates alone to $y_1$.

Since the game is symmetric, $(y_1, x_2)$ is not an equilibrium either.

After that we can see that both $(x_1, x_2)$ and $(y_1, y_2)$ are pure equilibria.

Indeed, in both cases, none of the player can gain more from deviating alone from their strategy :
\begin{itemize}
	\item If we consider equilibrium $(x_1, x_2)$ and that one player deviates to $y$, he will win 1 instead of 2
	\item If we consider equilibrium $(y_1, y_2)$ and that one player deviates to $x$, he will win 4 instead of 5
\end{itemize}

\subsection*{Pure strategy vs. randomized strategy}

Now, we can assume, without loss of generality (since the game is symmetric), that player 1 has a pure strategy (either $x_1$ or $y_1$) and that player 2 has a randomized strategy (between $x_2$ and $y_2$). In these conditions, it is impossible to find an equilibrium. Indeed, if the pure strategy of player 1 is to play $x_1$, the optimal strategy for player 2 will be to play $x_2$ (so it won't be a randomized strategy). If the pure strategy of player 1 is to play $y_1$, the optimal strategy for player 2 will be to play $y_2$ which is neither a randomized strategy.

\subsection*{Randomized strategy vs. randomized strategy}

For this last case, we consider that player 1 plays $x_1$ with probability $\alpha$ and $y_1$ with probability $(1-\alpha)$. Player 2 plays $x_2$ with probability $\beta$ and $y_2$ with probability $(1-\beta)$. Since the game is symmetric we can expect that if an equilibrium exists, we will have $\alpha = \beta$.

Since we want that none of the two players would gain from deviating alone from their strategy, we expect that :
\begin{align*}
w_1 = 2\beta + 4(1-\beta) = 1\beta + 5(1-\beta) \\
w_2 = 2\alpha + 4(1-\alpha) = 1\alpha + 5(1-\alpha)
\end{align*}
we find therefore, $\alpha = \beta = \dfrac{1}{2}$ and we have therefore the following Nash equilibrium : $(\dfrac{1}{2}x_1+\dfrac{1}{2}y_1, \dfrac{1}{2}x_2+\dfrac{1}{2}y_2)$.

\subsection*{Conclusion}

Finally, we can conclude that there exist three equilibria in this game (two of which are pure equilibria).
\begin{enumerate}
	\item $(x_1, x_2)$
	\item $(y_1, y_2)$
	\item $(\dfrac{1}{2}x_1+\dfrac{1}{2}y_1, \dfrac{1}{2}x_2+\dfrac{1}{2}y_2)$
\end{enumerate}

\end{solution}

\section{}
In the Prisoner's Dilemma from exercise 5., suppose you had to repeat the game 10 times. Your strategy is the following: You always play ``Cooperate'', except if your opponent played ``Defect'' on the previous round. Discuss how this procedure helps to improve your payoff in the long run. Can you imagine other procedures?

\nosolution

\section{}
You are playing the following game.

\begin{center}
\begin{tabular}[h!]{l|ccccc}
	&& \Large{$x_2$} && \Large{$y_2$} & \\
	\hline
	\Large{$x_1$} && \Large{2,2} && \Large{6,3} & \\
	\Large{$y_1$} && \Large{3,6} && \Large{5,5} &
\end{tabular}
\end{center}

\begin{itemize}
\item Can you find an equilibrium in this game?
\item Imagine that some neutral player told you and your opponent the following: \emph{``Let me choose what to play for you. I will secretly choose one of the strategies $(x_1, y_2)$, $(y_1, x_2)$ and $(y_1, y_2)$ with equal probability, but I promise,  never $(x_1, x_2)$. Then I will privately communicate to both of you the action that you should play.''} Would this guy's advice result in a better payoff for you? And would you follow his advice?
\end{itemize}

%Assume a mediator comes in and tells you the following: \emph{``I will secretly choose an action for you, either $x$ or $y$, each with probability $1/2$ and tell you each, privately, what to play.''} If you followed his instruction, would it result in a better payoff? But should you follow the mediator's instructions in the first place?

\nosolution

\end{document}
