With some intuition, we can guess that the optimal strategy when playing \textit{Rock, Paper, Scissors} is to play \textit{rock}, \textit{paper} or \textit{scissors} with probability 1/3 each. But, since we are in a Game Theory course, we are going to formalize this intuition in terms of Nash equilibrium.

At first, we are going to model the \textit{Rock, Paper, Scissors} game in strategic form. We consider a payoff of $1$ when winning the game, a payoff of $0$ in case of a tie and a payoff of $-1$ when we lose the game.
\begin{center}
\begin{tabular}[h!]{l|ccccccc}
	&& \Large{Rock} && \Large{Paper} && \Large{Scissors} & \\
	\hline
	\Large{rock} && \Large{0,0} && \Large{$-1$,1} && \Large{1,$-1$} & \\
	\Large{paper} && \Large{1,$-1$} && \Large{0,0} && \Large{$-1$,1} & \\
	\Large{scissors} && \Large{$-1$,1} && \Large{1,$-1$} && \Large{0,0} & \\
\end{tabular}
\end{center}

Then, in order to find the optimal strategy of the game, we are going to find the (Nash) equilibria of the game.

The first questions we can ask us is whether there exist some pure equilibria (using pure strategies) ? This question is relatively easy to answer. Indeed, if the opponent plays \textit{rock} the optimal strategy for me is to play \textit{paper}. But if I play \textit{paper}, the optimal strategy for my opponent is to play \textit{scissors} and if he plays \textit{scissors}, my optimal strategy is to play \textit{rock} and so one... (we can see that \textit{rock} is best response to \textit{scissors} while \textit{scissors} is best response to \textit{paper} and \textit{paper} is best response to \textit{rock}). So we can conclude that there isn't any pure equilibria.

In a similar way, we can prove that there does not exist any randomized equilibria for the following cases : 
\begin{enumerate}
\item pure strategy vs. randomized strategy between two moves
\item pure strategy vs. randomized strategy between three moves
\item randomized strategy between two moves vs. randomized strategy between two moves
\item randomized strategy between two moves vs. randomized strategy between three moves
\end{enumerate}

Finally, the last possibility is that both players play a randomized strategy between the three moves (\textit{rock}, \textit{paper} and \textit{scissors}). Since the game is symmetric, we expect that the strategy will be the same for both player. We suppose that a player play rock with probability $\alpha$, paper with probability $\beta$ and scissors with probability $\gamma$. 
We need 
\[
\alpha + \beta + \gamma = 1.
\]
Moreover we expect
\[
-\beta + \gamma = \alpha - \gamma = - \alpha + \beta.
\]
We therefore find 
\[
\beta = \gamma = \alpha = \dfrac{1}{3}
\]
and so the only one Nash equilibrium of this game is 
\[
\left(\dfrac{1}{3}[\text{rock}] + \dfrac{1}{3}[\text{paper}] + \dfrac{1}{3}[\text{scissors}], \dfrac{1}{3}[\text{Rock}] + \dfrac{1}{3}[\text{Paper}] + \dfrac{1}{3}[\text{Scissors}]\right).
\]
