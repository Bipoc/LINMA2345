With the knowledge of the previous courses, we know that we are tempted to randomize the strategies, that is, choose parameters $\lambda$ and $\mu$, then play with strategies $\parent{\lambda a_1, \parent{1 - \lambda} b_1}$ for player 1 and $\parent{\mu a_2, \parent{1 - \mu} b_2}$ for player 2.
However, now we will randomize over the outcomes, hence we will choose parameters $\alpha$, $\beta$, $\gamma$ and $\delta$ for outcomes of $\parent{a_1, a_2}$, $\parent{a_1, b_2}$, $\parent{b_1, a_2}$ and $\parent{b_1, b_2}$, respectively.
The optimization problem is
\begin{equation*}
    \begin{aligned}
    \underset{}{\text{max }} &
    f \parent{\alpha, \beta, \gamma, \delta} & & \\
    \text{such that  } &
        \alpha, \beta, \gamma, \delta \geq 0 & \\
        & \alpha + \beta + \gamma + \delta = 1 & \\
        & \alpha, \beta, \gamma, \delta \in C . &
    \end{aligned}
    \end{equation*}

The set $C$ is the set containing all the parameters such that the equilibrium is \textit{correlated}. If we look at the table, we see that the game is symmetric.
We can prove that the equilibrium is invariant under transformation, since the set of feasible solutions is convex. Hence we can say that $\beta = \gamma$. We have thus transformed a four-dimensional problem into a three-dimensional problem.
Furthermore, we can delete the parameter $\delta$ because $\alpha + 2 \beta + \delta = 1$, hence $\delta = 1 - \alpha - 2 \beta$. We have thus transformed a three-dimensional problem into a two-dimensional problem.

Now let us try to find the function $f$. We have to maximize the sum of the expected payoffs. Since the game is symmetric, we compute the expected payoff of player 1 and we multiply it by two. We find
\begin{equation*}
    \mathbb{E} \squared{\text{payoff of player 1}}
    = 4 \alpha + \beta + 6 \beta - 3 \delta
    = 7 \alpha + 13 \beta - 3
\end{equation*}

The objective function is thus
\begin{equation*}
    f
    = f \parent{\alpha, \beta, \gamma, \delta}
    = f \parent{\alpha, \beta}
    = 2 \cdot \mathbb{E} \squared{\text{payoff of player 1}}
    = 14 \alpha + 26 \beta - 6
\end{equation*}



Now we have to define the set $C$. We work with the formulas given in the reminder
\begin{equation}
\label{eq:constr}
	\sum_{c_{-i} \in C_{-i}} \mu(c) \, \big( u_i(c_{-i}, \; c_i) - u_i(c_{-i}, \; e_i) \big) \geq 0, \quad \forall i \in N, \ \forall c_i \in C_i, \ \forall e_i \in C_i,
\end{equation}

where $c = (c_{-i}, c_i)$.

Let us start with player 1, i.e., $i=1$. Of course, we don't have to consider the cases where $c_1 = e_1$, because in that case the left-hand side of constraint (\ref{eq:constr}) is zero, and the constraint becomes $0 \geq 0$, which is not very interesting. Hence, we only have two cases
\begin{enumerate}
    \item We have $c_1 = a_1$ and $e_1 = b_1$. The constraint becomes

    \begin{align*}
    & \sum_{c_{2} \in C_{2}} \mu(c_2,a_1) \parent{ u_1(c_{2}, a_1) - u_1(c_{2}, b_1) } \geq 0 \\
    & \Longleftrightarrow
    \mu(a_2,a_1) \parent{ u_1(a_{2}, a_1) - u_1(a_{2}, b_1) }
    +
    \mu(b_2,a_1) \parent{ u_1(b_{2}, a_1) - u_1(b_{2}, b_1) }
    \geq 0 \\
    & \Longleftrightarrow
    \alpha \parent{ 4 - 6 }
    +
    \beta \parent{ 1 - (-3) }
    \geq 0 \\
    & \Longleftrightarrow
    4 \beta - 2 \alpha
    \geq 0 \\
    & \Longleftrightarrow
    2 \beta \geq \alpha  \\
    \end{align*}

    \item We have $c_1 = b_1$ and $e_1 = a_1$. The constraint becomes

    \begin{align*}
    & \sum_{c_{2} \in C_{2}} \mu(c_2,b_1) \parent{ u_1(c_{2}, b_1) - u_1(c_{2}, a_1) } \geq 0 \\
    & \Longleftrightarrow
    \mu(a_2,b_1) \parent{ u_1(a_{2}, b_1) - u_1(a_{2}, a_1) }
    +
    \mu(b_2,b_1) \parent{ u_1(b_{2}, b_1) - u_1(b_{2}, a_1) }
    \geq 0 \\
    & \Longleftrightarrow
    \beta \parent{ 6 - 4 }
    +
    \delta \parent{ - 3 - 1 }
    \geq 0 \\
    & \Longleftrightarrow
    2 \beta
    -
    4 \parent{ 1 - \alpha - 2 \beta}
    \geq 0 \\
    & \Longleftrightarrow
    4 \alpha + 10 \beta - 4
    \geq 0 \\
    & \Longleftrightarrow
    2 \alpha + 5 \beta \geq 2 \\
    \end{align*}

\end{enumerate}

By symmetry, if we follow the same reasoning for player 2, we will find the same constraints. However, before defining the final optimization problem, we cannot forget the extra constraint: $\delta \geq 0$. Indeed, since we work with variables $\alpha$ and $\beta$, this constraint becomes $1 - \alpha - 2 \beta \geq 0 $.
Hence we have the following optimization problem

\begin{equation*}
    \begin{aligned}
    \underset{}{\text{max }} &
    14 \alpha + 26 \beta - 6 & & \\
    \text{such that  } &
        \alpha, \beta \geq 0 & \\
        & 2 \beta \geq \alpha & \\
        & 2 \alpha + 5 \beta \geq 2 & \\
        & 1 \geq \alpha + 2 \beta. &
    \end{aligned}
    \end{equation*}


We solve this problem graphically. On Figure \ref{fig:niceGraph}, we can see that we have four extreme points. Let us give color-labels to the constraints.
\begin{enumerate}
    \item The inequality constraint $2 \beta \geq \alpha$ is the \textbf{green} line, and looking at the sign of the inequality, we have to be above the green line.
    \item The inequality constraint $2 \alpha + 5 \beta \geq 2$ is the \textbf{red} line, and looking at the sign of the inequality, we have to be above the red line.
    \item The inequality constraint $1 \geq \alpha + 2 \beta$ is the \textbf{blue} line, and looking at the sign of the inequality, we have to be below the blue line.
\end{enumerate}

\begin{center}
    \begin{center}
        \begin{tikzpicture}[scale=7]
        % Axis
        \draw[axis] (0,0) -- (1.2,0) node[right=\nudge cm] {\(\alpha\)};
        \draw[axis] (0,0) -- (0,0.7) node[above=\nudge cm] {\(\beta\)};
        \begin{scope}
          % Avoid going too far
          \clip (-\nudge,-\nudge) rectangle (2+\nudge,2+\nudge);
          \draw[ineq1] (0,0) -- (1,0.5) coordinate (ineq1);
          \draw[ineq2] (0,2/5) -- (1,0) coordinate (ineq2);
          \draw[eq] (0,0.5) -- (1,0) coordinate (eq1);
         \begin{scope}
         \end{scope}
        \end{scope}
        \foreach \coord/\adj in {
          %\node[left] {A} at (0,1) "ultra thick point"
          %{(0,0)}/left,
          {(1,0.5)}/above,
          {(0,2/5)}/left,
          {(0,0.5)}/left,
          {(1,0)}/below,%
          %{(8/9,0)}/below,
          %{(2,0)}/below%
        } {
          \fill \coord circle (0.3pt) node[\adj] {$\coord$};
        }
        \foreach \coord/\adj/\name in {
          {(2/4,1/4)}/above/A,
          {(4/9,2/9)}/below/B,
          {(0,2/5)}/below right/C,
          {(0,0.5)}/above right/D%
        } {
          \fill \coord circle (0.3pt) node[\adj] {$\name$};
        }
        \end{tikzpicture}
      \end{center}
    \captionof{figure}{Solving the problem graphically}
    \label{fig:niceGraph}
\end{center}




To find the correlated equilibrium that maximizes (resp. minimizes) the expected sum of payoffs of the two players, to answer question a (resp. b), we can draw the line $14 \alpha + 26 \beta - 6$ and find the point where it is the greatest (resp. lowest). However, we can also compute the objective value for each of the four extreme points.

\begin{enumerate}
    \item Point $A$, the intersection of the green and the blue lines. Hence the coordinates of point $A$ satisfy
    \begin{equation*}
       \begin{cases}
        2 \beta = \alpha  \\
        1 = \alpha + 2 \beta
        \end{cases}
    \end{equation*}

    We find $\parent{\alpha, \beta} = \parent{\dfrac{1}{2}, \dfrac{1}{4}}$. The objective value is $7.5$.

    \item Point $B$, the intersection of the green and the red lines. Hence the coordinates of point $A$ satisfy
    \begin{equation*}
       \begin{cases}
        2 \beta = \alpha  \\
        2 \alpha + 5 \beta = 2
        \end{cases}
    \end{equation*}

    We find $\parent{\alpha, \beta} = \parent{\dfrac{4}{9}, \dfrac{2}{9}}$. The objective value is $6$.

    \item Point $C$, we have $\parent{\alpha, \beta} = \parent{0, \dfrac{2}{5}}$. The objective value is $4.4$.

    \item Point $D$, we have $\parent{\alpha, \beta} = \parent{0, \dfrac{1}{2}}$. The objective value is $7$.
\end{enumerate}

On the one hand, the point that maximizes the expected sum of payoffs of the two players is point $A$.
The associated correlated equilibrium is defined by the variables $\alpha = \dfrac{1}{2}$, $\beta = \dfrac{1}{4}$, $\gamma = \dfrac{1}{4}$ and $\delta = 0$. The fact that $\delta = 0$ coincides with our intuition.
We find 
\begin{align*}
    U_1 \parent{\mu}
    = \sum_{c \in C} \mu \parent{c} \cdot u_{1} \parent{c}
\end{align*}

with $c = \parent{c_1, c_2}$ and $C = C_1 \times C_2 = \bracket{a_1, b_1} \times \bracket{a_2, b_2}$. We obtain
\begin{align*}
    U_1 \parent{\mu}
    &= \mu \parent{a_1, a_2} \cdot u_{1} \parent{a_1, a_2}
    + \mu \parent{a_1, b_2} \cdot u_{1} \parent{a_1, b_2} \\
    &+ \mu \parent{b_1, a_2} \cdot u_{1} \parent{b_1, a_2}
    + \mu \parent{b_1, b_2} \cdot u_{1} \parent{b_1, b_2} \\
    &= \alpha \cdot 4 + \beta \cdot 1 + \gamma \cdot 6 + \delta \cdot \parent{-3} \\
    &= 2 + \dfrac{1}{4} + \dfrac{3}{2} + 0
    = \dfrac{15}{4} = 3.75.
\end{align*}

By symmetry, we also have $U_2 \parent{\mu} = 3.75$.
As we have seen above, the objective value is $z = 7.5$.


\vspace{5mm}


On the other hand, the point that minimizes the expected sum of payoffs of the two players is point $C$.
The associated correlated equilibrium is defined by the variables $\alpha = 0$, $\beta = \dfrac{2}{5}$, $\gamma = \dfrac{2}{5}$ and $\delta = \dfrac{1}{5}$. The fact that $\alpha = 0$ coincides with our intuition. We would like to see $\delta = 1$ but this is not possible since we can only consider correlated equilibria.
We find 
\begin{align*}
    U_1 \parent{\mu}
    &= \alpha \cdot 4 + \beta \cdot 1 + \gamma \cdot 6 + \delta \cdot \parent{-3} \\
    &= 0 + \dfrac{2}{5} + \dfrac{12}{5} - \dfrac{3}{5}
    = \dfrac{11}{5} = 2.2.
\end{align*}

By symmetry, we also have $U_2 \parent{\mu} = 2.2$.
As we have seen above, the objective value is $z = 4.4$.

