We notice that the game is invariant under cyclic permutation of the actions $x, y, z$.
That is, if we apply the same number of cyclic permutation to the actions $x_1, y_1, z_1$ and $x_2, y_2, z_2$, we payoff matrix is the same.
Moreover, the game is invariant under a permutation of the two players.
As optimizing a linear objective over the set of correlated equilibria is a convex program, if the objective is also invariant under one of these transformation, there exists an optimal solution which is invariant under this transformation.
For a., the objective is invariant under both transformations and for b., it is only invariant under cyclic permutations.

Therefore, for a., we can consider correlated strategies $\mu$ of the form:
\[
  \mu = \begin{pmatrix}
    \gamma & \alpha & \alpha\\
    \alpha & \gamma & \alpha\\
    \alpha & \alpha & \gamma
  \end{pmatrix}
\]
and for b. we can consider correlated strategies $\mu$ of the form:
\[
  \mu = \begin{pmatrix}
    \gamma & \alpha & \beta\\
    \beta  & \gamma & \alpha\\
    \alpha & \beta  & \gamma
  \end{pmatrix}.
\]

\subsection*{a.}
It suffices to take $\gamma = 0$, hence $\alpha = 1/6$ since $6\alpha + 3\gamma = 1$.

We have to check if it is a correlated equilibrium. Recall that a correlated equilibrium for a game in strategic form must satisfy the following strategic incentive constraints :
\begin{align*}
  \sum_{c_{-i} \in C_{-i}} \mu(c) \, \big( u_i(c_{-i}, \; c_i) - u_i(c_{-i}, \; e_i) \big) \geq 0, \quad \forall i \in N, \ \forall c_i \in C_i, \ \forall e_i \in C_i,
\end{align*}
where $c = (c_{-i}, c_i)$, $c_{-i}$ being the strategy of the players other than $i$. These constraints require that the players cannot improve their payoff by deviating from the correlated strategy $\mu$ suggested by the mediator.

As the game is symmetric under the permutation of players and under cyclic permutations of actions, we only need to check the following two cases\footnote{Indeed, for instance the case $i=1$, $c_1 = y_2$, $e_1 = x_2$ is obtained from the case $i=2, c_2 = x_2, e_2 = z_2$ for which we apply a player permutation and a cyclic action permutation}:
\begin{itemize}
  \item $i=2$ $c_2=x_2$  $e_2=y_2$
    \begin{equation*}
      \frac{1}{6}(5-0) + \frac{1}{6}(4-5) = \frac{2}{3} \ge 0
    \end{equation*}
  \item $i=2$ $c_2=x_2$  $e_2=z_2$
    \begin{equation*}
      \frac{1}{6}(5-4) + \frac{1}{6}(4-0) = \frac{5}{6} \ge 0
    \end{equation*}
\end{itemize}

This means that the players 2 has no interest in playing $y_2$ or $z_2$ rather than $x_2$. And so it is a correlated equilibrium and the payoff equal to $\frac{27}{6} = 4.5 \ge 4$

\subsection*{b.}

Note that we can not put $\gamma$ to zero even if its associated payoff is zero.
Indeed taking a non zero value for $\gamma$ can increase the value of $\alpha$ or $\beta$ and so we can have a possible better payoff.

The solution $\mu$ is not invariant under player permutation but both $\mu$ and the game are invariant under cyclic permutation of the actions.
Therefore, it is sufficient to check the following cases:

\begin{itemize}
  \item $i=1$ $c_1=x_1$  $e_1=y_1$
    \begin{equation*}
      \gamma(0-4) + \alpha(5-0) + \beta(4-5) = -4\gamma + 5\alpha - \beta \ge 0
    \end{equation*}
  \item $i=1$ $c_1=x_1$  $e_1=z_1$
    \begin{equation*}
      \gamma(0-5) + \alpha((5-4) + \beta(4-0) = -5\gamma + \alpha + 4\beta \ge 0
    \end{equation*}

  \item $i=2$ $c_2=x_2$  $e_2=y_2$
    \begin{equation*}
      \gamma(0-4) + \alpha(4-5) + \beta(5-0) = -4\gamma - \alpha + 5\beta \ge 0
    \end{equation*}
  \item $i=2$ $c_2=x_2$  $e_2=z_2$
    \begin{equation*}
      \gamma(0-5) + \alpha((4-0) + \beta(5-4) = -5\gamma + 4\alpha + \beta \ge 0
    \end{equation*}
\end{itemize}

Then we compute the payoff of the player 2 :
\begin{equation*}
    U_2 = (4+4+4)\alpha + (5+5+5)\beta = 12\alpha + 15\beta
\end{equation*}

Finally we have a maximization problem under strategic incentive constraints :
\begin{align*}
    \max &\  12\alpha + 15\beta \\
      s.t& \\
      -& 4\gamma + 5\alpha - \beta \ge 0 \\
      -& 5\gamma + \alpha + 4\beta \ge 0 \\
      -& 4\gamma -\alpha + 5\beta \ge 0 \\
      -& 5\gamma + 4\alpha + \beta \ge 0 \\
      & 3\alpha + 3\beta + 3\gamma = 1 \\
      & 0 \le \alpha,\beta,\gamma \le \frac{1}{3}
\end{align*}

We can solve the problem using a graph taking $\gamma=\frac{1}{3}-\alpha-\beta$ for example or we can simply see that it's more interesting to put all in $\beta$ and so the only conditions constraining are $-4\gamma + 5\alpha - \beta \ge 0$ and $3\alpha + 3\beta + 3\gamma = 1$. We find $\alpha=\frac{1}{18}$, $\beta=\frac{5}{18}$ and $\gamma=0$. So the payoff equal to :
\begin{equation*}
    U_2 = \frac{12}{18}+\frac{75}{18} = \frac{29}{6} \approx 4.834
\end{equation*}
