\documentclass{../ape}

\usepackage{../../linma2345}

\begin{document}

\session{8}{Correlated Equilibria}

\section{}
Consider the following game in strategic form:

	\begin{center}
		\begin{tabular}[h!]{l|ccccc}
			&& $a_2$ && $b_2$ & \\
			\hline
			$a_1$ && $4,4$ && $1,6$ & \\
			$b_1$ && $6,1$ && $-3,-3$ & 
		\end{tabular} 
	\end{center}

\begin{enumerate}
	\item[a.] Find the correlated equilibrium that maximizes the expected sum of payoffs of the two players.
	\item[b.] Also find the one that minimizes the expectation of that sum.
\end{enumerate}

\section{}
Sender-receiver games are a special case of Bayesian games with communication in which Player~1 (the sender) has a hidden type, but has no action to take, while player~2 (the receiver) has no type, but chooses an action from a set $C_2$.

\begin{enumerate}
	\item[a.] Write the expression that allows to compute the expected payoff of each player.
	\item[b.] Write the strategic incentive constraints and simplify them as much as possible.
\end{enumerate}

Now, consider that the payoff matrix for this sender-receiver game is given by the following table:

\begin{center}
	\begin{tabular}[h!]{l|ccccccc}
		&& $x_2$ && $y_2$ && $z_2$ & \\
		\hline
		$1.a$ && $2,3$ && $0,2$ && $-1,0$ & \\
		$1.b$ && $1,0$ && $2,2$ && $0,3$ & 
	\end{tabular} 
\end{center}
	
We will assume that $p(t_1 = a) = p(t_1 = b) = 1/2$.
	
\begin{enumerate}
	\item[c.] In the situation where there is no mediator, show that whatever the message sent by Player~1, Player~2 chooses action $y_2$. From there, deduce upper bounds for the achievable expected payoffs for the two players with a Nash equilibrium of this game without communication.
	\item[d.] Still without a mediator, assume now that Player~1 communicates with Player~2 through a noisy channel: he can send a signal (which we denote ``$m$'') to indicate that his type is $a$, but there is only one chance over two that Player~2 actually receives it. (We assumed that if Player~1 has type $b$, then he does not send any signal.) Show that this situation actually allows the players to improve their payoff. Explain this paradox in terms of correlated equilibria. 
	\item[e.] Show that the following correlated equilibrium mechanism maximizes the payoff of Player~2:
		\begin{alignat*}{3}
			& \mu(x_2 | 1.a) = 2/3, \quad && \mu(y_2 | 1.a) = 1/3, \quad && \mu(z_2 | 1.a) = 0 \\
			& \mu(x_2 | 1.b) = 0,   \quad && \mu(y_2 | 1.b) = 2/3, \quad && \mu(z_2 | 1.b) = 1/3.
		\end{alignat*}
\end{enumerate}

\section{}
In the exercise session on Nash equilibria, we found that the following two-player game allows a unique Nash equilibrium in which each player has an expected payoff of~3:

	\begin{center}
		\begin{tabular}[h!]{l|ccccccc}
			&& $x_2$ && $y_2$ && $z_2$ & \\
			\hline
			$x_1$ && $0,0$ && $5,4$ && $4,5$ & \\
			$y_1$ && $4,5$ && $0,0$ && $5,4$ & \\
			$z_1$ && $5,4$ && $4,5$ && $0,0$ & 
		\end{tabular} 
	\end{center}
	
\begin{enumerate}
	\item[a.] Show that this game allows a correlated equilibrium for which each player receives a payoff higher than~4.
	\item[b.] Find the correlated equilibrium that maximizes the expected payoff of Player~2.
\end{enumerate}

\thex{}
Let $u = \big( u_1, u_2 \big)^T$ and $v = \big( v_1, v_2 \big)^T$ be payoff vectors that correspond to two different correlated equilibria of a given game $\Gamma^c$. Show that for this game, there also exists a correlated equilibrium that yields the payoff vector $w = \big( \frac{1}{3} u_1 + \frac{2}{3} v_1, \, \frac{1}{3} u_2 + \frac{2}{3} v_2 \big)^T$. 

\newpage

\section*{Reminders}

\begin{itemize}[leftmargin=*]
\renewcommand{\labelitemi}{$\bullet$}

	\item \textbf{Correlated equilibria for games in strategic form}
	\vspace{.3cm}

	\begin{itemize}

		\item The payoff of player $i$, according to the correlated strategy $\mu$ proposed by a mediator, is given by:
		\begin{align*}
			U_i(\mu) = \sum_{c \in C} \mu(c) \, u_i(c),
		\end{align*}
		where $C$ is the set of possible combinations of pure strategies.
			
		\item A correlated equilibrium for a game in \emph{strategic form} must satisfy the following strategic incentive constraints:
		\begin{align*}
			\sum_{c_{-i} \in C_{-i}} \mu(c) \, \big( u_i(c_{-i}, \; c_i) - u_i(c_{-i}, \; e_i) \big) \geq 0, \quad \forall i \in N, \ \forall c_i \in C_i, \ \forall e_i \in C_i,
		\end{align*}
		where $c = (c_{-i}, c_i)$, $c_{-i}$ being the strategy of the players other than $i$. These constraints require that the players cannot improve their payoff by deviating from the correlated strategy $\mu$ suggested by the mediator.
			
	\end{itemize}
	\vspace{.3cm}

	\item \textbf{Correlated equilibria for Bayesian games}
	\vspace{.3cm}

	\begin{itemize}

		\item The payoff of player $i$, according to the correlated strategy $\mu$ and give his type $t_i$, is give by:
		\begin{align*}
			U_i(\mu \; | \; t_i) = \sum_{t_{-i} \in T_{-i}} \sum_{c \in C} p_i(t_{-i} \; | \; t_i) \, \mu(c \; | \; t) \, u_i(c, \; t),
		\end{align*}
		where $t = (t_{-i}, t_i)$ and $T_{-i}$ are the set of possible combinations of types for the  players other than $i$.
		
		\item A correlated equilibrium for a \emph{Bayesian} game must satisfy the following strategic incentive constraints:
		\begin{align*}
			U_i(\mu \; | \; t_i) \geq U_i^*(\mu, \; \delta_i, \; s_i \; | \; t_i), \quad \forall i \in N, \ \forall t_i \in T_i, \ \forall s_i \in T_i, \ \forall \delta_i : C_i \rightarrow C_i,
		\end{align*}
		with 
		\begin{align*}
			U_i^*(\mu, \; \delta_i, \; s_i \; | \; t_i) = \sum_{t_{-i} \in T_{-i}} \sum_{c \in C} p_i(t_{-i} \; | \; t_i) \, \mu(c \; | \; t_{-i}, s_i) \, u_i\big( \; (c_{-i}, \; \delta_i(c_i)) \;, \; t \big),
		\end{align*}
		where $t_i$ is the type of $i$, $s_i$ is the type that $i$ reveals to the mediator and $\delta_i$ is the function chosen by player $i$ that, for any action $c_i$ that can be suggested by the mediator, chooses an action $\delta_i(c_i)$ to play instead. These constraints impose that the players neither have interest to lie on their type, nor to deviate from the correlated strategy $\mu$ suggested by the mediator.
			
	\end{itemize}
	
	\vspace{.3cm}
	\item \textbf{Duality}
	\vspace{.3cm}
	
	Any Linear Programming problem $(P)$ (called \emph{primal}) is of the form:
	\begin{align*}
		\begin{array}{lll}
			\max && c^T x, \\ 
			\mbox{s.t.} && Ax \leq b, \\
			&& x \geq 0,
		\end{array}
	\end{align*} 
	can be transformed into the following problem $(D)$ (called \emph{dual}):
	\begin{align*}
		\begin{array}{lll}
			\min && b^T y, \\ 
			\mbox{s.t.} && A^T y \geq c, \\
			&& y \geq 0.
		\end{array}
	\end{align*} 
	Furthermore, we have that:
	\begin{itemize}
		\item $(P)$ allows an optimal solution $x^*$ iff $(D)$ allows an optimal solution $y^*$. Furthermore $c^T x^* = b^T y^*$;
		\item if $(P)$ (resp. $(D)$) is unbounded, then $(D)$ (resp. $(P)$) is impossible;
		\item $(P)$ et $(D)$ can also be both impossible.
	\end{itemize}
	\vspace{.5cm}
	Moreover, for the \emph{complementary slackness theorem}, solutions $x^*$ for $(P)$ and $y^*$ for $(D)$ are optimal iff:
	\begin{itemize}
		\item $y_i^* = 0 \quad$ for all $i$ such that the $i^{\text{th}}$ constraint in $(P)$ is not tight;
		\item $x_i^* = 0 \quad$ for all $i$ such that the $i^{\text{th}}$ constraint in $(D)$ is not tight.
	\end{itemize}
		
\end{itemize}

\end{document}










