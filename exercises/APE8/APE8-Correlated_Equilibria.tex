\documentclass{../ape}

\usepackage{../../linma2345}

\begin{document}

\session{8}{Correlated Equilibria}

\section{}
Consider the following game in strategic form:

	\begin{center}
		\begin{tabular}[h!]{l|ccccc}
			&& $a_2$ && $b_2$ & \\
			\hline
			$a_1$ && $4,4$ && $1,6$ & \\
			$b_1$ && $6,1$ && $-3,-3$ & 
		\end{tabular} 
	\end{center}

\begin{enumerate}
	\item[a.] Find the correlated equilibrium that maximizes the expected sum of payoffs of the two players.
	\item[b.] Also find the one that minimizes the expectation of that sum.
\end{enumerate}

\begin{solution}
With the knowledge of the previous courses, we know that we are tempted to randomize the strategies, that is, choose parameters $\lambda$ and $\mu$, then play with strategy $\parent{\lambda a_1, \parent{1 - \lambda} b_1}$ for player 1 and $\parent{\mu a_2, \parent{1 - \mu} b_2}$ for player 2. However, now we will randomize over the outcomes, hence we will choose parameters $\alpha$, $\beta$, $\gamma$ and $\delta$ for outcomes of $\parent{a_1, a_2}$, $\parent{a_1, b_2}$, $\parent{b_1, a_2}$ and $\parent{b_1, b_2}$, respectively. The optimization problem is

\begin{equation*}
    \begin{aligned}
    \underset{}{\text{max }} &
    f \parent{\alpha, \beta, \gamma, \delta} & & \\
    \text{such that  } &
        \alpha, \beta, \gamma, \delta \geq 0 & \\
        & \alpha + \beta + \gamma + \delta = 1 & \\
        & \alpha, \beta, \gamma, \delta \in C . &
    \end{aligned}
    \end{equation*}

The set $C$ is the set containing all the parameters such that the equilibrium is correlated. If we look at the table, we see that the game is symmetric. We can prove that the equilibrium is invariant under transformation, since the set of feasible solutions is convex. Hence we can say that $\beta = \gamma$. We have thus transformed a four-dimensional problem into a three-dimensional problem. Furthermore, we can delete the parameter $\delta$ because $\alpha + 2 \beta + \delta = 1$, hence $\delta = 1 - \alpha - 2 \beta$. We have thus transformed a three-dimensional problem into a two-dimensional problem.

Now let us try to find the function $f$. We have to maximize the sum of the expected payoffs. Since the game is symmetric, we compute the expected payoff of player 1 and we multiply it by two. We find

\begin{equation*}
    \mathbb{E} \squared{\text{payoff of player 1}}
    = 4 \alpha + \beta + 6 \beta - 3 \delta
    = 7 \alpha + 13 \beta - 3
\end{equation*}

The objective function is thus

\begin{equation*}
    f
    = f \parent{\alpha, \beta, \gamma, \delta}
    = f \parent{\alpha, \beta}
    = 2 \cdot \mathbb{E} \squared{\text{payoff of player 1}}
    = 14 \alpha + 26 \beta - 6
\end{equation*}



Now we have to define the set $C$. We work with the formulas given in the reminder

\begin{equation}
\label{eq:constr}
	\sum_{c_{-i} \in C_{-i}} \mu(c) \, \big( u_i(c_{-i}, \; c_i) - u_i(c_{-i}, \; e_i) \big) \geq 0, \quad \forall i \in N, \ \forall c_i \in C_i, \ \forall e_i \in C_i,
\end{equation}

where $c = (c_{-i}, c_i)$, $c_{-i}$.

Let us start with player 1, i.e., $i=1$. Of course, we don't have to consider the cases where $c_1 = e_1$, because in that case the left-hand side of constraint (\ref{eq:constr}) is zero, and the constraint becomes $0 \geq 0$, which is not very interesting. Hence, we only have two cases
\begin{enumerate}
    \item Case 1. We have $c_1 = a_1$ and $e_1 = b_1$. The constraint becomes
    
    \begin{align*}
    & \sum_{c_{2} \in C_{2}} \mu(c_2,a_1) \parent{ u_1(c_{2}, a_1) - u_1(c_{2}, b_1) } \geq 0 \\
    & \Longleftrightarrow
    \mu(a_2,a_1) \parent{ u_1(a_{2}, a_1) - u_1(a_{2}, b_1) }
    +
    \mu(b_2,a_1) \parent{ u_1(b_{2}, a_1) - u_1(b_{2}, b_1) }
    \geq 0 \\
    & \Longleftrightarrow
    \alpha \parent{ 4 - 6 }
    +
    \beta \parent{ 1 - (-3) }
    \geq 0 \\
    & \Longleftrightarrow
    4 \beta - 2 \alpha 
    \geq 0 \\
    & \Longleftrightarrow
    2 \beta \geq \alpha  \\
    \end{align*}
    
    \item Case 2. We have $c_1 = b_1$ and $e_1 = a_1$. The constraint becomes
    
    \begin{align*}
    & \sum_{c_{2} \in C_{2}} \mu(c_2,b_1) \parent{ u_1(c_{2}, b_1) - u_1(c_{2}, a_1) } \geq 0 \\
    & \Longleftrightarrow
    \mu(a_2,b_1) \parent{ u_1(a_{2}, b_1) - u_1(a_{2}, a_1) }
    +
    \mu(b_2,b_1) \parent{ u_1(b_{2}, b_1) - u_1(b_{2}, a_1) }
    \geq 0 \\
    & \Longleftrightarrow
    \beta \parent{ 6 - 4 }
    +
    \delta \parent{ - 3 - 1 }
    \geq 0 \\
    & \Longleftrightarrow
    2 \beta 
    -
    4 \parent{ 1 - \alpha - 2 \beta}
    \geq 0 \\
    & \Longleftrightarrow
    4 \alpha + 10 \beta - 4
    \geq 0 \\
    & \Longleftrightarrow
    2 \alpha + 5 \beta \geq 2 \\
    \end{align*}
    
\end{enumerate}

By symmetry, if we follow the same reasoning for player 2, we will find the same constraints. However, before defining the final optimization problem, we cannot forget the extra constraint: $\delta \geq 0$. Indeed, since we work with variables $\alpha$ and $\beta$, this constraint becomes $1 - \alpha - 2 \beta \geq 0 $.
Hence we have the following optimization problem

\begin{equation*}
    \begin{aligned}
    \underset{}{\text{max }} &
    14 \alpha + 26 \beta - 6 & & \\
    \text{such that  } &
        \alpha, \beta \geq 0 & \\
        & 2 \beta \geq \alpha & \\
        & 2 \alpha + 5 \beta \geq 2 & \\
        & 1 \geq \alpha + 2 \beta. &
    \end{aligned}
    \end{equation*}


We solve this problem graphically. On Figure \ref{fig:niceGraph}, we can see that we have four extreme points. Let us give color-labels to the constraints.
\begin{enumerate}
    \item The inequality constraint $2 \beta \geq \alpha$ is the \textbf{green} line, and looking at the sign of the inequality, we have to be above the green line.
    \item The inequality constraint $2 \alpha + 5 \beta \geq 2$ is the \textbf{red} line, and looking at the sign of the inequality, we have to be above the red line.
    \item The inequality constraint $1 \geq \alpha + 2 \beta$ is the \textbf{blue} line, and looking at the sign of the inequality, we have to be below the blue line.
\end{enumerate}

\begin{center}
    \begin{center}
        \begin{tikzpicture}[scale=7]
        % Axis
        \draw[axis] (0,0) -- (1.2,0) node[right=\nudge cm] {\(\alpha\)};
        \draw[axis] (0,0) -- (0,0.7) node[above=\nudge cm] {\(\beta\)};
        \begin{scope}
          % Avoid going too far
          \clip (-\nudge,-\nudge) rectangle (2+\nudge,2+\nudge);
          \draw[ineq1] (0,0) -- (1,0.5) coordinate (ineq1);
          \draw[ineq2] (0,2/5) -- (1,0) coordinate (ineq2);
          \draw[eq] (0,0.5) -- (1,0) coordinate (eq1);
         \begin{scope}
         \end{scope}
        \end{scope}
        \foreach \coord/\adj in {
          %\node[left] {A} at (0,1) "ultra thick point"
          %{(0,0)}/left,
          {(1,0.5)}/above,
          {(0,2/5)}/left,
          {(0,0.5)}/left,
          {(1,0)}/below,%
          %{(8/9,0)}/below,
          %{(2,0)}/below%
        } {
          \fill \coord circle (0.3pt) node[\adj] {$\coord$};
        }
        \foreach \coord/\adj/\name in {
          {(2/4,1/4)}/above/A,
          {(4/9,2/9)}/below/B,
          {(0,2/5)}/below right/C,
          {(0,0.5)}/above right/D%
        } {
          \fill \coord circle (0.3pt) node[\adj] {$\name$};
        }
        \end{tikzpicture}
      \end{center}
    \captionof{figure}{A very nice graph}
    \label{fig:niceGraph}
\end{center}




To find the correlated equilibrium that maximizes (resp. minimizes) the expected sum of payoffs of the two players, to answer question a (resp. b), we can draw the line $14 \alpha + 26 \beta - 6$ and find the point where it is the greatest (resp. lowest). However, we can also compute the objective value for each of the four extreme points.

\begin{enumerate}
    \item point $A$, the intersection of the green and the blue lines. Hence the coordinates of point $A$ satisfy
    \begin{equation*}
       \begin{cases}
        2 \beta = \alpha  \\
        1 = \alpha + 2 \beta
        \end{cases} 
    \end{equation*}
    
    We find $\parent{\alpha, \beta} = \parent{\dfrac{1}{2}, \dfrac{1}{4}}$. The objective value is $7.5$.
    
    \item point $B$, the intersection of the green and the red lines. Hence the coordinates of point $A$ satisfy
    \begin{equation*}
       \begin{cases}
        2 \beta = \alpha  \\
        2 \alpha + 5 \beta = 2
        \end{cases} 
    \end{equation*}
    
    We find $\parent{\alpha, \beta} = \parent{\dfrac{4}{9}, \dfrac{2}{9}}$. The objective value is $6$.
    
    \item point $C$, we have $\parent{\alpha, \beta} = \parent{0, \dfrac{2}{5}}$. The objective value is $4.4$.
    
    \item point $D$, we have $\parent{\alpha, \beta} = \parent{0, \dfrac{1}{2}}$. The objective value is $7$.
\end{enumerate}

On the one hand, the point that maximizes the expected sum of payoffs of the two players is point $A$. The associated correlated equilibrium is defined by the variables $\alpha = \dfrac{1}{2}$, $\beta = \dfrac{1}{4}$, $\gamma = \dfrac{1}{4}$ and $\delta = 0$. The fact that $\delta = 0$ coincides with our intuition. 



On the other hand, the point that maximizes the expected sum of payoffs of the two players is point $C$. The associated correlated equilibrium is defined by the variables $\alpha = 0$, $\beta = \dfrac{2}{5}$, $\gamma = \dfrac{2}{5}$ and $\delta = \dfrac{1}{5}$. The fact that $\alpha = 0$ coincides with our intuition. We would like to see $\delta = 1$ but this is not possible since we can only consider correlated equilibria.

\end{solution}

\section{}
Sender-receiver games are a special case of Bayesian games with communication in which Player~1 (the sender) has a hidden type, but has no action to take, while player~2 (the receiver) has no type, but chooses an action from a set $C_2$.

\begin{enumerate}
	\item[a.] Write the expression that allows to compute the expected payoff of each player.
	\item[b.] Write the strategic incentive constraints and simplify them as much as possible.
\end{enumerate}

Now, consider that the payoff matrix for this sender-receiver game is given by the following table:

\begin{center}
	\begin{tabular}[h!]{l|ccccccc}
		&& $x_2$ && $y_2$ && $z_2$ & \\
		\hline
		$1.a$ && $2,3$ && $0,2$ && $-1,0$ & \\
		$1.b$ && $1,0$ && $2,2$ && $0,3$ & 
	\end{tabular} 
\end{center}
	
We will assume that $p(t_1 = a) = p(t_1 = b) = 1/2$.
	
\begin{enumerate}
	\item[c.] In the situation where there is no mediator, show that whatever the message sent by Player~1, Player~2 chooses action $y_2$. From there, deduce upper bounds for the achievable expected payoffs for the two players with a Nash equilibrium of this game without communication.
	\item[d.] Still without a mediator, assume now that Player~1 communicates with Player~2 through a noisy channel: he can send a signal (which we denote ``$m$'') to indicate that his type is $a$, but there is only one chance over two that Player~2 actually receives it. (We assumed that if Player~1 has type $b$, then he does not send any signal.) Show that this situation actually allows the players to improve their payoff. Explain this paradox in terms of correlated equilibria. 
	\item[e.] Show that the following correlated equilibrium mechanism maximizes the payoff of Player~2:
		\begin{alignat*}{3}
			& \mu(x_2 | 1.a) = 2/3, \quad && \mu(y_2 | 1.a) = 1/3, \quad && \mu(z_2 | 1.a) = 0 \\
			& \mu(x_2 | 1.b) = 0,   \quad && \mu(y_2 | 1.b) = 2/3, \quad && \mu(z_2 | 1.b) = 1/3.
		\end{alignat*}
\end{enumerate}

\section{}
In the exercise session on Nash equilibria, we found that the following two-player game allows a unique Nash equilibrium in which each player has an expected payoff of~3:

	\begin{center}
		\begin{tabular}[h!]{l|ccccccc}
			&& $x_2$ && $y_2$ && $z_2$ & \\
			\hline
			$x_1$ && $0,0$ && $5,4$ && $4,5$ & \\
			$y_1$ && $4,5$ && $0,0$ && $5,4$ & \\
			$z_1$ && $5,4$ && $4,5$ && $0,0$ & 
		\end{tabular} 
	\end{center}
	
\begin{enumerate}
	\item[a.] Show that this game allows a correlated equilibrium for which each player receives a payoff higher than~4.
	\item[b.] Find the correlated equilibrium that maximizes the expected payoff of Player~2.
\end{enumerate}

\thex{}
Let $u = \big( u_1, u_2 \big)^T$ and $v = \big( v_1, v_2 \big)^T$ be payoff vectors that correspond to two different correlated equilibria of a given game $\Gamma^c$. Show that for this game, there also exists a correlated equilibrium that yields the payoff vector $w = \big( \frac{1}{3} u_1 + \frac{2}{3} v_1, \, \frac{1}{3} u_2 + \frac{2}{3} v_2 \big)^T$. 

\newpage

\section*{Reminders}

\begin{itemize}[leftmargin=*]
\renewcommand{\labelitemi}{$\bullet$}

	\item \textbf{Correlated equilibria for games in strategic form}
	\vspace{.3cm}

	\begin{itemize}

		\item The payoff of player $i$, according to the correlated strategy $\mu$ proposed by a mediator, is given by:
		\begin{align*}
			U_i(\mu) = \sum_{c \in C} \mu(c) \, u_i(c),
		\end{align*}
		where $C$ is the set of possible combinations of pure strategies.
			
		\item A correlated equilibrium for a game in \emph{strategic form} must satisfy the following strategic incentive constraints:
		\begin{align*}
			\sum_{c_{-i} \in C_{-i}} \mu(c) \, \big( u_i(c_{-i}, \; c_i) - u_i(c_{-i}, \; e_i) \big) \geq 0, \quad \forall i \in N, \ \forall c_i \in C_i, \ \forall e_i \in C_i,
		\end{align*}
		where $c = (c_{-i}, c_i)$, $c_{-i}$ being the strategy of the players other than $i$. These constraints require that the players cannot improve their payoff by deviating from the correlated strategy $\mu$ suggested by the mediator.
			
	\end{itemize}
	\vspace{.3cm}

	\item \textbf{Correlated equilibria for Bayesian games}
	\vspace{.3cm}

	\begin{itemize}

		\item The payoff of player $i$, according to the correlated strategy $\mu$ and give his type $t_i$, is give by:
		\begin{align*}
			U_i(\mu \; | \; t_i) = \sum_{t_{-i} \in T_{-i}} \sum_{c \in C} p_i(t_{-i} \; | \; t_i) \, \mu(c \; | \; t) \, u_i(c, \; t),
		\end{align*}
		where $t = (t_{-i}, t_i)$ and $T_{-i}$ are the set of possible combinations of types for the  players other than $i$.
		
		\item A correlated equilibrium for a \emph{Bayesian} game must satisfy the following strategic incentive constraints:
		\begin{align*}
			U_i(\mu \; | \; t_i) \geq U_i^*(\mu, \; \delta_i, \; s_i \; | \; t_i), \quad \forall i \in N, \ \forall t_i \in T_i, \ \forall s_i \in T_i, \ \forall \delta_i : C_i \rightarrow C_i,
		\end{align*}
		with 
		\begin{align*}
			U_i^*(\mu, \; \delta_i, \; s_i \; | \; t_i) = \sum_{t_{-i} \in T_{-i}} \sum_{c \in C} p_i(t_{-i} \; | \; t_i) \, \mu(c \; | \; t_{-i}, s_i) \, u_i\big( \; (c_{-i}, \; \delta_i(c_i)) \;, \; t \big),
		\end{align*}
		where $t_i$ is the type of $i$, $s_i$ is the type that $i$ reveals to the mediator and $\delta_i$ is the function chosen by player $i$ that, for any action $c_i$ that can be suggested by the mediator, chooses an action $\delta_i(c_i)$ to play instead. These constraints impose that the players neither have interest to lie on their type, nor to deviate from the correlated strategy $\mu$ suggested by the mediator.
			
	\end{itemize}
	
	\vspace{.3cm}
	\item \textbf{Duality}
	\vspace{.3cm}
	
	Any Linear Programming problem $(P)$ (called \emph{primal}) is of the form:
	\begin{align*}
		\begin{array}{lll}
			\max && c^T x, \\ 
			\mbox{s.t.} && Ax \leq b, \\
			&& x \geq 0,
		\end{array}
	\end{align*} 
	can be transformed into the following problem $(D)$ (called \emph{dual}):
	\begin{align*}
		\begin{array}{lll}
			\min && b^T y, \\ 
			\mbox{s.t.} && A^T y \geq c, \\
			&& y \geq 0.
		\end{array}
	\end{align*} 
	Furthermore, we have that:
	\begin{itemize}
		\item $(P)$ allows an optimal solution $x^*$ iff $(D)$ allows an optimal solution $y^*$. Furthermore $c^T x^* = b^T y^*$;
		\item if $(P)$ (resp. $(D)$) is unbounded, then $(D)$ (resp. $(P)$) is impossible;
		\item $(P)$ et $(D)$ can also be both impossible.
	\end{itemize}
	\vspace{.5cm}
	Moreover, for the \emph{complementary slackness theorem}, solutions $x^*$ for $(P)$ and $y^*$ for $(D)$ are optimal iff:
	\begin{itemize}
		\item $y_i^* = 0 \quad$ for all $i$ such that the $i^{\text{th}}$ constraint in $(P)$ is not tight;
		\item $x_i^* = 0 \quad$ for all $i$ such that the $i^{\text{th}}$ constraint in $(D)$ is not tight.
	\end{itemize}
		
\end{itemize}

\end{document}










