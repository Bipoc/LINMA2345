The main assumption here is that everyone knows if the other women are unfaithfuls but does not know if he is cuckold.

\paragraph{}Let's assume that there is one cuckold ($k = 1$). As he knows, that the other women are faithfuls, he concludes that he is cuckold and kills his wife the same day! 

\paragraph{}We assume now we have two cuckolds.
Using our assumption, cuckold 1 knows that the wife of cuckold 2 is cheating and that other women are monogamous. Analogically, cuckold 2 knows that the wife of cuckold 1 is cheating and the others wives are not. On day one, both cuckold were expecting the other to realise he had been cheated on and to kill his wife. But no one did ! 

Therefore, on day two, cuckold 1 understands that cuckold 2 did not kill his wife because cuckold 2 knows wife 1 is cheating. Cuckold 2 realizes the same thing and they both kill their wife the second day.  

\paragraph{}Keeping this logic, until \textit{k} cuckolds lead us to the conclusion, that all of them kill their wives the $k$th day. Indeed, if a certain person know that there's $k - 1$ unfaithfuls women, and nothing happened the $(k-1)$th day. He understands, that if the $(k-1)$ other cuckolds didn't act, it is because they know his wife is cheating and each of them was expecting him to act the $k-1$th day. Hence he was the $k$th cuckold!

In conclusion, there were 13 unfaithfuls women in Baghdad! 
