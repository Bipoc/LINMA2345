The main assumption here is that everyone knows if the other women are unfaithful but does not know if he is cuckold.

\paragraph{}Let us assume that there is one cuckold, i.e., $k = 1$. As he knows that the other women are faithful, he concludes that he is cuckold and kills his wife the same day! 

\paragraph{}We assume now that we have two cuckolds, i.e., $k=2$.
Using our assumption, cuckold 1 knows that the wife of cuckold 2 is cheating and that other women are monogamous. Analogically, cuckold 2 knows that the wife of cuckold 1 is cheating and the others wives are not. On day one, both cuckolds were expecting the other to realise he had been cheated on and to kill his wife. But no one did ! 

Therefore, on day two, cuckold 1 understands that cuckold 2 did not kill his wife because cuckold 2 knows wife 1 is cheating. Cuckold 2 realizes the same thing, and they both kill their wife the second day.  

\paragraph{}Keeping this logic until $k$ cuckolds leads us to the conclusion that all of them kill their wives the $k$th day. Indeed, let us take a certain man who knows that there are $k - 1$ unfaithful women, and nothing happened the $(k-1)$th day. He understands that if the $(k-1)$ other cuckolds didn't act, it is because they know his wife is cheating and each of them was expecting him to act the $k-1$th day. Hence he was the $k$th cuckold!

In conclusion, there were 13 unfaithful women in Baghdad! 
