\paragraph{Step 1.} Strategy $x_{2}$ is weakly dominated by the pure strategy strategy $z_{2}$. Here we look at the columns, hence we are looking at player 2, hence we have to consider the payoffs of player 2. The weak domination is clear, since $3 \leq 3$, $9 \leq 9$ and $5 \leq 6$. The table becomes

\begin{tabular}[h!]{l|ccccc}
%		&&&&& \Large{$C_2$} &&& \\
		&& $y_2$ && $z_2$ \\
		\hline
		$x_1$ && 9,4 && 1,3 & \\
		$y_1$ && 8,1 && 2,9 & \\
		$z_1$ && 6,6 && 7,6 &
	\end{tabular} 
	
	\paragraph{Step 2.} Strategy $y_{1}$ is weakly dominated by the randomized strategy $\frac{2}{3} x_{1} + \frac{1}{3} z_{1}$. Here we look at the rows, hence we are looking at player 1, hence we have to consider the payoffs of player 1. The weak domination is clear, since $8 \leq \frac{2}{3} \cdot 9 + \frac{1}{3} \cdot 6 = 9$ and $2 \leq \frac{2}{3} \cdot 1 + \frac{1}{3} \cdot 7 = 3$. The table becomes

\begin{tabular}[h!]{l|ccccc}
%		&&&&& \Large{$C_2$} &&& \\
		&& $y_2$ && $z_2$ \\
		\hline
		$x_1$ && 9,4 && 1,3 & \\
		$z_1$ && 6,6 && 7,6 &
	\end{tabular} 
	
\paragraph{Step 3.} Strategy $z_{2}$ is weakly dominated by the pure strategy $y_{2}$. Here we look at the columns, hence we are looking at player 2, hence we have to consider the payoffs of player 2. The weak domination is clear, since $3 \leq 4$ and $6 \leq 6$. The table becomes

\begin{tabular}[h!]{l|ccc}
%		&&&&& \Large{$C_2$} &&& \\
		&& $y_2$ \\
		\hline
		$x_1$ && 9,4 & \\
		$z_1$ && 6,6 &
	\end{tabular} 

\paragraph{Step 4.} Strategy $z_{1}$ is strongly dominated by the pure strategy $x_{1}$.Here we look at the rows, hence we are looking at player 1, hence we have to consider the payoffs of player 1. The strong domination is clear, since $6 < 9$. The table becomes

\begin{tabular}[h!]{l|ccc}
%		&&&&& \Large{$C_2$} &&& \\
		&& $y_2$ \\
		\hline
		$x_1$ && 9,4 &
	\end{tabular} 
	

\vspace{10mm}

We end this exercise with a note about randomized strategies. At step 2, we can guess the result with a little bit of mathematical intuition. We guess that $y_1$ is weakly dominated by a combination of $x_{1}$ and $x_{1}$ because $8 < 9$ and $2 < 7$. However, we don't have a robust method to find the coefficients. We can do so by defining the linear program


        

\begin{equation*}
    \begin{aligned}
    \underset{}{\text{min }} &
    0 & & \\
    \text{such that  } &
        \lambda_{1} + \lambda_{2} = 1 & \\
        & 8 \leq 9 \lambda_{1} + 6 \lambda_{2} &  \\
        & 2 \leq \lambda_{1} + 7 \lambda_{2} & \\
        & \lambda_{1}, \lambda_{2} \geq 0. &
    \end{aligned}
\end{equation*}
   

We can solve this graphically, see Figure \ref{fig:feasProb} on page \pageref{fig:feasProb}. Let us give color-labels to the constraints.
\begin{enumerate}
    \item The equality constraint $\lambda_{1} + \lambda_{2} = 1$ is the \textbf{blue} line, and since this is an equality constraint, we have to find a solution $\parent{\lambda_{1}, \lambda_{2}}$ on the blue line.
    \item The inequality constraint $8 \leq 9 \lambda_{1} + 6 \lambda_{2}$ is the \textbf{green} line, and looking at the sign of the inequality, we have to be above the red line.
    \item The inequality constraint $2 \leq \lambda_{1} + 7 \lambda_{2}$ is the \textbf{red} line, and looking at the sign of the inequality, we have to be above the green line.
\end{enumerate}


  


\begin{figure}[h]
    \begin{center}
        \begin{tikzpicture}[scale=4]
        % Axis
        \draw[axis] (0,0) -- (2.5,0) node[right=\nudge cm] {\(\lambda_1\)};
        \draw[axis] (0,0) -- (0,1.5) node[above=\nudge cm] {\(\lambda_2\)};
        \begin{scope}
          % Avoid going too far
          \clip (-\nudge,-\nudge) rectangle (2+\nudge,2+\nudge);
          \draw[ineq1] (0,4/3) -- (8/9,0) coordinate (ineq1);
          \draw[ineq2] (0,2/7) -- (2,0) coordinate (ineq2);
          \draw[eq] (0,1) -- (1,0) coordinate (eq1);
         \begin{scope}
         \end{scope}
        \end{scope}
        \foreach \coord/\adj in {
          %\node[left] {A} at (0,1) "ultra thick point"
          {(0,4/3)}/left,
          {(0,1)}/left,
          {(0,2/7)}/left,
          {(1,0)}/below,
          {(8/9,0)}/below,
          {(2,0)}/below%
        } {
          \fill \coord circle (0.5pt) node[\adj] {$\coord$};
        }
        \foreach \coord/\adj/\name in {
          {(2/3,1/3)}/above right/A,
          {(5/6,1/6)}/above right/B%
        } {
          \fill \coord circle (0.5pt) node[\adj] {$\name$};
        }
        \end{tikzpicture}
      \end{center}
    \caption{The feasible set is not empty}
    \label{fig:feasProb}
\end{figure}


We identify two important points:
\begin{enumerate}
    \item point $A$, the intersection of the green and the blue lines. Hence the coordinates of point $A$ satisfy
    \begin{equation*}
       \begin{cases}
        8 = 9 \lambda_{1} + 6 \lambda_{2}  \\
        \lambda_{1} + \lambda_{2} = 1
        \end{cases} 
    \end{equation*}
    
    We find $\parent{ \lambda_{1},\lambda_{2} } = \parent{ \dfrac{2}{3},\dfrac{1}{3} } $.
    
    \item point $B$, the intersection of the red and the blue lines. Hence the coordinates of point $B$ satisfy
    \begin{equation*}
    \begin{cases}
    2 = \lambda_{1} + 7 \lambda_{2}  \\
    \lambda_{1} + \lambda_{2} = 1
    \end{cases}
    \end{equation*}
    We find $\parent{ \lambda_{1},\lambda_{2} } = \parent{ \dfrac{5}{6},\dfrac{1}{6} } $.
\end{enumerate}
    
Of course, we could take one of these two combinations of $\lambda_{1},\lambda_{2}$ as the coefficients. We can also take any point between $A$ and $B$, on the blue line. The whole point of the above figure, Figure \ref{fig:feasProb}, was to show that the feasible set is not empty. If the graph was like on Figure \ref{fig:unFeasProb}, then the feasible set would be empty and in that case, it is not possible to find coefficients $\lambda_{1},\lambda_{2}$ such that $y_1$ is weakly dominated by $\lambda_{1} x_{1} + \lambda_{2} z_{1}$.


\begin{figure}[h!]
    \centering
   
    \begin{center}
        \begin{tikzpicture}[scale=4]
        % Axis
        \draw[axis] (0,0) -- (2.5,0) node[right=\nudge cm] {\(\lambda_1\)};
        \draw[axis] (0,0) -- (0,1.5) node[above=\nudge cm] {\(\lambda_2\)};
        \begin{scope}
          % Avoid going too far
          \clip (-\nudge,-\nudge) rectangle (2+\nudge,2+\nudge);
          \draw[ineq1] (0,4/3) -- (3/2,0) coordinate (ineq1);
          \draw[ineq2] (0,2/7) -- (2,0) coordinate (ineq2);
          \draw[eq] (0,1) -- (1,0) coordinate (eq1);
         \begin{scope}
         \end{scope}
        \end{scope}
        \foreach \coord/\adj in {
          %\node[left] {A} at (0,1) "ultra thick point"
          {(0,4/3)}/left,
          {(0,1)}/left,
          {(0,2/7)}/left,
          {(1,0)}/below,
          {(3/2,0)}/below,
          {(2,0)}/below%
        } {
          \fill \coord circle (0.5pt) node[\adj] {$\coord$};
        }
        \end{tikzpicture}
      \end{center}
    \caption{The feasible set is empty}
    \label{fig:unFeasProb}
\end{figure}


Finally, let us analyze the inequalities depending on which point we choose. 
\begin{enumerate}
    \item point $A: \parent{ \lambda_{1},\lambda_{2} } = \parent{ \dfrac{2}{3},\dfrac{1}{3} }$.
    Then we have
    \begin{equation*}
    \begin{cases}
    8 \leq 9 \lambda_{1} + 6 \lambda_{2} = 8  \\
    2 \leq \lambda_{1} + 7 \lambda_{2} = 3
    \end{cases}
    \end{equation*}
    Since we are on the green line, it was predictable that the first inequality is not strict ($8 = 8$) and the second one is strict ($2 < 3$). We have a \textit{margin} for the second inequality.
    
    \item point $B: \parent{ \lambda_{1},\lambda_{2} } = \parent{ \dfrac{5}{6},\dfrac{1}{6} }$.
    Then we have
    \begin{equation*}
    \begin{cases}
    8 \leq 9 \lambda_{1} + 6 \lambda_{2} = 8.5  \\
    2 \leq \lambda_{1} + 7 \lambda_{2} = 2
    \end{cases}
    \end{equation*}
    Since we are on the red line, it was predictable that the first inequality is strict ($8 < 8.5$) and the second one is not strict ($2 = 2$). We have a \textit{margin} for the first inequality.
    
    \item between points $A$ and $B$: $\dfrac{2}{3} \leq \lambda_{1} \leq \dfrac{5}{6}$ and $\dfrac{1}{5} \leq \lambda_{2} \leq \dfrac{1}{3}$.
    Then we have
    \begin{equation*}
    \begin{cases}
    8 \leq 9 \lambda_{1} + 6 \lambda_{2}  \\
    2 \leq \lambda_{1} + 7 \lambda_{2}
    \end{cases}
    \end{equation*}
    Since we are not on the green line, nor on the red line, it was predictable that both inequalities are strict. We have a \textit{margin} for both inequalities.
    
    
\end{enumerate}
    
    
    
    
    




        
