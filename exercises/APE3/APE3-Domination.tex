\documentclass{../ape}

\usepackage{../../linma2345}

\newcommand{\X}{X}
\newcommand{\Y}{Y}

\begin{document}

\setlength{\parindent}{0pt}
\setlength{\parskip}{1ex plus 0.5ex minus 0.2ex}

{
  \Large
  \textbf{Session 3: Common knowledge and domination}
}

%\section*{Exercices}

\section{}
Analyze the strategic form game represented by the following table. Eliminate iteratively as many strategies as possible by weak domination.
\begin{center}
	\begin{tabular}[h!]{l|ccccccc}
%		&&&&& \Large{$C_2$} &&& \\
		&& $x_2$ && $y_2$ && $z_2$ \\
		\hline
		$x_1$ && 2,3 && 9,4 && 1,3 & \\
		$y_1$ && 2,9 && 8,1 && 2,9 & \\
		$z_1$ && 5,5 && 6,6 && 7,6 &
	\end{tabular} 
\end{center}

\begin{solution}
\paragraph{Step 1.} Strategy $x_{2}$ is weakly dominated by the pure strategy strategy $z_{2}$. Here we look at the columns, hence we are looking at player 2, hence we have to consider the payoffs of player 2. The weak domination is clear, since $3 \leq 3$, $9 \leq 9$ and $5 \leq 6$. The table becomes

\begin{tabular}[h!]{l|ccccc}
%		&&&&& \Large{$C_2$} &&& \\
		&& $y_2$ && $z_2$ \\
		\hline
		$x_1$ && 9,4 && 1,3 & \\
		$y_1$ && 8,1 && 2,9 & \\
		$z_1$ && 6,6 && 7,6 &
	\end{tabular} 
	
	\paragraph{Step 2.} Strategy $y_{1}$ is weakly dominated by the randomized strategy $\frac{2}{3} x_{1} + \frac{1}{3} z_{1}$. Here we look at the rows, hence we are looking at player 1, hence we have to consider the payoffs of player 1. The weak domination is clear, since $8 \leq \frac{2}{3} \cdot 9 + \frac{1}{3} \cdot 6 = 9$ and $2 \leq \frac{2}{3} \cdot 1 + \frac{1}{3} \cdot 7 = 3$. The table becomes

\begin{tabular}[h!]{l|ccccc}
%		&&&&& \Large{$C_2$} &&& \\
		&& $y_2$ && $z_2$ \\
		\hline
		$x_1$ && 9,4 && 1,3 & \\
		$z_1$ && 6,6 && 7,6 &
	\end{tabular} 
	
\paragraph{Step 3.} Strategy $z_{2}$ is weakly dominated by the pure strategy $y_{2}$. Here we look at the columns, hence we are looking at player 2, hence we have to consider the payoffs of player 2. The weak domination is clear, since $3 \leq 4$ and $6 \leq 6$. The table becomes

\begin{tabular}[h!]{l|ccc}
%		&&&&& \Large{$C_2$} &&& \\
		&& $y_2$ \\
		\hline
		$x_1$ && 9,4 & \\
		$z_1$ && 6,6 &
	\end{tabular} 

\paragraph{Step 4.} Strategy $z_{1}$ is strongly dominated by the pure strategy $x_{1}$.Here we look at the rows, hence we are looking at player 1, hence we have to consider the payoffs of player 1. The strong domination is clear, since $6 < 9$. The table becomes

\begin{tabular}[h!]{l|ccc}
%		&&&&& \Large{$C_2$} &&& \\
		&& $y_2$ \\
		\hline
		$x_1$ && 9,4 &
	\end{tabular} 
	

\vspace{10mm}

We end this exercise with a note about randomized strategies. At step 2, we can guess the result with a little bit of mathematical intuition. We guess that $y_1$ is weakly dominated by a combination of $x_{1}$ and $x_{1}$ because $8 < 9$ and $2 < 7$. However, we don't have a robust method to find the coefficients. We can do so by defining the linear program


        

\begin{equation*}
    \begin{aligned}
    \underset{}{\text{min }} &
    0 & & \\
    \text{such that  } &
        \lambda_{1} + \lambda_{2} = 1 & \\
        & 8 \leq 9 \lambda_{1} + 6 \lambda_{2} &  \\
        & 2 \leq \lambda_{1} + 7 \lambda_{2} & \\
        & \lambda_{1}, \lambda_{2} \geq 0. &
    \end{aligned}
\end{equation*}
   

We can solve this graphically, see Figure \ref{fig:feasProb} on page \pageref{fig:feasProb}. Let us give color-labels to the constraints.
\begin{enumerate}
    \item The equality constraint $\lambda_{1} + \lambda_{2} = 1$ is the \textbf{blue} line, and since this is an equality constraint, we have to find a solution $\parent{\lambda_{1}, \lambda_{2}}$ on the blue line.
    \item The inequality constraint $8 \leq 9 \lambda_{1} + 6 \lambda_{2}$ is the \textbf{green} line, and looking at the sign of the inequality, we have to be above the red line.
    \item The inequality constraint $2 \leq \lambda_{1} + 7 \lambda_{2}$ is the \textbf{red} line, and looking at the sign of the inequality, we have to be above the green line.
\end{enumerate}


  


\begin{figure}[h]
    \begin{center}
        \begin{tikzpicture}[scale=4]
        % Axis
        \draw[axis] (0,0) -- (2.5,0) node[right=\nudge cm] {\(\lambda_1\)};
        \draw[axis] (0,0) -- (0,1.5) node[above=\nudge cm] {\(\lambda_2\)};
        \begin{scope}
          % Avoid going too far
          \clip (-\nudge,-\nudge) rectangle (2+\nudge,2+\nudge);
          \draw[ineq1] (0,4/3) -- (8/9,0) coordinate (ineq1);
          \draw[ineq2] (0,2/7) -- (2,0) coordinate (ineq2);
          \draw[eq] (0,1) -- (1,0) coordinate (eq1);
         \begin{scope}
         \end{scope}
        \end{scope}
        \foreach \coord/\adj in {
          %\node[left] {A} at (0,1) "ultra thick point"
          {(0,4/3)}/left,
          {(0,1)}/left,
          {(0,2/7)}/left,
          {(1,0)}/below,
          {(8/9,0)}/below,
          {(2,0)}/below%
        } {
          \fill \coord circle (0.5pt) node[\adj] {$\coord$};
        }
        \foreach \coord/\adj/\name in {
          {(2/3,1/3)}/above right/A,
          {(5/6,1/6)}/above right/B%
        } {
          \fill \coord circle (0.5pt) node[\adj] {$\name$};
        }
        \end{tikzpicture}
      \end{center}
    \caption{The feasible set is not empty}
    \label{fig:feasProb}
\end{figure}


We identify two important points:
\begin{enumerate}
    \item point $A$, the intersection of the green and the blue lines. Hence the coordinates of point $A$ satisfy
    \begin{equation*}
       \begin{cases}
        8 = 9 \lambda_{1} + 6 \lambda_{2}  \\
        \lambda_{1} + \lambda_{2} = 1
        \end{cases} 
    \end{equation*}
    
    We find $\parent{ \lambda_{1},\lambda_{2} } = \parent{ \dfrac{2}{3},\dfrac{1}{3} } $.
    
    \item point $B$, the intersection of the red and the blue lines. Hence the coordinates of point $B$ satisfy
    \begin{equation*}
    \begin{cases}
    2 = \lambda_{1} + 7 \lambda_{2}  \\
    \lambda_{1} + \lambda_{2} = 1
    \end{cases}
    \end{equation*}
    We find $\parent{ \lambda_{1},\lambda_{2} } = \parent{ \dfrac{5}{6},\dfrac{1}{6} } $.
\end{enumerate}
    
Of course, we could take one of these two combinations of $\lambda_{1},\lambda_{2}$ as the coefficients. We can also take any point between $A$ and $B$, on the blue line. The whole point of the above figure, Figure \ref{fig:feasProb}, was to show that the feasible set is not empty. If the graph was like on Figure \ref{fig:unFeasProb}, then the feasible set would be empty and in that case, it is not possible to find coefficients $\lambda_{1},\lambda_{2}$ such that $y_1$ is weakly dominated by $\lambda_{1} x_{1} + \lambda_{2} z_{1}$.


\begin{figure}[h!]
    \centering
   
    \begin{center}
        \begin{tikzpicture}[scale=4]
        % Axis
        \draw[axis] (0,0) -- (2.5,0) node[right=\nudge cm] {\(\lambda_1\)};
        \draw[axis] (0,0) -- (0,1.5) node[above=\nudge cm] {\(\lambda_2\)};
        \begin{scope}
          % Avoid going too far
          \clip (-\nudge,-\nudge) rectangle (2+\nudge,2+\nudge);
          \draw[ineq1] (0,4/3) -- (3/2,0) coordinate (ineq1);
          \draw[ineq2] (0,2/7) -- (2,0) coordinate (ineq2);
          \draw[eq] (0,1) -- (1,0) coordinate (eq1);
         \begin{scope}
         \end{scope}
        \end{scope}
        \foreach \coord/\adj in {
          %\node[left] {A} at (0,1) "ultra thick point"
          {(0,4/3)}/left,
          {(0,1)}/left,
          {(0,2/7)}/left,
          {(1,0)}/below,
          {(3/2,0)}/below,
          {(2,0)}/below%
        } {
          \fill \coord circle (0.5pt) node[\adj] {$\coord$};
        }
        \end{tikzpicture}
      \end{center}
    \caption{The feasible set is empty}
    \label{fig:unFeasProb}
\end{figure}


Finally, let us analyze the inequalities depending on which point we choose. 
\begin{enumerate}
    \item point $A: \parent{ \lambda_{1},\lambda_{2} } = \parent{ \dfrac{2}{3},\dfrac{1}{3} }$.
    Then we have
    \begin{equation*}
    \begin{cases}
    8 \leq 9 \lambda_{1} + 6 \lambda_{2} = 8  \\
    2 \leq \lambda_{1} + 7 \lambda_{2} = 3
    \end{cases}
    \end{equation*}
    Since we are on the green line, it was predictable that the first inequality is not strict ($8 = 8$) and the second one is strict ($2 < 3$). We have a \textit{margin} for the second inequality.
    
    \item point $B: \parent{ \lambda_{1},\lambda_{2} } = \parent{ \dfrac{5}{6},\dfrac{1}{6} }$.
    Then we have
    \begin{equation*}
    \begin{cases}
    8 \leq 9 \lambda_{1} + 6 \lambda_{2} = 8.5  \\
    2 \leq \lambda_{1} + 7 \lambda_{2} = 2
    \end{cases}
    \end{equation*}
    Since we are on the red line, it was predictable that the first inequality is strict ($8 < 8.5$) and the second one is not strict ($2 = 2$). We have a \textit{margin} for the first inequality.
    
    \item between points $A$ and $B$: $\dfrac{2}{3} \leq \lambda_{1} \leq \dfrac{5}{6}$ and $\dfrac{1}{5} \leq \lambda_{2} \leq \dfrac{1}{3}$.
    Then we have
    \begin{equation*}
    \begin{cases}
    8 \leq 9 \lambda_{1} + 6 \lambda_{2}  \\
    2 \leq \lambda_{1} + 7 \lambda_{2}
    \end{cases}
    \end{equation*}
    Since we are not on the green line, nor on the red line, it was predictable that both inequalities are strict. We have a \textit{margin} for both inequalities.
    
    
\end{enumerate}
    
    
    
    
    




        

\end{solution}

\section{}
The caliph of Baghdad one day summoned all married men of his city. At the time of the story, monogamy was the rule. The caliph spoke the following words: ``\textit{In order to fight against adultery, I ask each of you, if he realizes that he was deceived by his wife, to kill her that same evening at midnight. Moreover, I can tell you that at least one woman is unfaithful to her husband.} ''

Of course, the people of Baghdad are very obedient towards their caliph, and apply to the letter all orders. However, as it is still often the case these days, cuckolds are the only ones to ignore the infidelity of their wives. Every husband knows which wives of other husbands are unfaithful, but does not know if his own wife also is or not. Nevertheless, it is assumed that the people of Baghdad have great logical intelligence, and are therefore quite able to draw conclusions about their own situation from the behavior of others.

Nothing happens for 12 days. But the thirteenth day, at midnight, all the cuckolded husbands execute their women. How many unfaithful women were there in Baghdad?

\section{}
\TAone{} and \TAtwo{} play the following game: a positive integer $\X$ is chosen randomly according to a geometric distribution with parameter $0.9$ (that is, for every integer $k \geq 0$, $\Pr(\X = k) = 0.9 ^ k \times 0.1$). Once this integer is selected, \TAone{} receives the reward $\Y_1$ and \TAtwo{} receives the reward $\Y_2$ such that:
\begin{equation*}
	(\Y_1, \Y_2) = 
	\begin{cases}
		(\X, \X+1) \ \text{with probability} \ 1/2, \\
		(\X+1, \X) \ \text{with probability} \ 1/2.
	\end{cases}
\end{equation*}
Both players only observe their own reward. Therefore, one of the players observed $\X$ and the other player observes $\X + 1$, both players having the same probability of observing $\X$. The player who observed the highest value wins \SI{1}{\EUR} from the other player.
\begin{itemize}
	\item[a.] Show that if player $i$ observes $\Y_i \neq 0$, then that player will think that he has more than one in two chance of winning (that is, for all $k > 0$, we have $\Pr(\Y_i = \X+1 | \Y_i = k) > 1/2$).
	\item[b.] Consider the following assertion: \\
	\\
      \textit{Both players know that both players know that ... both players know that both players think they have more than one in two chance to win \SI{1}{\EUR}.} \\
	\\
	If $\X$ is worth $k$ and it is common knowledge that each player has only observed its own reward, then how many times at most can the words ``\textit{both players know that}'' be repeated so that the assertion remains true?
	\item[c.] If $\Y_1$ is worth $k$ and it is common knowledge that each player has only observed its own reward, then what is the least likely event that is common knowledge of both players? (An event is an information of the kind ``$\X \geq k$ for a certain $k$''.)
\end{itemize}

\section{}
In exercise 1., suppose you had not eliminated the weakly dominated strategies. Would there exist other Nash equilibria?

(\textbf{Teaser:} A \emph{Nash equilibrium} is a strategy profile for all players such that neither of them would gain from deviating alone from their strategy. Equivalently, at a Nash equilibrium, each player's strategy is a best response to that of the other players.)

%\newpage

\section{}
\textbf{(Theoretical exercise\footnote{\footnotesize Exercises with the label ``\textbf{Theoretical exercise}'' will be solved later in a dedicated exercise session.}.)} We consider a two-player game in which Player~1 has the choice between the actions $a, b$ and $c$. The payoffs for Player~1 are given by:
\begin{align*}
	u_1 = \begin{bmatrix} u_a \\ u_b \\ u_c \end{bmatrix},
\end{align*} 
where the $u_x$ are row vectors with appropriate dimensions. The strategy $[a]$ is strongly dominated iff there is no probability distribution $p \in \Delta(\Omega)$ such that $u_a \, p \geq u_x \, p$ for all $x \in \{ b, c \}$.

Show that this definition implies that $[a]$ is strongly dominated iff there exists a probability distribution $q \in \Delta(\Omega)$ such that $u_a < q^T \bar{u}_1$, where $\bar{u}_1$ is the matrix of payoffs of Player~1 without its first row.

(\textbf{Hint:} Restate the definition as an appropriate optimization problem and obtain its dual.)

\end{document}
