\documentclass{../ape}

\usepackage{../../linma2345}

% For 0901setF, see https://tex.stackexchange.com/questions/299560/strange-error-with-tikz-arrows
\tikzstyle{line}=[draw]

\begin{document}

\session{9}{Bargaining and Cooperation}

%\textbf{Session 9: bargaining and cooperation}

%\section*{Exercices}

\paragraph{1. } Alice and Bob are discussing to decide whether to go to the boxing match, to the ballet or to stay home tonight. For Bob, only the boxing match is of interest, and he considers that going to the ballet or staying home are equally bad options. For Alice, the best option is to go to the ballet. However, for her, going to the boxing match would be equivalent to choosing a lottery with a probability $1/4$ of going to the ballet and a probability $3/4$ of staying home. During her studies, Alice has learned how to use Matlab and so she knows how to generate random numbers to simulate a lottery that can help them make a decision. If they fail to agree, the couple will stay home. It is to be noted that both Bob and Alice follow the axioms of von Neumann-Morgenstern utility theory.

\begin{enumerate}
	\item[a.] Describe this situation by a two-player bargaining problem $(F,v)$ and compute its Nash's bargaining solution.
	\item[b.] The problem description above does not explicitly mention any two-player bargaining problem. However, this did not prevent you from answering question a. Which axiomatic property allowed you to find the solution of the negotiation anyway?
	\item[c.] Find natural utility scales for Alice and Bob in which the utilitarian and egalitarian solutions coincide with the Nash bargaining solution.
	\item[d.] If the television was broken, the prospect of staying home would become much worse for Bob. To be specific, Bob would prefer going to the ballet over choosing a lottery with a probability $2/3$ of going to the boxing match and $1/3$ to stay home. Conversely, a broken television would not affect Alice's preferences. Moreover, if Alice quietly broke the television, her brother (who repairs televisions, and comes for breakfast tomorrow morning) would fix it for free. According to the solution of the corresponding bargaining problem, how does breaking the television affect the probability of going to the ballet tonight?
\end{enumerate}

\begin{solution}
\begin{enumerate} [label=\alph*.]
	%%%%%% a
	\item As often, let us use $u_{A}$ and $u_{B}$ to denote the utility of Alice and Bob respectively. Since Alice and Bob respect the axioms of the von Neumann- Morgenstern utility theory, we know that we can use the von Neumann- Morgenstern utility theorem to find a utility function $u$ to describe the situation. Also, we know that $u$ can be chosen such that it is defined on $\squared{0, 1}$. We can find
	\begin{align*}
	    & u_{A} \parent{\squared{\text{ballet}}} = 1 \\
	    & u_{A} \parent{\squared{\text{boxing}}} = u_{A} \parent{ \dfrac{1}{4} \squared{\text{ballet}} + \dfrac{3}{4} \squared{\text{home}}} \\ \\
	    & u_{B} \parent{\squared{\text{boxing}}} = 1 \\
	    & u_{B} \parent{\squared{\text{ballet}}} = u_{B} \parent{\squared{\text{home}}} = 0
	\end{align*}
	
	We decide that $u_{A} \parent{\squared{\text{home}}} = 0$, hence we have $u_{A} \parent{\squared{\text{boxing}}} = u_{A} \parent{ \dfrac{1}{4} \squared{\text{ballet}}} = \dfrac{1}{4}$, by the linearity of the utility function.
	
	
	At this moment, we have to be careful. Let us imagine that for another situation, we have $u_{A} \parent{\squared{\text{ballet}}} = 1$ and $u_{B} \parent{\squared{\text{ballet}}} = 1$. This does not mean that the preferences are the same! Indeed, we are working with "Alice dollars" and "Bob dollars" and we cannot compare these two units. We cannot compare apples and oranges! We cannot add them together but we can multiply them. 
	
	We see that the disagreement point, denoted by $v$ happens when the couple stays at home. Hence, at $\squared{\text{home}}$, we have $\parent{v_{A}, v_{B}} = \parent{0, 0}$.
	
	Putting everything together, we find the utilies on Table \ref{tab:utilAB}. 
	

\begin{tabular}[h!]{r|lll}
            & $\squared{\text{boxing}}$ & $\squared{\text{ballet}}$ & $\squared{\text{home}}$ \\ \hline
$u_{A}$     & 1/4            & 1                         & 0 \\
$u_{B}$     & 1                         & 0                         & 0
\end{tabular}
\captionof{figure}{Utilities of Alice and Bob}
\label{tab:utilAB}


We can note that putting $u_{B} \parent{\squared{\text{boxing}}} = 1$ or $u_{B} \parent{\squared{\text{boxing}}} = 10$ would not change a thing, because we can rescale the problem, and use the equivalence of games.

Now let us construct the closed convex set $F$, representing the set of possible payoffs. The representation is given on Figure \ref{fig:setF}.

\begin{center}
\centering
        \begin{tikzpicture}[scale=2]
        % Axis
        \draw[axis] (0,0) -- (1.5,0) node[right=\nudge cm] {$u_{A}$};
        \draw[axis] (0,0) -- (0,1.5) node[above=\nudge cm] {$u_{B}$};
        \begin{scope}
          % Avoid going too far
          %\clip (-\nudge,-\nudge) rectangle (200+\nudge,100+\nudge);
          \draw[line] (0,0) -- (1/4,1) coordinate (ineq1);
          \draw[line] (1/4,1) -- (1,0) coordinate (ineq2);
          \draw[line] (0,0) -- (1,0) coordinate (ineq2);
          \begin{scope}
          %Clip sélectionne ce qu'il y a dans le polygone dessiné.
            \clip (0,0) |- (1/4,1) -- (1,0);
            \fill[hash] (0,0)--(1/4,1)--(1,0);
            %\draw[dashed,line] (0,1) -- (2,2) -- (3,1) -- (2,0);
          % (4,4) to be enough, anyway clip restrict us to the good part
          \end{scope}
        \end{scope}
        \foreach \coord/\adj in {
          {(0,0)}/below,
          {(0.25,1)}/above right,
          {(1,0)}/below%
        } {
          \fill \coord circle (1pt) node[\adj] {$\coord$};
        }
        \end{tikzpicture}
        
\captionof{figure}{Set $F$ of possible payoffs}
\label{fig:setF}
\end{center}
	
	On this Figure, we can add a few comments. The point $v$ is situated in the origin. Let us explain the three points
	\begin{enumerate}
	    \item $\parent{0,0}$ is the utilities when the couple stays at home;
	    \item $\parent{0.25,1}$ is the utilities when the couple goes to the boxing match;
	    \item $\parent{1,0}$ is the utilities when the couple goes to the ballet.
	\end{enumerate}
	
	Of course, there are other points in the set $F$. In other words
	\begin{enumerate}
	    \item if we consider a point on the line between $\parent{0,0}$ and $\parent{1,0}$, it will correspond to a random choice between staying home and going to the ballet;
	    \item if we consider a point on the line between $\parent{0,0}$ and $\parent{0.25,1}$, it will correspond to a random choice between staying home and going to the boxing match;
	    \item if we consider a point on the line between $\parent{0.25,1}$ and $\parent{1,0}$, it will correspond to a random choice between going to the boxing match and going to the ballet.
	\end{enumerate}
	
	The last category of points is the following: all the points in the interior of $F$. These points correspond to a random choice between staying home, going to the boxing match and going to the ballet.
	
	\vspace{5mm}
	
	Now let us find the Nash Bargaining Solution, also written as NBS. We know that $v = \parent{v_{A}, v_{B}} = \parent{0, 0}$. We also know that there is a unique solution function $\phi(\cdot, \cdot)$ that satisfies the axioms given in the Reminder. This solution function satisfies, for every two-person bargaining problem $(F, v)$
	\begin{equation*}
		\phi(F, v) \, \in \, \underset{x \, \in \, F, \, x \geq v}{\mathrm{argmax}} \ (x_1 - v_1)(x_2 - v_2).
	\end{equation*}
	
	Hence, let us write
	\begin{equation*}
	    \underset{x \, \in \, F, \, x \geq v}{\mathrm{argmax}} \ x_{A} \cdot x_{B}
	\end{equation*}
	
	If we define $f$ as the objective function, we have $f \parent{x_{A}, x_{B}} = x_{A} \cdot x_{B}$.
	
	We guess that the solution will be on the line between $\parent{0.25,1}$ and $\parent{1,0}$, because that is the Pareto front, and it looks like a good guess of where the ideal solution $\parent{x^{*}_{A}, x^{*}_{B}}$ will be. We can find the equation of the line between $\parent{0.25,1}$ and $\parent{1,0}$
    \begin{equation*}
        y = - \dfrac{4}{3} x + \dfrac{4}{3}
        \Rightarrow x_{B} = - \dfrac{4}{3} x_{A} + \dfrac{4}{3}
    \end{equation*}
    
    This means that we can rewrite the objective function in function of $x_{A}$ only
    \begin{equation*}
        f = f \parent{x_{A}, x_{B}} = f \parent{x_{A}} = x_{A} \cdot \parent{- \dfrac{4}{3} x_{A} + \dfrac{4}{3}} = - \dfrac{4}{3} x_{A} \parent{x_{A} - 1}.
    \end{equation*}
    
    In order to find the maximum, we will differentiate with respect to $x_{A}$, and find the value of $x_{A}$ to cancels the differential. We find
    \begin{align*}
        & \dfrac{\partial}{\partial x_{A}} f \parent{x_{A}} = f' \parent{x_{A}} = - \dfrac{8}{3} x_{A} + \dfrac{4}{3} \\
        & f' \parent{x^{*}_{A}} = 0 \Leftrightarrow x^{*}_{A} = \dfrac{1}{2}
    \end{align*}
    
    Then we can easily find $x^{*}_{B}$ with the equation of the line between $\parent{0.25,1}$ and $\parent{1,0}$. We find
    \begin{equation*}
        x^{*}_{B} = - \dfrac{4}{3} x^{*}_{A} + \dfrac{4}{3} = \dfrac{2}{3}
    \end{equation*}
    
    Hence we find $\parent{x^{*}_{A}, x^{*}_{B}} = \parent{\dfrac{1}{2}, \dfrac{2}{3}}$. The last step is to make sure that this solution is indeed in the set $F$. It is indeed the case. This step is necessary because the solution might be on the line between $\parent{0.25,1}$ and $\parent{1,0}$, but outside of $F$.
    
    \vspace{5mm}
    
    Now let us find the optimal strategy. We look for parameters $\alpha$, $\beta$ and $\gamma$ with
    \begin{align*}
        & p \parent{\squared{\text{boxing}}} = \alpha \\
        & p \parent{\squared{\text{ballet}}} = \beta \\
        & p \parent{\squared{\text{home}}} = \gamma
    \end{align*}
    
    We know that $\alpha, \beta, \gamma \geq 0$, and $\alpha + \beta + \gamma = 1$. Also, since the NBS is on the line between $\parent{0.25,1}$ and $\parent{1,0}$, we know that $\gamma = 0$, hence $\alpha + \beta = 1$ and we can write $\beta = 1 - \alpha$. 
    
    
    We know that 
    \begin{align*}
        & u_{A} \parent{\alpha \squared{\text{boxing}} + \parent{1 - \alpha} \squared{\text{ballet}}} = \dfrac{1}{2} \\
        & u_{B} \parent{\alpha \squared{\text{boxing}} + \parent{1 - \alpha} \squared{\text{ballet}}} = \dfrac{2}{3}
    \end{align*}
    
    
    We know that we only need one of these two equations to find $\alpha$. Let us take the second equation. Using the linearity of the utility function, we have
    \begin{equation*}
        \alpha \cdot u_{B} \parent{\squared{\text{boxing}}} + \parent{1 - \alpha} \cdot u_{B} \parent{ \squared{\text{ballet}}}
        = \alpha \cdot 1 + \parent{1 - \alpha} \cdot 0
        = \dfrac{2}{3}
    \end{equation*}
    
    Hence we find $\alpha = \dfrac{2}{3}$ and $\beta = \dfrac{1}{3}$.
	
	
		%%%%%% b
	\item The utility function $u$ that we have used in the previous question, corresponds to the von Neumann- Morgenstern utility theorem but not necessarely to the real payoffs of the players. Using Theorem 1.3 from Meyerson's textbook, we know that there exist numbers $A$ and $C$ such that $A > 0$ and
	\begin{equation*}
	    v (x) = A u(x) + C \quad , \quad \forall x \in X
	\end{equation*}
	
	where $X$ denotes the set of possible prizes that the decision-maker could ultimately get.
 
	
	
	In other words, to answer the question, the axiomatic property that allowed us to find the solution of the negotiation anyway was Axiom $\parent{3}$, i.e., the Axiom of scale covariance. 
	
		%%%%%% c
	\item 
	
		%%%%%% d
	\item First
	
\end{enumerate}
	
	
\end{solution}

\newpage
\paragraph{2. } We consider the following three-player game with cooperation:

	\begin{center}
		\begin{tabular}[h!]{l|ccc}
			& \Large{$x_3$} && \Large{$y_3$} \\
			&
			\begin{tabular}[h!]{cccccc}
				\hline
				& \Large{$x_2$} &&& \Large{$y_2$} & 
			\end{tabular}
			&&
			\begin{tabular}[h!]{cccccc}
				\hline
				& \Large{$x_2$} &&& \Large{$y_2$} & 
			\end{tabular} 
			\\[.2cm]
			\hline
			\\[-.4cm]
			\begin{tabular}[h!]{l}
				\Large{$x_1$} \\ \Large{$y_1$}
			\end{tabular}
			&
			\begin{tabular}[h!]{ccccc}
				& \Large{0,0,0} && \Large{6,5,4} & \\ 
				& \Large{5,4,6} && \Large{0,0,0} & 
			\end{tabular}
			&&
			\begin{tabular}[h!]{ccccc}
				& \Large{4,6,5} && \Large{0,0,0} & \\ 
				& \Large{0,0,0} && \Large{0,0,0} & 
			\end{tabular}
		\end{tabular} 
	\end{center}
	
\begin{enumerate}
	\item[a.] Propose a generalization of Theorem~1 (see Reminders below) for bargaining problems with more than two players (without a proof).
	\item[b.] For the specific problem above, assuming that players can negotiate and sign a contract that binds them to the directives of a mediator, describe the situation as a three-player bargaining problem $(F, v)$.
	\item[c.] Compute the Nash bargaining solution using your generalization.
	\item[d.] Explain why this solution is not satisfactory.
\end{enumerate}
\begin{solution}
\input{0902}
\end{solution}
\paragraph{3. } \textbf{(Theoretical exercise.)} Consider the optimization problem~\eqref{thm1} appearing in the statement of Theorem~1 (see Reminders).
\begin{enumerate}
	\item[a.] Formulate the KKT conditions of this optimization problem and interpret them geometrically.
	
	For simplicity, we will assume that $v = (0, 0)$ (without loss of generality), that $F$ is essential and that there exists a homogeneous and derivable function $f : \mathbb{R}^2 \rightarrow \mathbb{R}$ such that $F = \{ \, (x_1, x_2) \; | \; f(x_1, x_2) \leq 1 \, \}$.
	\item[b.] Use these conditions to prove Theorem~2 (see Reminders).
\end{enumerate}

	
\paragraph{4. } \textbf{(Theoretical exercise.)} Suppose that $x, y \in F, \; x_2 = v_2, \; y_1 = v_1$ and that $1/2 \, x + 1/2 \, y$ is a strongly efficient allocation in $F$. Show that $1/2 \, x + 1/2 \, y$ is the Nash bargaining solution of $(F,v)$.
	
\newpage

\section*{Reminders}

\begin{itemize}[leftmargin=*]
\renewcommand{\labelitemi}{$\bullet$}

	\item \textbf{The two-players bargaining problem}
	\vspace{.3cm}
	
	A bargaining problem with two players is defined by a pair $(F, v)$ where $F$ is a closed convex set of $\mathbb{R}^2$ which represents the set of possible payoffs and $v = (v_1, v_2) \in \mathbb{R}^2$ is the \emph{disagreement point} and represents the payoffs that both players would receive in the event of failure of the negotiation.
	
	Such a problem can be obtained for example from a game in strategic form $\Gamma = (\{ 1,2 \}, C_1, C_2, u_1, u_2)$, if the players can negotiate the establishment of a mechanism, and sign a contract binding them to the realization of the injunction imposed by this mechanism. In this case, the set $F$ can be defined by:
	\begin{align*}
		F = \Big\{ \big( \, u_1(\mu), \, u_2(\mu) \, \big) \; | \; \mu \in \Delta(C) \, \Big\}, \quad \text{where } \, u_i(\mu) \, = \, \sum_{c \in C} \mu(c) u_i(c).
	\end{align*}
	The disagreement point can be defined in different ways depending on the problem studied.
	
	We write $\phi(F, v) \in \mathbb{R}^2$ the solution of such a negotiation. 
	
	We also write $x \geq y$ iff $x_1 \geq y_1$ and $x_2 \geq y_2$ and $x > y$ iff $x_1 > y_1$ and $x_2 > y_2$. 
	
	We say that $(F, v)$ is \emph{essential} iff there exists at least one allocation $y$ in $F$ such that $y > v$. 
	
	\vspace{.3cm}
	\item \textbf{The axioms}
	\vspace{.3cm}

	\begin{enumerate}
	
		\item[(1)\phantom{'}] \textbf{Strong efficiency.} $\phi(F, v)$ is an allocation in $F$ and, for all $x$ in $F$, if $x \geq \phi(F, v)$, then $x = \phi(F, v)$.
		\item[(1')] \textbf{Weak efficiency.} $\phi(F, v) \, \in \, F$ and there does not exist any $y$ in $F$ such that $y > \phi(F, v)$.
		\item[(2)\phantom{'}] \textbf{Individual rationality.} $\phi(F, v) \geq v$.
		\item[(3)\phantom{'}] \textbf{Scale covariance.} For all $\lambda_1 > 0, \lambda_2 > 0, \gamma_1, \gamma_2$, let
		\begin{itemize}
			\item $G = \{ \, ( \, \lambda_1 x_1 + \gamma_1, \, \lambda_2 x_2 + \gamma_2 \, ) \; | \; (x_1, x_2) \, \in \, F \, \}$,
			\item $w = ( \, \lambda_1 x_1 + \gamma_1, \, \lambda_2 x_2 + \gamma_2 \, )$.
		\end{itemize}
		Then $\phi(G, w) = ( \, \lambda_1 \,  \phi_1(F, v) + \gamma_1, \,  \lambda_2 \,  \phi_2(F, v) + \gamma_2 \, )$.
		\item[(4)\phantom{'}] \textbf{Independence of irrational alternatives.} For any closed convex set $G$, if $G \subseteq F$ and $\phi(F, v) \, \in \, G$, then $\phi(G, v) = \phi(F, v)$.
		\item[(5)\phantom{'}] \textbf{Symmetry.} If $v_1 = v_2$ and $\{ \, (x_2, x_1) \; | \; (x_1, x_2) \, \in \, F \, \} = F$, then $\phi_1(F, v) = \phi_2(F, v)$.
		
	\end{enumerate}
	
	\vspace{.3cm}
	\item \textbf{The Nash bargaining Theorem}
	\vspace{.3cm}
	
	\textbf{Theorem 1.} There is a unique solution function $\phi(\cdot, \cdot)$ that satisfies Axioms $(1)$ to $(5)$ above. This solution function satisfies, for every two-person bargaining problem $(F, v)$:
	\begin{equation} \label{thm1}
		\phi(F, v) \, \in \, \underset{x \, \in \, F, \, x \geq v}{\mathrm{argmax}} \ (x_1 - v_1)(x_2 - v_2).
	\end{equation}
	
	\vspace{.3cm}
	\item \textbf{Interpersonal comparison of weighted utility}
	\vspace{.3cm}
	
	In real bargaining situations, the players often reason by comparing their respective utilities. They usually do so in two different ways:
	\begin{itemize}
		\item The ``equal gains'' principle: \emph{``You should do that for me because I do more for you.''} 
		
		In agreement with this principle, we define the \emph{$\lambda-$equalitarian solution} of a bargaining problem $(F, v)$ as the unique point $x$ of $F$ that is weakly efficient in $F$ and that satisfies the following condition:
		\begin{align*}
			\lambda_1 (x_1 - v_1) = \lambda_2 (x_2 - v_2).
		\end{align*}
		
		\item The ``greater good'' principle: \emph{``You should do this for me because it helps me more than it harms you.''}
		
		The \emph{$\lambda-$utilitarian solution} of a bargaining problem $(F, v)$ that derives from this principle is any solution function that yields $x \, \in \, F$ such that:
		\begin{align*}
			\lambda_1 \, x_1 + \lambda_2 \, x_2 = \max_{y \in F} \ (\lambda_1 \, y_1 + \lambda_2 \, y_2).
		\end{align*}
	\end{itemize}
	The following theorem expresses the fact that the Nash bargaining solution is a natural synthesis of the equal gains and greater good principles.
	
	\vspace{.3cm}
	\textbf{Theorem 2.} Let $(F, v)$ be an essential bargaining problem with two players and let $x$ be an allocation vector such that $x \, \in \, F$ and $x \geq v$. Then $x$ is the Nash bargaining solution for $(F, v)$ iff there exist $\lambda_1 > 0$ and $\lambda_2 > 0$ such that:
	\begin{align*}
		& \lambda_1 \, x_1 - \lambda_1 \, v_1 \hspace{.03cm} \ = \ \lambda_2 \, x_2 - \lambda_2 \, v_2 \qquad \; \text{and} \\
		& \lambda_1 \, x_1 + \lambda_2 \, x_2 \ = \ \max_{y \in F} \ (\lambda_1 \, y_1 + \lambda_2 \, y_2).
	\end{align*}
	
	\newpage
	\item \textbf{The KKT conditions}
	\vspace{.3cm}
	
	Let the optimization problem with $n$ variables, $p$ equality constraints and $q$ inequality constraints:
	\begin{equation*}
		\begin{array}{lll}
			\max && f(x), \\ 
			\mbox{s.t.} && g_i \hspace{.1cm} (x) = 0 \quad \forall \; i = 1, ..., p, \\
			&& h_j(x) \geq 0 \quad \forall \; j = 1, ..., q.
		\end{array}
	\end{equation*}
	The KKT conditions guarantee (under some regularity conditions) that there exist coefficients $\lambda_i$ $(i = 1, ..., p)$ and $\gamma_j$ $(j = 1, ..., q)$ such that:
	\begin{align*}
		\nabla f(x) & = \sum_{i = 1}^p \lambda_i \nabla g_i(x) - \sum_{j = 1}^q \gamma_j \nabla h_j(x), \\
		g_i(x) & = 0 \hspace{3.1cm} \forall \hspace{.07cm} \; i = 1, ..., p, \\
		h_j(x) & \geq 0 \hspace{3.1cm} \forall \; j = 1, ..., q, \\
		\gamma_j & \geq 0 \hspace{3.1cm} \forall \; j = 1, ..., q, \\
		\gamma_j h_j(x) & = 0 \hspace{3.1cm} \forall \; j = 1, ..., q.
	\end{align*}
		
\end{itemize}

\end{document}










