\documentclass{../ape}

\usepackage{../../linma2345}

\begin{document}

\session{4}{Decision Theory}

\section{}
It is common knowledge that Rapha\"el is gifted to get a good idea of the structure of complex games. Once again, he chose a subjective probability distribution over all possible states $\Omega$ of the game, which we write $p = (p(s))_{s \in \Omega}$, that will probably allow him to win the game once more. \TAtwo{}, who is generally less gifted than Rapha\"el for this kind of games, would like to use the same distribution as the latter. However, when asked to reveal it, Raphael can cheat and give any distribution $q$ in $\Delta(\Omega)$ of his choice to \TAtwo{}. Not content with the situation, \TAtwo{} would like to force Rapha\"el to reveal his true subjective probability distribution. For this, he promises Rapha\"el to give him some monetary reward $Y(q, s)$ when they discover the true state of the world $s$. (We here assume that Rapha\"el's utility function for money is linear and that he does not care about winning the game afterwards.)
\begin{itemize}
	\item Suppose that $q(s) = .2$ and $q(s') = .8$. Then, which among $Y(q,s)$ and $Y(q,s')$ should be the largest? Building on this observation, can you state a condition on the function $Y(q,s)$?
	\item Suppose now that $Y(q,s) = q(s)$. Will Rapha\"el reveal his true subjective probability distribution to \TAtwo{}? If not, what will he actually reveal?
	\item Same question if $Y(q, s) = \mathrm{ln}\,q(s)$.
	\item What should $Y(q, s)$ become if Rapha\"el's utility function was now given by $\mathop{\mathrm{sign}}(x)\cdot \sqrt{|x|}$ for any monetary amount $x$?
	\item Can you think of a real-life application where a similar situation as the one described above may arise?
\end{itemize}

\section{}
\TAtwo{} expresses the following preference relations for monetary lotteries:
%Arnaud exprime les relations de préférences suivantes pour des loteries monétaires :
\begin{align*}
	[600\text{\euro}] & \succ [400\text{\euro}] \\
							 			& \succ 0.90 \, [600\text{\euro}] + 0.10 \, [0\text{\euro}] \\
							 			& \succ 0.20 \, [600\text{\euro}] + 0.80 \, [0\text{\euro}] \\
							 			& \succ 0.25 \, [400\text{\euro}] + 0.75 \, [0\text{\euro}] \\
							 			& \succ [0\text{\euro}].
\end{align*}
Do these preferences make sense? Can you explain why on a theoretical basis?
%sont-elles consistantes avec une fonction d'utilité état-indé\-pen\-dante pour l'argent? Si oui, donner une telle fonction d'utilité. Si non, montrez que ces préférences violent un axiome.
\begin{solution}
One should show that there exist any utility function satisfying the axioms of the lotteries [0], [400] and [600].   Let denote their expected gain respectively $v_0,v_{400}$ and $v_{600}$. From the inequalities of the exercise, we can deduce the following inequalities : 
\begin{align}
&v_{400} > 0.9v_{600} +0.1 v_0  \\
&0.2v_{600} + 0.8v_0 > 0.25v_{400} +0.75v_0 \\
&v_{600} > v_0 
\end{align}
The inequality (2) can be rewritten as : 
$$v_{400} < 0.8v_{600} +0.2v_0$$ 
When combining it with inequality (1) we have that : 
\begin{align*}
    0.9v_{600} +0.1 v_0 &< v_{400} < 0.8v_{600} +0.2v_0\\
     &\Downarrow \\
     v_{600}&<v_0
\end{align*}

Which contradicts the inequality (3) . Hence, there exist none utility function satisfying the axioms.

\end{solution}
\section{}
Suppose that the utility of our gains depends on decisions and states as summarized in Table~\ref{ex7}. Let $(p(\theta_1), p(\theta_2))$ our subjective probability distribution on $\Omega = \{ \theta_1, \theta_2 \}$.
\begin{table}[h!]
	\begin{center}
		\begin{tabular}{ccccc}
			\hline
			Decision & & $\theta_1$ & & $\theta_2$ \\
			\hline
			$\alpha$ & & 15 & & 90 \\
			$\beta$ & & B & & 75 \\
			$\gamma$ & & 55 & & 40 \\
			\hline
		\end{tabular}
		\caption{Utility of the gains for $\theta_1$ and $\theta_2$} \label{ex7}
	\end{center}
\end{table}
\begin{itemize}
	%\item Let us assume first that $B = 35$. For which intervals of $p \triangleq p(\theta_1)$ are the decisions $\alpha, \beta$ and $\gamma$ optimal? Are some of these decisions strongly dominated? If so, by which randomized strategy?
	\item Let us assume first that $B = 35$. For which intervals of $p \triangleq p(\theta_1)$ are the decisions $\alpha, \beta$ and $\gamma$ optimal? %Is there a decision is \emph{never} optimal?
	\item For $B = 20$, show that $\beta$ is never an optimal decision.
	\item For $B = 20$, show that there is a value for $(p(\theta_1), p(\theta_2))$ such that  both the decisions $\alpha$ and $\gamma$ have a better expected utility payoff than  $\beta$.
	%\item For which values of the parameter $B$ is the decision $\beta$ strongly dominated?
\end{itemize}
\begin{solution}
\paragraph{a)}Let $p(\theta_1) \triangleq  p $ and $p(\theta_2) \triangleq  1-p $ be the probabilities of the events $\theta_1$ and $\theta_2$.  The expected gain $v$ of our decisions depend on those probabilities and is given by : 
\begin{align*}
    v_\alpha&= 15p + 90(1-p) = 90 - 75p\\
    v_\beta &= 35p + 75(1-p) = 75 - 40p\\
    v_\gamma&= 55p + 40(1-p) = 40 + 15p
\end{align*}
A decision is optimal when it gives a greater gain than other decisions. Based on this, one can found which decision is the best according to a given value of $p$ :
\begin{itemize}
    \item $\alpha$
    \begin{align*}
        v_\alpha &\geq  v_\beta \Leftrightarrow 90 - 75p \geq 75 -40 p \Rightarrow p\leq \frac{3}{7}\\
        v_\alpha &\geq  v_\gamma \Leftrightarrow 90 - 75p \geq 40 +15p \Rightarrow p\leq \frac{5}{9}
    \end{align*}
    The decision $\alpha$ is optimal when $p \in [0;\frac{3}{7}]$.
    \item $\beta$
    \begin{align*}
       v_\beta &\geq  v_\gamma \Leftrightarrow 75 - 40p \geq 40 +15p \Rightarrow p\leq \frac{7}{11}
    \end{align*}
    The decision $\beta$ is optimal when $p \in [\frac{3}{7},\frac{7}{11}]$. \newline The condition on $\gamma$ is deduced from the conditions above and is optimal when $p \in [\frac{7}{11},1]$.
\end{itemize}

\paragraph{b)} When $B = 20$, only the gain $v_\beta$ changes : 
\begin{align*}
    v_\alpha&= 15p + 90(1-p) = 90 - 75p\\
    v_\beta &= 20p + 75(1-p) = 75 - 55p\\
    v_\gamma&= 55p + 40(1-p) = 40 + 15p
\end{align*}

Let use the same strategy as above : 

\begin{itemize}
    \item $\alpha$
    \begin{align*}
        v_\alpha &\geq  v_\beta \Leftrightarrow 90 - 75p \geq 75 - 55 p \Rightarrow p\leq \frac{3}{4}\\
        v_\alpha &\geq  v_\gamma \Leftrightarrow 90 - 75p \geq 40 +15p \Rightarrow p\leq \frac{5}{9}
    \end{align*}
    The decision $\alpha$ is optimal when $p \in [0;\frac{5}{9}]$.
    \item $\beta$
    \begin{align*}
       v_\beta &\geq  v_\gamma \Leftrightarrow 75 - 55p \geq 40 +15p \Rightarrow p\leq \frac{1}{2}
    \end{align*}
    However, $\frac{1}{2} < \frac{5}{9}$ and we showed that under this condition, one would prefer to decide $ \alpha$ against $\beta$ and $\gamma$. Hence $\beta$ is never optimal. \newline
    Finally, $\gamma$ is optimal when $p\in [\frac{5}{9};1]$. 
\end{itemize}

\paragraph{c)}Let \textit{c} $\in[0;1]$ be a constant. The decision $\beta$ is strongly dominated by a mixed strategy $\delta = c\alpha + (1-c)\gamma $, if the payoffs of that strategy are strictly greater than the payoffs of $\beta$ : 
\begin{align*}
     w_{\theta_1} &= 15c + (1-c)55 > 20 \Leftrightarrow 40 c < 35 \Rightarrow c < \frac{7}{8} \\ 
     w_{\theta_2} &= 90c + (1-c)40 > 75 \Leftrightarrow 50 c > 35 \Rightarrow c > \frac{7}{10}
\end{align*}
The above inequations on $c$ are compatible which means that such a strategy $\delta$ exist. For example one can take c = $\frac{4}{5}$ and see indeed that $\beta$ is strongly dominated by $\delta = \frac{4}{5}\alpha + \frac{1}{5}$ : 
\begin{align*}
     w_{\theta_1} &= 23 > 20  \\ 
     w_{\theta_2} &= 80 > 75 
\end{align*}

\end{solution}
\section*{Reminders}

\begin{itemize}[leftmargin=*]
\renewcommand{\labelitemi}{$\bullet$}

	\item \textbf{The KKT conditions}
	\vspace{.3cm}
	
	Consider the optimization problem with $n$ variables, $p$ equality constraints and $q$ inequality constraints:
	\begin{equation*}
		\begin{array}{lll}
			\max && f(x), \\ 
			\mbox{s.t.} && g_i \hspace{.1cm} (x) = 0 \quad \forall \; i = 1, ..., p, \\
			&& h_j(x) \geq 0 \quad \forall \; j = 1, ..., q.
		\end{array}
	\end{equation*}
	The KKT conditions guarantee (under some regularity conditions) that, if $x^*$ is the optimum, there exist coefficients $\lambda_i$ $(i = 1, ..., p)$ and $\gamma_j$ $(j = 1, ..., q)$ such that:
	\begin{align*}
		\nabla f(x^*) & = \sum_{i = 1}^p \lambda_i \nabla g_i(x^*) - \sum_{j = 1}^q \gamma_j \nabla h_j(x^*), \\
		g_i(x^*) & = 0 \hspace{3.58cm} \forall \hspace{.07cm} \; i = 1, ..., p, \\
		h_j(x^*) & \geq 0 \hspace{3.58cm} \forall \; j = 1, ..., q, \\
		\gamma_j & \geq 0 \hspace{3.58cm} \forall \; j = 1, ..., q, \\
		\gamma_j h_j(x^*) & = 0 \hspace{3.58cm} \forall \; j = 1, ..., q.
	\end{align*}
		
\end{itemize}

\end{document}










