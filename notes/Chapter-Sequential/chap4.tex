\ifx \globalmark \undefined %% This is default.
\documentclass[a4paper,12pt]{report}

%%% PACKAGES %%%

\usepackage[french, english] {babel}	% langue principale
\usepackage[ansinew]{inputenc}
\usepackage[T1]{fontenc}		% Police contenant les caractères français
\usepackage{lmodern}			% plus beau
\usepackage[a4paper]{geometry}
	\geometry{hscale=0.75,vscale=0.8,centering}

\usepackage[hidelinks]{hyperref} % pas de couleurs ici
%
\usepackage[natbibapa]{apacite}

%% Maths
\usepackage{amsfonts} % Equations etc
\usepackage{amsmath}
\usepackage{mathtools}
\usepackage{amssymb}

\usepackage{cleveref}

%% Itemizes
\usepackage{enumerate}
\usepackage{enumitem}

%% Dessins & Plots
\usepackage[pdftex]{graphicx} %Images dans le PDF
\usepackage{epstopdf}
\usepackage{color, xcolor}
%% Dates
\usepackage{datetime}
\newdateformat{monthyeardate}{\monthname[\THEMONTH] \THEYEAR}
\usepackage{chessboard}
\usepackage{booktabs,tabularx,multirow}
\usepackage[babel=true,kerning=true]{microtype}
%> Tikz
\usepackage{tikz}
\usepackage{hf-tikz}
\usetikzlibrary{
	calc,
	arrows,
	arrows.meta,
	automata,
	shapes,
	snakes,
	positioning,
	decorations,
	decorations.text,
	fit,
	matrix,
	mindmap
	}
	\tikzstyle{noeud-std}=[draw,fill=black,circle,inner sep=0pt,minimum size=7pt]% 7pt est la taille des cercles noirs
\usepackage{tkz-graph}
\newcommand{\tikzmark}[2]{\tikz[overlay,remember picture,baseline=(#1.base)] \node (#1) {#2};}
\newcommand{\Highlight}[1][submatrix]{%
    \tikz[overlay,remember picture]{
    \node[highlight,fit=(left.north west) (right.south east)] (#1) {};}
    }
\tikzset{%
  highlight/.style={rectangle,rounded corners,fill=ocre!50,draw,
    fill opacity=0.5,thick,inner sep=0pt}
}
\newcommand{\mytikzmark}[2]{\tikz[overlay,remember picture, baseline=(#1.base)] \node (#1) {#2};}


%% Meta
\usepackage{etoolbox}

% Debugging purposes: Catch when a reference is missing
\makeatletter
	\patchcmd{\@setref}{\bfseries ??}{{\color{red}[\texttt{\detokenize{ #3 }}]}}{}{%
	  \GenericWarning{}{Failed to patch \protect\@setref}}
	\patchcmd{\@citex}{\bfseries ?}{{\color{red}[\texttt{\detokenize{ #3 }}]}}{}{%{{\color{red}\texttt{\@citeb}}}{}{%
	  \GenericWarning{}{Failed to patch \protect\@citex}}
\makeatother

%% CUSTOM ENVIRONMENTS
% Packages
\usepackage{framed}
\usepackage{mdframed}
\usepackage[amsmath,thref, framed]{ntheorem}

%% Counters for theorem.
% The current choice is that all environments share a counter within chapters.
% It that regards, it becomes easy to navigate the document.

\newcounter{theo}[chapter]\setcounter{theo}{0}
\renewcommand{\thetheo}{\arabic{chapter}.\arabic{theo}}



% The following is adapted from https://texblog.org/2015/09/30/fancy-boxes-for-theorem-lemma-and-proof-with-mdframed/
% \newenvironment{name}[args]{begin_def}{end_def}
\newenvironment{generic_theo}[2][] % 2 arguments, and one is optional. The first argument will say if we have a Thm or somth, the second is the name of the thm.
	{% begin_def
		\refstepcounter{theo}%
		\ifstrempty{#2} % looking at the name of the thm
		{
			\mdfsetup{
				frametitle={
					\tikz[baseline=(current bounding box.east),outer sep=0pt]
					\node[anchor=east,rectangle,fill=white!20, draw = black!20, line width = 2pt]
					{\strut #1~\thetheo};}
			} % The #1 will say Theorem, or other.
		}
		{
			\mdfsetup{
				frametitle={
				\tikz[baseline=(current bounding box.east),outer sep=0pt]
				\node[anchor=east,rectangle,fill=white!20, draw = black!20, line width = 2pt]
				{\strut #1~\thetheo:~#2};}
			}
		}
		\mdfsetup{innertopmargin=10pt,linecolor=black!20,
					linewidth=2pt,topline=true,
					frametitleaboveskip=\dimexpr-\ht\strutbox\relax
				}
\begin{mdframed}[]\relax}{\end{mdframed}}


%% Proofs (slightly different)
\newenvironment{generic_proof}[2][] % 2 arguments, and one is optional. The first argument will say if we have a Thm or somth, the second is the name of the thm.
	{% begin_def
		\refstepcounter{theo}%
		\ifstrempty{#2} % looking at the name of the thm
		{
			\mdfsetup{
				frametitle={
					\tikz[baseline=(current bounding box.east),outer sep=0pt]
					\node[anchor=east,rectangle,fill=black!20]
					{\strut #1~\thetheo};}} % The #1 will say Theorem, or other.
		}
		{
			\mdfsetup{
				frametitle={
				\tikz[baseline=(current bounding box.east),outer sep=0pt]
				\node[anchor=east,rectangle,fill=black!20]
				{\strut #1~\thetheo:~#2};}}
		}
		\mdfsetup{innertopmargin=10pt,linecolor=black!20,
					linewidth=1pt,topline=true,
					frametitleaboveskip=\dimexpr-\ht\strutbox\relax
				}
\begin{mdframed}[]
}{\end{mdframed}}

%% Examples/Exercices
\newenvironment{generic_ex}[2][] % 2 arguments, and one is optional. The first argument will say if we have a Thm or somth, the second is the name of the thm.
	{% begin_def
		\refstepcounter{theo}%
		\ifstrempty{#2} % looking at the name of the thm
		{
			\mdfsetup{
				frametitle={
					\tikz[baseline=(current bounding box.east),outer sep=0pt]
					\node[anchor=east,rectangle,fill=black!20]
					{\strut #1~\thetheo};}} % The #1 will say Theorem, or other.
		}
		{
			\mdfsetup{
				frametitle={
				\tikz[baseline=(current bounding box.east),outer sep=0pt]
				\node[anchor=east,rectangle,fill=black!20]
				{\strut #1~\thetheo:~#2};}}
		}
		\mdfsetup{innertopmargin=10pt,linecolor=black!20,
					linewidth=1pt,topline=true,
					frametitleaboveskip=\dimexpr-\ht\strutbox\relax
				}
\begin{mdframed}[]
}{\end{mdframed}}



% The different environments that we need (Main stuff are highlighted.
\newenvironment{theorem}[1][]{\begin{generic_theo}[Theorem]{#1}}{\end{generic_theo}}
\newenvironment{lemma}[1][]{\begin{generic_theo}[Lemma]{#1}}{\end{generic_theo}}
\newenvironment{definition}[1][]{\begin{generic_theo}[Definition]{#1}}{\end{generic_theo}}
\newenvironment{notation}[1][]{\begin{generic_theo}[Notation]{#1}}{\end{generic_theo}}
\newenvironment{proposition}[1][]{\begin{generic_theo}[Proposition]{#1}}{\end{generic_theo}}
\newenvironment{procedure}[1][]{\begin{generic_theo}[Procedure]{#1}}{\end{generic_theo}}
\newenvironment{hypothese}[1][]{\begin{generic_theo}[Hypothesis]{#1}}{\end{generic_theo}}


% These ones, let's not overblow them
% The true reason why I'm doing this is that there may be floats in Exercices and Examples...
% You can't have  begin{figures} in mdframed env, it crashes.
%\newtheorem{notation}[theo]{Notation}
\newcounter{axiomc}[chapter]\setcounter{axiomc}{0}
\renewcommand{\theaxiomc}{\arabic{chapter}.\arabic{axiomc}}

\newtheorem{axiom}[axiomc]{Axiom}
\newtheorem{proof}[theo]{Proof}
\newtheorem{exercise}[theo]{Exercise}
\newtheorem{example}[theo]{Example}



%% CUSTOM Commands

% Identify TAs
\newcommand{\TAone}{Mahsa}
\newcommand{\TAtwo}{Beno\^it}

\newcommand{\reels}{\mathbb{R}}

\DeclareMathOperator*{\argmin}{arg\,min}
\DeclareMathOperator*{\argmax}{arg\,max}

% Nice brackets
\newcommand\parent[1]{\left(#1\right)}
\newcommand\abs[1]{\left\lvert#1\right\rvert}
\newcommand\norm[1]{\left\lVert#1\right\rVert}
\newcommand\bracket[1]{\left\{#1\right\}}
\newcommand\squared[1]{\left[#1\right]}

\DeclarePairedDelimiter{\floor}{\lfloor}{\rfloor}
\DeclarePairedDelimiter{\ceil}{\lceil}{\rceil}
\usepackage{eurosym}
\usepackage{siunitx}
\DeclareSIUnit{\EUR}{\text{\euro}}
\sisetup{
  per-mode = fraction,
  inter-unit-product = \ensuremath{{}\cdot{}},
}



\usepackage{textcomp}
\let\texteuro\euro





	\begin{document} %% Crashes if put after (one of the many mysteries of LaTeX?).	
\else 	
\fi



\chapter{Sequential Equilibria} \label{chap:Seq}

{\large{\itshape
``Life is a journey, not a destination.''} --- Ralph Waldo Emerson.\\
}
  {\small{\itshape
Chapter based on pages 154 to 177 of the book  ``Game theory - Analysis of conflict'' by R. Myerson.}\\
}

The Nash equilibrium is a powerful tool for the analysis of games. When \emph{intelligent and rational} players have to make decisions, they will make sure to maximize their own payoffs (rationality) knowing that the other players will do the same (intelligence and rationality of the others).
Thus, every player will play at a Nash Equilibrium.\\
Nash equilibria are computed from the strategic form of a game. The main idea behind the use of the strategic form is that it provides a concise summary of the game, highlighting which strategies can be played, and what are the payoffs resulting of a choice of strategies.  Since the aim of rational players is to  maximize their payoffs, we are tempted to believe that the study of the strategic form is sufficient for understanding the outcome of games.\\
In this chapter, we show that this is not the case. A game is a sequential process. A strategic form ignores sequentiality, and hides away some features of the game. The main consequence of this is that, even if when studying the game \emph{a priori} a strategy seems to be a best response to the opponents' strategies, \emph{its implementation may not be rational}. This is due to the fact that rational players wish to maximize their payoff \emph{at every move}. This is essentially due to the fact that our rational players actually take several decisions.  And, if we certainly expect that their different decisions have to be coherent with one another, we cannot deduce from this assumption that the decisions about every actions they will take are taken at once, at the beginning of the game.\\
In this chapter, we begin with examples showing that 
some Nash equilibria may translate into decisions (in the initial sequential game) which are irrational, and thus should naturally be excluded (Section \ref{ch4:sec:motivations}). Then, we will define several notions relative to extensive forms game (Section \ref{ch4:sec:extform}). In particular, we are going to be using the concept of \emph{behavioural strategies}. Finally, we will study  \emph{sequential rationality} and,  the key solution concept of the chapter, the \emph{sequential equilibrium}.   





\section{Rational and irrational equilibrium}
\label{ch4:sec:motivations}
Using simple examples, we are going to bring intuition for answering the questions \emph{what are the rational moves in a game?} 

Our first example represents a ``good'' scenario, where the Nash equilibrium is rational.
\begin{example}[The card game: bluff or no bluff?]
\label{ch4:ex:cardgame}
Let us go back to the card game of Example \ref{chap2:example:game}.
\begin{figure}[!ht]
\begin{center}
\begin{tikzpicture}
\node[noeud-std] (src) {}
   [sibling distance=7cm]
   child {node[noeud-std] (n1a) {}
        [sibling distance=3cm]
         child[level distance=2cm]{node[noeud-std] (n2c1){} 
		 [sibling distance=3cm]     
         	child[level distance=1cm]{node[noeud-std,fill=white] (leaf1){} }
         	child[level distance=1cm]{node[noeud-std,fill=white] (leaf2){} }
         }
         child[level distance=1cm]{node[noeud-std,fill=white] (leaf3){} }
   }
	child {node[noeud-std] (n1b) {}
        [sibling distance=3cm]
         child[level distance=2cm]{node[noeud-std] (n2c2){} 
		 [sibling distance=3cm]     
         	child[level distance=1cm]{node[noeud-std,fill=white] (leaf4){} }
         	child[level distance=1cm]{node[noeud-std,fill=white] (leaf5){} }
         }
         child[level distance=1cm]{node[noeud-std,fill=white] (leaf6){} }
   }
   ;
\node[above=5pt] at (src) {$0$ };  
\node[above=5pt] at (n1a) {$1.a$ };  
\node[above=5pt] at (n1b) {$1.b$ };
\node[above=5pt] at (n2c1) {$2.c$ };
\node[above=5pt] at (n2c2) {$2.c$ };
  
\node[above=-20pt] at (leaf1) {$ 4 \, / 0$ }; 
\node[above=-20pt] at (leaf2) {$ 3 \, / 1$ };  
\node[above=-20pt] at (leaf3) {$ 3 \, / 1$ };  
\node[above=-20pt] at (leaf4) {$ 0 \, / 4$ };  
\node[above=-20pt] at (leaf5) {$ 3 \, / 1$ };  
\node[above=-20pt] at (leaf6) {$ 1 \, / 3$ };    

 
\node[above left] at ($(src)!{0.25}!(n1a)$) {red: $0.5$};
\node[above right] at ($(src)!{0.25}!(n1b)$) {black: $0.5$};
\node[above right] at ($(n1a)!{0.25}!(leaf4)$) {Fold};
\node[above left] at ($(n1a)!{0.25}!(n2c1)$) {Raise};
\node[above left] at ($(n2c1)!{0.25}!(leaf1)$) {meet};
\node[above right] at ($(n2c1)!{0.25}!(leaf2)$) {pass};
\node[above right] at ($(n1b)!{0.25}!(leaf6)$) {fold};
\node[above left] at ($(n1b)!{0.25}!(n2c2)$) {raise};
\node[above left] at ($(n2c2)!{0.25}!(leaf4)$) {meet};
\node[above right] at ($(n2c2)!{0.25}!(leaf5)$) {pass};
\end{tikzpicture}
\end{center}


\caption{Extensive form of the game in Example \ref{chap2:example:game}.}
\label{chap4:example:figtree}
\end{figure}
 We computed the Nash equilibrium of the game in Example \ref{chap3:ex:cardgame}, which is given by the mixed strategy profile 
$$(1/3[Rr] + 2/3[Rf], 2/3[m] + 1/3[p]).$$ 
The equilibrium corresponds to the following decisions:
\begin{enumerate}
\item At information state $1.a$ (Player 1 has a red card), the move ``Raise'' is chosen.
\item At information state $1.b$ (Player 1 has a black card), the move ``raise'' is chosen with probability $\frac{1}{3}$, and ``fold'' with probability $\frac{2}{3}$.
\item At information state $2.c$, the move ``meet'' is chosen with probability $\frac{2}{3}$ and the move ``pass'' is chosen with probability $\frac{1}{3}$. 
\end{enumerate} 
Now, let us put ourselves in the shoes of Player 2. Imagine for the moment that we had \emph{no idea} whatsoever of the strategy of player $1$.  An idealistic strategy for us would be to ``pass'' on a red card, and to ``meet'' on a black card. This strategy is however unavailable to us - since we have no information on the card's color. This is translated in the game by the fact that information state $2.c$ is shared between two decision nodes.\\
However, it is quite easy to see (left as exercise) that  if we believed that the card was red with probability $\alpha$, and black with probability $1-\alpha$, then our rational move would be to
\begin{itemize}
\item play ``meet'' if $\alpha < \frac{3}{4}$,
\item play ``pass'' if $\alpha > \frac{3}{4}$,
\item and play anyone of ``meet'' or ``pass'' if $\alpha = \frac{3}{4}$.
\end{itemize}
In our case, the probability $\alpha = \frac{3}{4}$ is actually enforced by the strategy of player $1$ at the equilibrium. 
Indeed, the probability of reaching information state $2.c$ is given by 
$$ p(2.c) = \frac{1}{2}p(R | 1.a) + \frac{1}{2} p(r |  1.b) = \frac{1}{2} + \frac{1}{2} \frac{1}{3} = \frac{2}{3}. $$
Thus, the probability of being at the upper branch of the game and at node $2.c$ is 
$$ p(2.c |  \text{red}) = \frac{\frac{1}{2}p(R | 1.a)}{p(2.c)} = \frac{3}{4},$$
where the above is nothing else that an application of \emph{Bayes} formula.\\

In this example, we saw that the move of player 2 would need to depend on the probability that he believes to be either in the upper branch (red card) of the game or the lower branch (black card). In any scenario, a rational move can be computed. The Nash Equilibrium of the game corresponds to a case where player 2 believes there are 3 chances over 4 that player 1 has a red card should he raise.


\end{example}

\begin{example}[A delicate situation]
\label{ch4:ex:badEqu}
Let us now consider the extensive form game presented in Figure \ref{ch4:fig:badEquilibrium}.
\begin{figure}[!ht]
\begin{center}
\begin{tikzpicture}
\node[noeud-std] (noeud0) {}
	[sibling distance=4cm]		
	child[level distance=1cm]{node[noeud-std,fill=white] (noeud1){}}
	child[level distance=1cm] {node[noeud-std] (noeud2) {}
		[sibling distance=3cm]
			child[level distance=1cm]{node[noeud-std,fill=white] (noeud3){}}
			child[level distance=1cm]{node[noeud-std,fill=white] (noeud4){}}
		}
	
    ;

\node[above=5pt] at (noeud0) {$1.a$}; 
\node[above=-25pt] at (noeud0) {$X_0$}; 
\node[above left] at ($(noeud0)!{0.4}!(noeud1)$) {\texttt{$y_1$}};
\node[above right] at ($(noeud0)!{0.4}!(noeud2)$) {\texttt{$x_1$}};
\node[above=-20pt] at (noeud1) {$1\, / \, 9$}; 
\node[above=5pt] at (noeud2) {$2.b$};
\node[above=-25pt] at (noeud2) {$X_1$};   
\node[above left] at ($(noeud2)!{0.4}!(noeud3)$) {\texttt{$y_2$}};
\node[above right] at ($(noeud2)!{0.4}!(noeud4)$) {\texttt{$x_2$}};
\node[above=-20pt] at (noeud3) {$0\, / \, 2$};  
  \node[above=-20pt] at (noeud4) {$2\, / \, 3$};
\end{tikzpicture}
\end{center}
\caption{This game has two Nash Equilibria: $([x1],[x2])$ and $([y1], [y2])$. The second equilibrium assumes an irrational move from player 2...}
\label{ch4:fig:badEquilibrium}
\end{figure}

The strategic form is given by 
\begin{center}
\begin{tabular}{c|cc}
 & x2 & y2 \\
\hline
x1 & 2, 3 & 0, 2 \\
y1 & 1, 9 & 1, 9
\end{tabular}
\end{center}

Of the two Nash Equilibria, $([y_1], [y_2])$ is disputable. 
Indeed, let us put ourselves in the shoes of player 1, and consider the game again through the eye of decision theory. The uncertainty here is one move picked by player $2$; so let us say player 2, at information state $2.b$, plays $x_2$ with probability $\alpha$, and $y_2$ with probability $1-\alpha$. 
We need to compare our two moves:
\begin{itemize}
\item $y_1$ wins us a payoff of $1$,
\item $x_1$ wins us a payoff of $2 \alpha$.
\end{itemize}
Thus, we prefer $x_1$ over $y_1$ if $\alpha > \frac{1}{2}$. 
Now, let us try and get more information on the possible values of $\alpha$. If player $2$ is at information state $2.b$ then his choice is trivial: playing $x_2$ wins him a payoff of $3$, dominating the decision $y_2$! Consequently, player 1,  knowing that player 2 is rational, should infer $\alpha = 1$, and thus prefer $x_1$ over $y_1$. \\
We conclude that as long as both players are rational and intelligent, the equilibrium $(y_1, y_2)$ cannot occur.

Remark that this conclusion would be hidden from us if we only studied the strategic form. Here, the extensive form contains information allowing us to discard an equilibrium in the name of rationality. 
It is hopeless to attempt to obtain the same conclusion from using the strategic form alone. Indeed, the game of Figure \ref{ch4:fig:badEquilibrium2} has the same strategic form, and here both equilibria seem acceptable. Moreover, by playing $[y2]$, player 2 now guarantees himself a payoff of $9$ instead of $3$, which is thus the rational thing to do here. 
\begin{figure}[!ht]
\begin{center}
\begin{tikzpicture}
\node[noeud-std] (n2b) {}
   [sibling distance=5cm]
   child[level distance=1cm]{node[noeud-std] (n1a1) {}
        [sibling distance=3cm]
        child[level distance=1cm]{node[noeud-std,fill=white](x2x1){}}  
        child[level distance=1cm]{node[noeud-std,fill=white](x2y1){}}
        }
	child[level distance=1cm]{node[noeud-std] (n1a2) {}
        [sibling distance=3cm]
        child[level distance=1cm]{node[noeud-std,fill=white](y2x1){}}  
        child[level distance=1cm]{node[noeud-std,fill=white](y2y1){}}
        }
   ;
\node[above=5pt] at (n1a1) {$1.a$}; 
\node[above=5pt] at (n1a2) {$1.a$};   
\node[above=5pt] at (n2b) {$2.b$ };
  
\node[above=-20pt] at (x2x1) {$ 2\, / 3$ }; 
\node[above=-20pt] at (x2y1) {$ 1 \, / 9$};  
\node[above=-20pt] at (y2x1) { $0 \, / 2$};  
\node[above=-20pt] at (y2y1) {$ 1 \, / 9$ }; 

\node[above left] at ($(n2b)!{0.4}!(n1a1)$) {$x_2$};
\node[above right] at ($(n2b)!{0.4}!(n1a2)$) {$y_2$};
\node[above left] at ($(n1a1)!{0.4}!(x2x1)$) {$x_1$};
\node[above right] at ($(n1a1)!{0.4}!(x2y1)$) {$y_1$};
\node[above left] at ($(n1a2)!{0.4}!(y2x1)$) {$x_1$};
\node[above right] at ($(n1a2)!{0.4}!(y2y1)$) {$y_1$};

\end{tikzpicture}
\end{center}
\caption{A game with the same strategic form as the one of Figure \ref{ch4:fig:badEquilibrium}, for which $([y1],[y2])$ is a Nash Equilibrium which is rational to implement.}
\label{ch4:fig:badEquilibrium2}
\end{figure}

\end{example}

Throughout this chapter, we will study games having the \emph{perfect recall} property. At any step of the game, all the players of the game are always able to recall the whole history of their past moves.

\section{Behavioural strategies}
\label{ch4:sec:extform}
The example of the previous section shows that Nash Equilibria may or may not be rational when analyzed through the extensive form. Indeed, a rational agent is going to play a \emph{rational move at every information state}.\\

We will now introduce a formalism for studying this behaviour. Our formalism needs to enable us to access several elements. First and foremost, we will need players, information states and decisions. Additionally, for  a player at a given information state, there is uncertainty about his exact position in the extensive form tree. Thus, we also need to access the different nodes of the game. 
 To do so, we recall the following relevant objects for a game in extensive form $\Gamma_e$ from Chapter \ref{chap:Models}.
\begin{definition}[Game in extensive form]
 For a game in extensive form $\Gamma_e$, we define the following:
\begin{itemize}
\item $N$ is the set of players.
\item $S =  \bigcup_{i \in N} S_i$ is the set of all the information states, where $S_i$ are the information states of $i \in N$, and $S_i \cap S_j = \emptyset$ if $i \neq j$.
\item $D = \bigcup_{s \in S} D_s$ is the set of all possible moves, where $D_s$ is the set of moves available at information state $s \in S$.
\end{itemize}
Regarding the structure of the tree, we have
\begin{itemize}
\item $Y_s$ is the  set of nodes having information state $s \in S$.
\item $X = \bigcup_{s \in S} Y_s$ is the set of nodes in the tree.
\item $\Omega \subset X$ is the set of leafs of the tree, and $X^0 \in X$ is its root.
\end{itemize}
\label{chap4:def:GameInExtForm}
\end{definition}

Again, there is a clear distinction between information states ($S$) and  nodes in the tree ($X$). 
\begin{example}
Consider the  game in extensive form of Figure \ref{ch4:fig:exampleFormalism}.
\begin{figure}[!ht]
\begin{center}
\begin{tikzpicture}
\node[noeud-std] (noeud0) {}
   [sibling distance=7cm]
   child {node[noeud-std] (noeud1) {}
        [sibling distance=3cm]
         child[level distance=1cm]{node[noeud-std,fill=white] (noeud2){} }
         child[level distance=1cm]{node[noeud-std,fill=white] (noeud3){} }
   }
   child {node[noeud-std] (noeud4) {}
        [sibling distance=4cm]
        child[level distance=1cm]{node[noeud-std] (noeud5){} 
         	[sibling distance=3cm]
         	child[level distance=1cm]{node[noeud-std,fill=white] (noeud6){} }
         	child[level distance=1cm]{node[noeud-std,fill=white] (noeud7){} }}
        child[level distance=1cm]{node[noeud-std] (noeud8){} 
         	[sibling distance=3cm]
         	child[level distance=1cm]{node[noeud-std,fill=white] (noeud9){} }
         	child[level distance=1cm]{node[noeud-std,fill=white] (noeud10){} }}
   };
   
   \node[above=3pt] at (noeud0) {$1.a$};  
\node[above=3pt] at (noeud1) {$2.b$};  
\node[above=-20pt] at (noeud2) {$3 \, / 2$};   
\node[above=-20pt] at (noeud3) {$2 \, / 3$};  
\node[above=3pt] at (noeud4) {$2.b$};   
\node[above=3pt] at (noeud5) {$1.c$};   
\node[above=-20pt] at (noeud6) {$4 \, / 1 $};   
\node[above=-20pt] at (noeud7) {$2 \, / 3$};  
\node[above=3pt] at (noeud8) {$1.c$}; 
\node[above=-20pt] at (noeud9) {$0 \, / 5$}; 
\node[above=-20pt] at (noeud10) {$3 \, / 2$};   
 

\node[above=-25pt] at (noeud0) {$X_0$};   
\node[above=-25pt] at (noeud1) {$X_1$};   
\node[above=-35pt] at (noeud2) {$X_2$};   
\node[above=-35pt] at (noeud3) {$X_3$};   
\node[above=-25pt] at (noeud4) {$X_4$};   
\node[above=-25pt] at (noeud5) {$X_5$};   
\node[above=-35pt] at (noeud6) {$X_6$};   
\node[above=-35pt] at (noeud7) {$X_7$};   
\node[above=-25pt] at (noeud8) {$X_8$};   
\node[above=-35pt] at (noeud9) {$X_9$};   
\node[above=-35pt] at (noeud10) {$X_{10}$};
   
\node[above left] at ($(noeud0)!{0.5}!(noeud1)$) {$a_1$};
\node[above right] at ($(noeud0)!{0.5}!(noeud4)$) {$b_1$};
\node[above left] at ($(noeud1)!{0.5}!(noeud2)$) {$x_2$};
\node[above right] at ($(noeud1)!{0.5}!(noeud3)$) {$w_2$};
\node[above left] at ($(noeud4)!{0.5}!(noeud5)$) {$x_2$};
\node[above right] at ($(noeud4)!{0.5}!(noeud8)$) {$w_2$};
\node[above right] at ($(noeud5)!{0.5}!(noeud7)$) {$y_1$};
\node[above left] at ($(noeud5)!{0.5}!(noeud6)$) {$z_1$};
\node[above right] at ($(noeud8)!{0.5}!(noeud10)$) {$y_1$};
\node[above left] at ($(noeud8)!{0.5}!(noeud9)$) {$z_1$};
\end{tikzpicture}
\end{center}
\caption{Information state are above the nodes, and the nodes' names are below.}
\label{ch4:fig:exampleFormalism}
\end{figure}

We have
\begin{enumerate}
\item  $N = \{1,2\}$,
\item $S_1 = \{1.a, 1.c\}$, $S_2 = \{2.b\}$, and $S = S_1 \bigcup S_2$. 
\item $D_{1.a} = \{a_1,b_1\}$, $D_{2.b} = \{x_2,w_2\}$, $D_{1.c} = \{z_1, y_1\}$. 
\end{enumerate}
Regarding the nodes, we have:
\begin{enumerate}[resume]
\item $X = \{X_0, \ldots, X_{10}\}$, 
\item $Y_{1.a} = \{X_0\}$, $Y_{2.b} = \{X_1, X_4\} $, $Y_{1.c} = \{X_5, X_8\}$, 
\item $\Omega = \{X_2, X_3, X_6, X_7, X_9, X_{10}\}$ and $X^0 = X_0$. 
\end{enumerate}
\label{ch4:ex:Formalism}
\end{example}

Next, we are going to define notions that relate to the players strategies. 
Recall that in Chapter \ref{chap:Models}, a \emph{pure strategy} was defined as a choice of one move per information state. More formally, the set of pure strategies for player $i \in N$ is 
$$ C_i = \times_{s \in S_i} D_s.$$
For each $c_i \in C_i$ and each information state $s \in S_i$, we can then define $c_i(s)$ has the move played at information state $s$ for the pure strategy $c_i$.
Then, by considering possible randomizations between strategies, we define the set of \emph{mixed strategies} for player $i \in N$ as
\begin{equation}
\Delta C_i = \Delta\left ( \times_{s \in S_i} D_s \right ),
\label{ch4:eq:mixedStrat}
\end{equation}
and finally, the set of all mixed strategy profiles as $\times_{i \in N} \Delta C_i = \times_{i \in N}  \left( \Delta\left ( \times_{s \in S_i} D_s \right ) \right )$. 

\begin{definition}
A \emph{behavioural strategy} for player $i \in N$ specifies, at each information state $s \in S_i$, a probability distribution on the moves $D_s$ of the player. That is, the set of all behavioural strategies for $i \in N$ is 
\begin{equation}
\times_{s \in S_i}  \Delta D_s.
\label{ch4:eq:behav}
\end{equation}
For $s \in S_i$, $d \in D_s$ and $\tau_i \in \times_{s \in S_i}  \Delta D_s$, $\tau_i(d)$ is the probability of occurrence of the move $d$ at information state $s$ when implementing $\tau$. \\
We define the set of all \emph{behavioural strategy profiles} as
$$ \times_{i \in N} \times_{s \in S_i} \Delta D_s.$$
\end{definition}

Mathematically speaking, mixed strategies (\ref{ch4:eq:mixedStrat}) and behavioural strategies (\ref{ch4:eq:behav}) have  a different nature - one is a set of distributions on a cartesian product, the other is a cartesian product on distributions. However, there are some clear links between the two notions.

\begin{example}
Consider the game of Example \ref{ch4:ex:Formalism}.
We take for the example the following mixed strategy for player 1:
$$( 0.5[a1,y1] + 0.5 [b1,z1] ). $$
When implementing its strategy, player 1 is first going to randomize between $a1$ and $b1$. Then, if he did do $b1$, he will have to play again at information state $1.c$, where he should play $z1$. 
Thus, the mixed strategy actually corresponds to the following behavioural strategy:
$$ (0.5 [a1] + 0.5 [b1], [z1]), $$
where the first part corresponds to the move at informations state $1.a$, and the second to the move at information state $1.b$. \\
From a mixed strategy profile, it is quite easy to compute a corresponding behavioural strategy. However, a same behavioural strategy may correspond to several strategy profiles.
\end{example}



\begin{definition}
Consider an extensive form game $\Gamma^e$. For any behavioural strategy profile $\tau$ and two nodes $x,y \in X$, we define 
\begin{enumerate}
\item $P(y | x, \tau)$ as the probability to reach node $y$ from node $x$ by implementing $\tau$.
\item $P(y|\tau) = P(y | x^0, \tau)$ as the probability of reaching node $y$ from the root of the tree by implementing $\tau$.
\end{enumerate}
\end{definition}

\begin{example}
Consider again the game of Example \ref{ch4:ex:Formalism}. Take a general behavioural strategy profile of the form 
$$\tau_1 = (\alpha [a_1] + (1-\alpha) [b_1], \, \beta [y_1] + (1-\beta) [z_1]), \, \tau_2 = (\gamma [x_2] + (1-\gamma) [w_2]).$$
The implementation of this strategy actually infers that each path in the extensive form tree is going to be followed with some probabilities. This is represented in Figure \ref{ch4:fig:ExtFormProba}.
\begin{figure}[!ht]
\begin{center}
\begin{tikzpicture}
\node[noeud-std] (noeud0) {}
   [sibling distance=7cm]
   child {node[noeud-std] (noeud1) {}
        [sibling distance=3cm]
         child[level distance=1cm]{node[noeud-std,fill=white] (noeud2){} }
         child[level distance=1cm]{node[noeud-std,fill=white] (noeud3){} }
   }
   child {node[noeud-std] (noeud4) {}
        [sibling distance=4cm]
        child[level distance=1cm]{node[noeud-std] (noeud5){} 
         	[sibling distance=3cm]
         	child[level distance=1cm]{node[noeud-std,fill=white] (noeud6){} }
         	child[level distance=1cm]{node[noeud-std,fill=white] (noeud7){} }}
        child[level distance=1cm]{node[noeud-std] (noeud8){} 
         	[sibling distance=3cm]
         	child[level distance=1cm]{node[noeud-std,fill=white] (noeud9){} }
         	child[level distance=1cm]{node[noeud-std,fill=white] (noeud10){} }}
   };
   
%   \node[above=3pt] at (noeud0) {$1.a$};  
%\node[above=3pt] at (noeud1) {$2.b$};  
%\node[above=-20pt] at (noeud2) {$3 \, / 2$};   
%\node[above=-20pt] at (noeud3) {$2 \, / 3$};  
%\node[above=3pt] at (noeud4) {$2.b$};   
%\node[above=3pt] at (noeud5) {$1.c$};   
%\node[above=-20pt] at (noeud6) {$4 \, / 1 $};   
%\node[above=-20pt] at (noeud7) {$2 \, / 3$};  
%\node[above=3pt] at (noeud8) {$1.c$}; 
%\node[above=-20pt] at (noeud9) {$0 \, / 5$}; 
%\node[above=-20pt] at (noeud10) {$3 \, / 2$};   
 

\node[above=-25pt] at (noeud0) {$X_0$};   
\node[above=-25pt] at (noeud1) {$X_1$};   
\node[above=-25pt] at (noeud2) {$X_2$};   
\node[above=-25pt] at (noeud3) {$X_3$};   
\node[above=-25pt] at (noeud4) {$X_4$};   
\node[above=-25pt] at (noeud5) {$X_5$};   
\node[above=-25pt] at (noeud6) {$X_6$};   
\node[above=-25pt] at (noeud7) {$X_7$};   
\node[above=-25pt] at (noeud8) {$X_8$};   
\node[above=-25pt] at (noeud9) {$X_9$};   
\node[above=-25pt] at (noeud10) {$X_{10}$};
   
\node[above left] at ($(noeud0)!{0.5}!(noeud1)$) {$\alpha$};
\node[above right] at ($(noeud0)!{0.5}!(noeud4)$) {$1-\alpha$};
\node[above left] at ($(noeud1)!{0.5}!(noeud2)$) {$\gamma$};
\node[above right] at ($(noeud1)!{0.5}!(noeud3)$) {$1-\gamma$};
\node[above left] at ($(noeud4)!{0.5}!(noeud5)$) {$\gamma$};
\node[above right] at ($(noeud4)!{0.5}!(noeud8)$) {$1-\gamma$};
\node[above right] at ($(noeud5)!{0.5}!(noeud7)$) {$\beta$};
\node[above left] at ($(noeud5)!{0.5}!(noeud6)$) {$1-\beta$};
\node[above right] at ($(noeud8)!{0.5}!(noeud10)$) {$\beta$};
\node[above left] at ($(noeud8)!{0.5}!(noeud9)$) {$1-\beta$};
\end{tikzpicture}
\label{ch4:fig:exampleFormalism2}
\end{center}
\caption{Implementing a behavioural strategy.}
\label{ch4:fig:ExtFormProba}
\end{figure}
Here we have for example:
\begin{enumerate}
\item $P(X_{10} | \tau) = (X_{10} | X_0, \tau) = (1-\alpha)  (1-\gamma) (\beta)$, which is simply the probability of occurrence of the path from $X_0$ to $X_{10}$.
\item $P(X_7 | X_4,\tau ) = \gamma \beta$. Note that this can be understood as the probability of reaching $X_7$ conditioned upon the fact that we visited $X_4$.
\end{enumerate}
\end{example}

\subsection{Behavioural and Mixed strategies representation.}

In this section, we formalize the link between the behavioural and mixed representation of a same strategy. 
Intuitively, these two ways of describing the probabilistic behaviour of an agent must be coherent with the laws of probability, and in particular, their relation must satisfy Bayes' formula.
\begin{definition}
Consider an extensive form game $\Gamma^e$. For any \emph{mixed} strategy profile $\sigma$ and two nodes $x,y \in X$, we define 
\begin{enumerate}
\item $\tilde P(y | x, \sigma)$ as the probability to reach node $y$ from node $x$ by implementing $\sigma$.
\item $ \tilde P(y|\sigma) = \tilde P(y | x^0, \sigma)$ as the probability of reaching node $y$ from the root of the tree by implementing $\sigma$.
\end{enumerate}
\end{definition}

\begin{definition}[Compatible pure strategies]
For $s \in S_i$,  the set of pure strategies compatible with $s$ is defined as 
$$ C_i^*(s) = \{c_i \in C_i : \exists c_{-i} \in C_{-i} : \sum_{x \in Y_s} \tilde P(x | (c_{-i}, c_i)) > 0\}.$$
For $s \in S_i$ and $d \in D_s$, the set of pure strategies compatible with $s$ and using $d$ is 
$$ C_i^{**}(d,s) = \{c_i \in C_i^*(s) : c_i(s) = d\}. $$
\end{definition}

\begin{definition}[Behavioural representation of mixed strategy]
For a player $i \in N$, let $\sigma \in \Delta C_i$ be a mixed strategy.
We say that a behavioural strategy $\tau$ is a \emph{behavioural representation} of $\sigma$ if, 
$$ \forall s \in S_i, d \in D_s, \, \tau_s(d)\cdot\left ( \sum_{e_i \in C_i^*(s)} \sigma_i(e_i) \right ) = \sum_{c_i \in C_i^{**}(d,s)} \sigma_i(c_i). $$
Conversely, we say that $\sigma$ is a mixed representation of $\tau$ if 
$$\forall c_i \in C_i, \, \sigma_i(c_i) = \prod_{s \in S_i} \tau_s(c_i(s)).$$
\end{definition}

\begin{definition}[Mixed representation of behavioural strategy]
For a player $i \in N$, let $\tau \in \times_{s \in S_i} \Delta  D_s$ be a behavioural strategy.
We say that a mixed strategy $\sigma$ is a \emph{mixed representation} of $\tau$ if, 
$$\forall c_i \in C_i, \, \sigma_i(c_i) = \prod_{s \in S_i} \tau_s(c_i(s)).$$
\end{definition}

\begin{proposition}
Consider a game in extensive form $\Gamma^e$ and its strategic form $\Gamma$. 
At a leaf node $x \in \Omega$, let $w_i(x)$ be the payoff for player $i$ in $\Gamma^e$. 
Let $\tau$ be a behavioural strategy and $\sigma$ be a mixed representation of $\tau$.
Then
$$ \forall i \in N, \sum_{x \in \Omega} P(x | \tau) w_i(x) = u_i(\sigma),$$
that is, the payoff of $\tau$ in the extensive form is the same than the one of $\sigma$ in the normal form.
\end{proposition}

\begin{theorem}[Kuhn]
Any two mixed strategies in $\Delta(C_i)$ that are behaviourally equivalent are also payoff equivalent.
\end{theorem}

The theorem above has some interesting consequences. If rational and intelligent players only seek to maximize their payoffs, then this justifies the focus on strategic form games as in Chapter \ref{chap:Nash}. However, the Nash Equilibrium alone does not allow us to discard all irrational strategies. We require something more, a notion of equilibrium where players are rational at every information state of the game. 

\section{Sequential Equilibrium}
\label{ch4:sec:Eq}

Our goal is now to construct a solution concept allowing to understand the behaviour of rational and intelligent agents engaging in a sequential game. More formally, we would like to understand which are the behavioural strategy profiles that correspond to rational and intelligent decisions from all players in a game.

\subsection{Multi-agent representation}

A first approach would be to adapt the concept of Nash equilibrium so that it applies to behavioural strategies instead of mixed strategies. This can be done almost trivially simply by describing an extensive form game $\Gamma^e$ through its \emph{multi-agent} representation.
\begin{definition}[Multi-agent representation]
Given a game in extensive form $\Gamma^e$ with players $N$, information states $S = \bigcup_{i \in N} S_i$ and moves $D = \times_{s \in S} D_s$ the multi-agent representation is a normal-form game
$\Gamma(S,(D_s)_{s \in S}, (v_s)_{s \in S})$ where
\begin{itemize}
\item The set of players of the multi-agent representation is the set of the types $S$, 
\item The set of pure strategies for type $s \in S$ is the set of moves $D_s$, 
\item For any player $i \in N$, for  any strategy profile $d = (d_s)_{s \in S}$, for any $s \in S_i$, 
the payoff for type $s$ is given by
$$ v_s(d) = u_i(c(d)),$$
where $c(d)$ is a mixed representation of $d$ and $u_i(c(d))$ is the payoff of player $i$ for this mixed-strategy. 
\end{itemize} 
\end{definition}
\begin{example}
\label{ch4:ex:cardGameMultiAgent}
The multi-agent representation of the card game of Example \ref{ch4:ex:cardgame} is given by
 \begin{center}
 \begin{tabular}{l|cc}
 & 1.a: Raise & 1.a: Fold \\
\begin{tabular}{l}
\\
\hline 2.c: meet \\
2.c: pass
\end{tabular} &
\begin{tabular}{cc}
\hline 1.b: raise & 1.b: fold \\
\hline 2, 2, 2 & 2.5, 2.5, 1.5 \\
3, 3, 1 & 2, 2, 2
\end{tabular}&
\begin{tabular}{cc}
\hline 1.b: raise & 1.b: fold \\
\hline 1.5, 1.5, 2.5 & 2, 2, 2 \\
3, 3, 1 & 1, 1, 3
\end{tabular}
\end{tabular}
.
\end{center} 
\end{example}
\begin{theorem}
Given an extensive form game $\Gamma^e$ and a Nash Equilibrium $\sigma$ of its strategic representation, any behavioural representation of $\sigma$ is a Nash Equilibrium of its multi-agent representation.
\end{theorem}

 The converse, as shown in the next example, is not true.

\begin{example}
\label{ch4:ex:cardGame2}
Consider the following variation of the card game of Example \ref{ch4:ex:cardgame}.
\begin{itemize}
\item Both players bet 1 EUR.
\item Player 1 picks a red or a black card. If it is a red card, he raises the bet and puts an additional 1 EUR on the table. If it is a black card, either he raises by 1 EUR, or folds. If he folds, player 2 wins the 1 EUR of player 1, and the game ends.
\item If player 1 raises, then player 2 privately decides whether he meets the bet or passes by putting 1 EUR in a closed envelope, or not.
\item Then, player 1 shows his card. If the card is red, then he wins the envelope. If it is black, then he has one additional move to make. Either he surrenders, at which point he takes back 1 EUR and leaves the rest to player 2. Or he attacks, and if player 2 met the raise, then player 2 wins everything and if player 2 passed on the bet, player 1 wins everything.
\end{itemize}

After a few simplifications, we can model this game with the following extensive form.

\begin{center}
\begin{tikzpicture}
\node[noeud-std] (noeud0) {}
   [sibling distance=7cm]
   child {node[noeud-std] (noeud1) {}
        [sibling distance=3cm]
         child[level distance=1cm]{node[noeud-std,fill=white] (noeud2){} }
         child[level distance=1cm]{node[noeud-std,fill=white] (noeud3){} }
   }
   child {node[noeud-std] (noeud4) {}
        [sibling distance=3cm]
        child[level distance=2cm]{node[noeud-std] (noeud5){} 
         	[sibling distance=5cm]
         	child[level distance=1cm]{node[noeud-std] (noeud6){} 
				[sibling distance=3cm]
				child[level distance=1cm]{node[noeud-std,fill=white] (noeud7){} }
    	     	child[level distance=1cm]{node[noeud-std,fill=white] (noeud8){} }
         	}
         	child[level distance=1cm]{node[noeud-std] (noeud9){}          	
         		[sibling distance=2cm]
         		child[level distance=1cm]{node[noeud-std,fill=white] (noeud10){} }
         		child[level distance=1cm]{node[noeud-std,fill=white] (noeud11){} }
         	}
        	}	
  		child[level distance=1cm]{node[noeud-std, fill=white] (noeud12){}}       	
   	 	}
   ;
\node[above=5pt] at (noeud0) {$0$ };  
\node[above=-20pt] at (noeud1) {$2.b$ };  
\node[above=-20pt] at (noeud2) {$ 4 \, / 0$ };   
\node[above=-20pt] at (noeud3) {$3 \, / 1$ };  
\node[above=5pt] at (noeud4) {$1.a$ };   
\node[above=5pt] at (noeud5) {$2.b$ };   
\node[above=5pt] at (noeud6) {$1.c$ };   
\node[above=-20pt] at (noeud7) {$0 \, / 4$ };  
\node[above=-20pt] at (noeud8) {$1 \, / 3$ }; 
\node[above=5pt] at (noeud9) {$1.c$ }; 
\node[above=-20pt] at (noeud10) {$3 \, / 1$ };   
\node[above=-20pt] at (noeud11) {$1 \, / 3$ }; 
\node[above=-20pt] at (noeud12) {$1 \, / 3$}; 

 
\node[above left] at ($(noeud0)!{0.25}!(noeud1)$) {red: $0.5$};
\node[above right] at ($(noeud0)!{0.25}!(noeud4)$) {black: $0.5$};
\node[above left] at ($(noeud1)!{0.25}!(noeud2)$) {meet};
\node[above right] at ($(noeud1)!{0.25}!(noeud3)$) {pass};
\node[above left] at ($(noeud4)!{0.25}!(noeud5)$) {raise};
\node[above left] at ($(noeud5)!{0.25}!(noeud6)$) {meet};
\node[above right] at ($(noeud5)!{0.25}!(noeud9)$) {pass};
\node[above left] at ($(noeud9)!{0.4}!(noeud10)$) {surr};
\node[above right] at ($(noeud9)!{0.4}!(noeud11)$) {atck};
\node[above left] at ($(noeud6)!{0.4}!(noeud7)$) {surr};
\node[above right] at ($(noeud6)!{0.4}!(noeud8)$) {atck};
\node[above right] at ($(noeud4)!{0.25}!(noeud12)$) {fold};
\end{tikzpicture}
\end{center}

The multi-agent representation  is given by
 \begin{center}
 \begin{tabular}{l|cc}
 & 1.a: raise & 1.a: fold \\
\begin{tabular}{l}
\\
\hline 2.b: meet \\
2.b: pass
\end{tabular} &
\begin{tabular}{cc}
\hline 1.c: surr & 1.c: atck \\
\hline 2, 2, 2 & 2.5, 2.5, 1.5 \\
3, 3, 1 & 2, 2, 2
\end{tabular}&
\begin{tabular}{cc}
\hline 1.c: surr & 1.c: atck \\
\hline 2.5, 2.5, 1.5 & 2.5, 2.5, 1.5 \\
2, 2, 2 & 2, 2, 2
\end{tabular}
\end{tabular}
.
\end{center} 
where the payoffs are written as $(v_{1.a} / v_{1.c} / v_{2.b})$. Observe that the behavioural strategy profile (fold, pass, atck) is a Nash Equilibrium for the multi-agent representation. The equilibrium brings a payoff of 2 to player 1. However, if player 1  chooses the behavioural strategy (raise, surr) instead of (fold, atck), then he would obtain a payoff of 3!. As a conclusion, the mixed representation of   (fold, pass, atck) is not a Nash Equilibrium for the strategic form of the game, even though it is an equilibrium of the multi-agent representation.  
\end{example}


\subsection{Sequential rationality}

We now set out to develop the notion of sequential rationality, for characterizing the behavioural strategies that may be played by rational players. We want to capture the fact that a player will make a sequence of moves, and that each move in the sequence is a rational move. Roughly speaking, this is going to be translated into the need for each move in the sequence to be a move that maximizes the expected utility of the player.

\begin{definition}[Expected Utility at node $x$]
Consider a game in extensive form $\Gamma^e$.
Given a player $i \in N$, a node $x \in X$ and a behavioural strategy profile $\tau$, we let 
$$U_i(\tau, x) = \sum_{y \in \Omega} P(y | \tau, x) w_i(y),$$
where $w_i(y)$ is the payoff of i at the terminal node $y \in \Omega$. $U_i(\tau, x)$ denotes the expected utility of $i$ at $x$ following $\tau$.  
\end{definition}

Now of course, at any given node $x \in X$ if a player can make a decision, this player wants to maximize his utility. However, players only have access to their information states, they don't necessarily know at which node they are. As it is the case in Example \ref{ch4:ex:cardgame}, the behaviour of a player will then depend of a probability distribution he assigns to nodes sharing an information state. We call these \emph{belief vectors}. 

\begin{definition}[Belief vectors]
Consider a game in extensive form $\Gamma^e$ with players $N$, types $S = (S_i)_{i \in N}$ and nodes $X$. Let $Y_s \subset X$ be the set of nodes of the tree that have information state $s$. 
A belief vector $\pi$ is any element of the set
$$ \times_{s \in S} \Delta Y_s.$$
For $\pi \in  \times_{s \in S} \Delta Y_s $, $i \in N$ $s \in S_i$, and $x \in Y_s$, $\pi_s(x)$ is the probability assigned by player $i$ to be at node $x$ given the fact that $i$ knows his information state $s$.
\end{definition}

Because players know only their information state when playing a game, instead of maximizing their expected utility at every node (which they cannot do), the next best thing to do is to use their belief vector and maximize their \emph{sequential values}.

\begin{definition}[Sequential value]
Consider a game in extensive form $\Gamma^e$, a belief vector $\pi$ for this game and a behavioural strategy profile $\tau$.
For a player $i \in N$ at an information state $s \in S_i$, the \emph{sequential value} of the move $d \in D_s$ is given by
$$ U_i(d| s, (\tau, \pi) ) = \sum_{x \in Y_s} \pi_s(x) U_i((\tau_{-i}, d), x). $$ 
\end{definition}

We can finally define the concept of sequential rationality:
\begin{definition}[Sequential Rationality]
Consider a game in extensive form $\Gamma^e$, a belief vector $\pi$ for this game and a behavioural strategy profile $\tau = (\tau_i)_{i \in N}$. The strategy $\tau_i$ is sequentially rational for player $i \in N$ at information state $s \in S_i$ according to $\pi$ if 
$$\tau_i(s) \in \argmax_{d \in D_s}  U_i(d| s, ( \tau, \pi) ).$$
We say that $\tau$ is sequentially rational if it is rational at every information state according to $\pi$.
\end{definition}

Clearly, sequential rationality needs to be part of any reasonable definition of an equilibrium for extensive form games.
\begin{example}
\label{ch4:ex:seqRatio}
Consider the game of Example \ref{ch4:ex:badEqu}. Let $a$ be the node at information state $1.a$, and $b$ be the node at information state $2.b$. 
Clearly, we have $\pi_{1.a}(a) = 1$ and $\pi_{1.b}(b) = 1$, since there are one node per information state. These beliefs vectors are actually independent of the strategy played. 
Consider the profile $\tau = (y1, y2)$. We already focused on this profile in Example \ref{ch4:ex:badEqu}, as it is a Nash Equilibrium, but it is not rational. \\
We can indeed verify that 
$$ \argmax_{d \in \{x_2, y_2\}} U_2(d | 2.b, (\tau, \pi)) \, = \{x_2\}. $$
\end{example}

The example above shows an interesting feature of belief vectors: even if we consider the strategy profile $(y_1, y_2)$, for which information state $2.b$ is not encountered, $\pi_{1.b}$ is still well defined. In the next section, we investigate what constitutes valid \emph{belief vectors}. 
\subsection{Weak consistency: Rationality at states that are believed to occur}


How does a rational player compute his believe vector?
It is clear that belief vectors and behavioural strategy profiles need to be related. A fairly natural approach  would be to use Bayes formula and to compute belief vectors from strategy profiles. 
More precisely, one's first idea is to compute the belief vector through the formula:
$$ \pi_s(y) = \frac{P(y | \tau, x^0)}{\sum_{x \in Y_s} P(x | \tau, x^0)}. $$
However, the above can fail, simply because the denominator could be equal to zero. The above fails, because some elements of the belief vector may not be well defined. This is the case in Example \ref{ch4:ex:seqRatio}, where $\pi_{2.b}(b) = 1$ but  $ \sum_{y \in Y_{2.b}} P(y | \tau) = 0$. 

Bayes formula cannot be used when $\sum_{x \in Y_s} P(x | \tau, x^0) = 0$, corresponding to cases where a given information state cannot be reached (see Example \ref{ch4:ex:seqRatio}). 

\begin{definition}[Weak Consistency]
A strategy profile $\pi$ is \emph{weakly consistent} with a behavioural strategy profile $\tau$  if it
satisfies 
$$\forall s \in S, \, \forall y \in Y_s, \, \left (\sum_{x \in Y_s} P(x | \tau, x^0)\right ) \pi_s(y) =  P(y | \tau, x^0).$$
\end{definition}
% Faire remarque avec Th de Decision

\begin{theorem}
Consider a game in extensive form $\Gamma^e$. Let $\tau$ be an equilibrium in behavioural strategy and $\pi$ be a belief vector \emph{weakly consistent} with $\tau$. 
For all player $i \in N$, let 
$$
S^0_i = \{s \in S_i : \sum_{x \in Y_s} P(x| \tau) > 0\}
$$ 
be the set of all information states occurring with positive probability.
The equilibrium strategy profile $\tau$ is \emph{sequentially rational} at all information states in $\bigcup_{i \in N} S^0_i$ according to the belief vector $\pi$.
\label{ch4:th:WeakRatio}
\end{theorem}

However, the scope of the result remains limited. When applied to Example \ref{ch4:ex:seqRatio} for the strategy $([y_1],[y_2])$, one can verify through the theorem that the equilibrium is rational for player 1. But it does not tell us anything for player 2.\\
In order to tackle information states that occur with probability 0, we need a stronger notion of consistency for belief vectors.

\subsection{Sequential equilibrium: Rationality at all information states}

As explained above,  all of the difficulties in computing beliefs lies in dealing with information states that arise with zero probability. \\
But how can we characterize the sequential rationality at nodes of  probability zero?  Bayes' formula is not useful, because the zero denominator prevents us to compute an actual value.  However, this does not imply that any arbitrary value is admissible.  The following definition characterizes the admissible belief vectors, not only at positive probability nodes, but also at zero-probability nodes.
\begin{definition}
A belief vector $\pi$ is \emph{fully consistent} with a behavioural strategy profile $\tau$ in the game $\Gamma^e$ if there exists a sequence of profiles $(\tau^k)_{k = 1}^\infty$ such that
\begin{itemize}
\item $\tau^k \in \times_{s \in S} \Delta^0 D_s, \forall k = 1,2,\ldots$ \\ (non-zero probabilities on every move), 
\item  $\tau(d) = \lim_{k \rightarrow \infty} \tau^k(d),  \forall i \in N, \forall s \in S_i, \forall d \in D_s $\\
 (sequence converges to the strategy), 
\item $ \pi_s(x) = \lim_{k \rightarrow \infty} \frac{P(x | \tau^k, x^0)}{\sum_{y \in Y_s}P(y| \tau^k, x^0)}, \forall s \in S, \forall x \in Y_s, $\\
(belief vector is the limit of the sequence of beliefs given by Bayes rule).
\end{itemize}
\label{ch4:def:StrongCons}
\end{definition}

The sequences appearing in the definition can be understood as the result of an iterative process , where players go through a succession of strategies that do not discard any moves, for eventually converging to a given behavioural strategy profile. At every step of the process, one can compute his belief vector simply by applying Bayes formula. A belief vector strongly consistent with a strategy profile is then obtained as the limit of the intermediate profiles. \\

The following is easily obtained from the definition of strong consistency.
\begin{proposition}
If a belief vector $\pi$ is strongly consistent with  a strategy $\tau$, it is also weakly consistent.
\end{proposition}

We are finally able to define the concept of \emph{sequential equilibria}:
\begin{definition}
A \emph{sequential equilibrium} of an extensive form game $\Gamma^e$ is a pair $(\tau, \pi)$ where $\tau \in \times_{s \in S} \Delta D_s$ is sequentially rational according to $\pi$, and $\pi \in \times_{s \in S} \Delta Y_s$ is a belief vector strongly consistent with $\tau$. 
\label{ch4:def:SeqEq}
\end{definition}

We can show that a sequential equilibrium is always an equilibrium in behavioural strategy.

\begin{theorem}
If $(\tau, \pi)$ is a sequential equilibria, then $\tau$ is an equilibrium in behavioural strategy.
\end{theorem}

\begin{theorem}
For any finite extensive form game, the set of sequential equilibria is non-empty.
\end{theorem}

\begin{example}[The signalling game]
\label{ch4:ex:signalling}
We consider a two-player game. Player 1 is a student seeking to be hired for a job. He has some private information - he is either highly motivated by the job and will be very productive (type $h$ for \emph{high profile candidate}) or maybe he just wants to get the money 
and do the least amount of job possible (type $l$ for \emph{low profile candidate}).\\ Based on the result of an exam, Player 2 needs to decide whether to hire player 1 ($H$) or not ($N$). \\
Regarding this test, player 1 can decide to study for it ($S$) or decide to relax instead ($R$).\\
The game is represented by the following extensive form.
\begin{center}
\begin{tikzpicture}
\node[noeud-std] (noeud0) {}
   [sibling distance=8.5cm]
      child {node[noeud-std] (noeud1) {}
      [sibling distance=4cm]
      child[level distance=1cm]{node[noeud-std] (noeud2){} 
         	[sibling distance=2.5cm]
         	child[level distance=1.5cm]{node[noeud-std,fill=white] (noeud3){} }
         	child[level distance=1.5cm]{node[noeud-std,fill=white] (noeud4){} }}
      child[level distance=1cm]{node[noeud-std] (noeud5){} 
         	[sibling distance=2.5cm]
         	child[level distance=1.5cm]{node[noeud-std,fill=white] (noeud6){} }
         	child[level distance=1.5cm]{node[noeud-std,fill=white] (noeud7){} }}
   }
   child {node[noeud-std] (noeud8) {}
       [sibling distance=4cm]
       child[level distance=1cm]{node[noeud-std] (noeud9){} 
         	[sibling distance=2.5cm]
         	child[level distance=1.5cm]{node[noeud-std,fill=white] (noeud10){} }
         	child[level distance=1.5cm]{node[noeud-std,fill=white] (noeud11){} }}
       child[level distance=1cm]{node[noeud-std] (noeud12){} 
         	[sibling distance=2.5cm]
         	child[level distance=1.5cm]{node[noeud-std,fill=white] (noeud13){} }
         	child[level distance=1.5cm]{node[noeud-std,fill=white] (noeud14){} }}   
   }
;

\node[above=5pt] at (noeud0) {$0$}; 
\node[above=-25pt] at (noeud0) {$X_0$}; 
\node[above=5pt] at (noeud1) {$1.a$};
\node[above=-25pt] at (noeud1) {$X_1$};   
\node[above=5pt] at (noeud2) {$2.c$};  
\node[above=-25pt] at (noeud2) {$X_2$};
\node[above=-20pt] at (noeud3) {$-6\, / \, 0$};     
\node[above=-35pt] at (noeud3) {$X_3$}; 
\node[above=-20pt] at (noeud4) {$0 \, / \, -2$};   
\node[above=-35pt] at (noeud4) {$X_4$}; 
\node[above=5pt] at (noeud5) {$2.d$}; 
\node[above=-25pt] at (noeud5) {$X_5$};
\node[above=-20pt] at (noeud6) {$-1 \, / \, 0$};  
\node[above=-35pt] at (noeud6) {$X_6$};   
 \node[above=-20pt] at (noeud7) {$ 4 \, / \, -2$};  
\node[above=-35pt] at (noeud7) {$X_7$};
\node[above=5pt] at (noeud8) {$1.b$}; 
\node[above=-25pt] at (noeud8) {$X_8$};
\node[above=5pt] at (noeud9) {$2.c$};
\node[above=-25pt] at (noeud9) {$X_9$};
\node[above=-35pt] at (noeud10) {$X_{10}$}; 
\node[above=-20pt] at (noeud10) {$-2 \, / \,0$};   
\node[above=-35pt] at (noeud11) {$X_{11}$}; 
\node[above=-20pt] at (noeud11) {$3 \, / \, 4$}; 
\node[above=5pt] at (noeud12) {$2.d$}; 
\node[above=-25pt] at (noeud12) {$X_{12}$};
\node[above=-35pt] at (noeud13) {$X_{13}$};   
\node[above=-20pt] at (noeud13) {$-2\,/\,0$};
\node[above=-35pt] at (noeud14) {$X_{14}$}; 
\node[above=-20pt] at (noeud14) {$4 \, / \, 4$}; 

  
\node[above left] at ($(noeud0)!{0.25}!(noeud1)$) {\texttt{l : 0.5}};
\node[above right] at ($(noeud0)!{0.25}!(noeud8)$) {\texttt{h : 0.5}};
\node[above right] at ($(noeud1)!{0.25}!(noeud5)$) {\texttt{R}};
\node[above left] at ($(noeud1)!{0.3}!(noeud2)$) {\texttt{S}};
\node[above right] at ($(noeud8)!{0.3}!(noeud12)$) {\texttt{r}};
\node[above left] at ($(noeud8)!{0.3}!(noeud9)$) {\texttt{s}};
\node[above right] at ($(noeud2)!{0.3}!(noeud4)$) {\texttt{H}};
\node[above left] at ($(noeud2)!{0.3}!(noeud3)$) {\texttt{N}};
\node[above right] at ($(noeud5)!{0.3}!(noeud7)$) {\texttt{h}};
\node[above left] at ($(noeud5)!{0.3}!(noeud6)$) {\texttt{n}};
\node[above right] at ($(noeud9)!{0.3}!(noeud11)$) {\texttt{H}};
\node[above left] at ($(noeud9)!{0.3}!(noeud10)$) {\texttt{N}};
\node[above right] at ($(noeud12)!{0.3}!(noeud14)$) {\texttt{h}};
\node[above left] at ($(noeud12)!{0.3}!(noeud13)$) {\texttt{n}};
\end{tikzpicture}
\end{center}

The strategic form is given by:
\begin{center}
\begin{tabular}{c|cccc}
 & Nn & Nh & Hn & Hh \\
\hline
Ss & -4, 0 & -4, 0 & 1.5, 1 & 1.5, 1 \\
Sr & -4, 0 & -2, 2 & -1, -1 & 2, 1\\
Rs & -1.5, 0 & 1, -1 & 1, 2& 3.5, 1\\
Rr & -1.5, 0 & 4, 1 & 1.5, 0 & 4, 1 \\
\end{tabular}
\end{center}

A preliminary analysis of the game\footnote{Which may have been written by a relaxed student...} reveals that $Nn$ is dominated: the employer is trying to hire someone. So is $Sr$. 


After reduction, we have the following game.
\begin{center}
\begin{tabular}{c|ccc}
  & Nh  & Hn & Hh \\
\hline
Ss  & -4, 0 & 1.5, 1 & 1.5, 1 \\
Rs  &  1, -1 & 1, 2& 3.5, 1 \\
Rr  &  4, 1  & 1.5, 0 & 4, 1 \\
\end{tabular}
\end{center}
 The strategies $Nh$ and $Ss$ are only weakly dominated here, and thus they are conserved. From this game, we will highlight candidate sequential equilibria, and check whether or not they are indeed equilibria.\\
  The first step is to figure our what are the Nash equilibria of the game. There are only three of them, and they are pure strategy profiles: $([Ss], [Hn])$,  $([Rr], [Hh])$ and $([Rr] , [Nh])$.
Then, we translate these equilibria in behavioural strategies, and look for strongly consistent belief vectors. Finally, we verify whether or not these behavioural representations are sequentially rational for these belief vectors. When this is the case, we obtain sequential equilibria.
\begin{enumerate}
\item Pure equilibrium $([Ss], [Hn])$. 
\end{enumerate}
The behavioural representation of this equilibrium is $\tau_1 = ([S],[s])$, $\tau_2 = ([H],[n])$.
The belief vectors for information states $1.a$, $1.b$, and $2.c$ are uniquely defined: 
$$
\pi_{1.a}(X_1) = \pi_{1.b}(X_8) =1, \,  \pi_{2.c}(X_2) = \pi_{2.c}(X_9) = \frac{1}{2}.
$$
For this behavioural strategy, the information state $2.d$ cannot be reached: 
$$P(X_5 | \tau) + P(X_{12} | \tau) = 0. $$
By setting for example $\pi_{2.d}(X_5) = \pi_{2.d}(X_{12}) = \frac{1}{2}$, the obtained belief vector is \emph{weakly consistent}. Thus, we know from Theorem \ref{ch4:th:WeakRatio} that the strategy is sequentially rational at information states $1.a, 1.b$ and $2.c$.\\
Yet the action at information state $2.d$ is very important. Indeed, if it is not rational for player 2 to play the move $n$, then we fall back to a situation alike the one of Example \ref{ch4:ex:badEqu}, where the Nash Equilibrium is not rational.\\
Thus, assume player $1$ plays $\tau_1^{\alpha, \beta} = (1-\alpha[S] + \alpha [R],1-\beta[s] + \beta [r] )$, and let $\tau = (\tau_{1}^{\alpha, \beta}, \tau_2)$\footnote{Here, $\tau_2 \not \in \times_{s \in S_2} \Delta^0 D_s$. This, however, does not impact the beliefs of the players.}. 
For all $\alpha, \beta \in [0,1]$, we can compute 
$$\pi_{2.d}(X_5) = \frac{P(X_5 | \tau)}{ P(X_5 | \tau) + P(X_{12} | \tau) }  = \frac{\alpha}{\alpha + \beta}.$$
For $Hn$ to be sequentially rational at information state $2.d$, there must be $\alpha, \beta$ small such that the \emph{sequential value} of the move $n$ should be greater or equal to that of $h$. Here, we have
$$ U_2(n | 2.d, (\tau, \pi)) = \frac{0\alpha+0\beta}{\alpha+\beta} = 0 ,$$
whereas 
$$U_2(h | 2.d, (\tau, \pi)) = \frac{-2 \alpha+ 4\beta}{\alpha+\beta}. $$
We wish to investigate the case where it would be rational for player $2$ to play $n$ over $h$ at information state $2.d$. Thus, we require $-2 \alpha+ 4\beta \leq 0$, which is the case when  $\beta \leq 0.5\alpha$. For this to be true, we may set $\beta = \alpha/2$, in which case we obtain
$$ \pi_{2.d}(X_5) = \frac{2}{3}, \, \pi_{2.d}(X_{12}) = \frac{1}{3}. $$
So far, we have seen that the behavioural representation of our considered equilibrium was \emph{sequentially rational} when following the belief vector defined above. In order to verify the definition of \emph{sequential equilibrium}, we just need to show that the belief vector is strongly consistent. 

To do so, we set $$ \tau_1^k = (1 - \alpha_k [S] + \alpha_k [R], 1-\beta_k [s] + \beta_k [r]), \tau_2^k = (\epsilon_k [N] + 1 - \epsilon_k [H], \epsilon_k [h] +  1-\epsilon_k [n]), $$
and we wish to find expressions for $\alpha_k, \beta_k, \epsilon_k$ such that, first, $(\tau_1^k , \tau_2^k )\rightarrow (\tau_1, \tau_2)$ as $k \rightarrow \infty$, and second, $\pi$ is the limit of $\pi^k$, obtained by applying Bayes formula with $\tau_1^k, \tau_2^k$. \\
In our case, it suffice to choose
$$ \alpha_k =  \frac{1}{k}, \, \beta_k = \frac{0.5}{k}, \, \epsilon_k = \frac{1}{k}.$$


\begin{enumerate}[resume]
\item The equilibrium $([Rr], [Hh])$.
\end{enumerate}

This situation is similar to the one above, and can be solved using the same approach. 
The problem lies in defining the beliefs at state $2.c$, which is the node reached when a candidate  (unexpectedly) studies instead of relaxing. Intuitively, our strategy would then be rational under the condition that our beliefs would be ``the candidate is then likely to have a high profile''.

Thus, assume player $1$ plays $\tau_1^{\alpha, \beta} = (\alpha[S] + 1-\alpha [R],\beta[s] + 1-\beta [r] )$, and let $\tau = (\tau_{1}^{\alpha, \beta}, \tau_2)$. Note that as $\alpha, \beta  \rightarrow 0$, $\tau$ converges to a behavioural representation of $([Rr], [Hh])$.
For all $\alpha, \beta \in [0,1]$, we can compute 
$$\pi_{2.c}(X_2) = \frac{P(X_2 | \tau)}{ P(X_2 | \tau) + P(X_9 | \tau) }  = \frac{\alpha}{\alpha + \beta}.$$ 

We may then compare the sequential values of the moves $H$ versus $N$, which should satisfy the inequality
$$ U_2(N | 2.c, \pi) = \frac{0 (\alpha + \beta)}{\alpha + \beta} \leq \frac{( - 2\alpha + 4 \beta)}{\alpha + \beta} = U_2(H | 2.c , \pi),  $$
implying $\beta  \geq \alpha/2$. 
We are now able to show that the behavioural strategy $([R] + [r], [H] + [h])$, coupled with the strongly consistent belief vector generated e.g. by the sequence of strategies
$$ \tau_1^k = (\alpha_k [S] + 1-\alpha_k [R], \beta_k [s] + 1-\beta_k [r]), \tau_2^k = (\epsilon_k [N] + 1 - \epsilon_k [H], \epsilon_k [h] +  1-\epsilon_k [n]), $$
where 
$$\alpha_k =\frac{1}{k}, \, \beta_k = \frac{0.5}{k}, \, \epsilon_k = \frac{1}{k},$$
forms a sequential equilibrium.

\begin{enumerate}[resume]
\item The equilibrium $([Rr], [Nh])$.
\end{enumerate}

We can show that the above corresponds to a sequential equilibria, with a belief vector that reflects that for player 2, in the unlikely scenario that the candidate did study, it is more likely that the candidate has a low profile. 
\end{example}
\ifx \globalmark \undefined %% This is default.
\bibliographystyle{plain}
\bibliography{../gametheorybibliography}
	\end{document}
\else 
	
\fi
