As a reminder, we have two players, player 1 and player 2, who each have a valuation for an item that is being sold. We know that $v_1$ and $v_2$ are independent and identically distributed from a uniform distribution with parameters $0$ and $1$. We use $b_1$ and $b_2$ to denote the bid of player 1 and 2, respectively. Also, we use $u_1$ and $u_2$ to denote the utility of player 1 and 2, respectively. We have three cases
\begin{enumerate}
    \item If $b_1 < b_2$, then $u_1 = 0$ and $u_2 = v_2 - b_1$;
    \item If $b_1 > b_2$, then $u_1 = v_1 - b_2$ and $u_2 = 0$;
    \item If $b_1 = b_2$, then $u_1 = \frac{1}{2} \cdot \parent{v_1 - b_1}$ and $u_2 = \frac{1}{2} \cdot \parent{v_2 - b_2}$.
\end{enumerate}

However, we know that the last case will happen with a probability zero.
Notice the difference in the utility functions between the proof of Theorem 17 and this proof. This difference is due to the fact that Theorem 17 is looking at a sealed-bid \textit{first-price} auction, whereas Theorem 18 is looking at a sealed-bid \textit{second-price} auction. 

\vspace{5mm}


First, let us show that $b_1 \parent{v_1} = v_1$ is a best response for player 1 to the strategy of player 2. To be more precise, we need to check that for any fixed type $v_1$ for player 1, $b_1 \parent{v_1} = v_1$ is a best response to the strategy of player 2, knowing that the strategy of player 2 will be $b_2 \parent{v_2} = v_2$.
We need to compute
\begin{equation*}
    b_1^{*}
    = \text{argmax} \ 
    \parent{v_1 - b_2} \cdot \mathbb{P} \squared{b_1 > B_2}
    + \dfrac{1}{2} \parent{v_1 - b_2} \cdot \mathbb{P} \squared{b_1 = B_2}
\end{equation*}

where $B_2$ is written with a capital letter, because it denotes a random variable. Indeed, in this part of the proof, we fix $b_2$ such that $b_2 \parent{v_2} = v_2$. Since we know that $V_2 \sim \text{Unif} \parent{0, 1}$, we know that $B_2 \sim \text{Unif} \parent{0, 1}$. Now let us compute the two probabilities. We have
\begin{equation*}
    \mathbb{P} \squared{b_1 = B_2}
    = 0,
\end{equation*}

because $B_2$ is a continuous random variable, and
\begin{equation*}
    \mathbb{P} \squared{b_1 > B_2}
    = \mathbb{P} \squared{B_2 \leq b_1}
    = F_{B_2} \parent{b_1}
    = \dfrac{b_1 - 0}{1 - 0}
    = b_1.
\end{equation*}


where $F_{B_2}$ denotes the cumulative distribution function of $B_{2}$. Again, to be more precise, we should write

\begin{equation*}
    \mathbb{P} \squared{b_1 > B_2} = \max \bracket{0, \min \bracket{1, b_1}}
\end{equation*}

Let us use $\tilde{b}_1$ to shorten the notations, with $\tilde{b}_1 = \max \bracket{0, \min \bracket{1, b_1}}$. Hence we have
\begin{equation*}
    b_1^{*}
    = \text{argmax} \ \parent{v_1 - b_2} \cdot \tilde{b}_1.
\end{equation*}

Knowing that $B_2$ is a random variable, it should be apparent that the above expression is incorrect: we should write $\parent{v_1 - B_2}$ instead of $\parent{v_1 - b_2}$. Hence we have
\begin{equation*}
    b_1^{*}
    = \text{argmax} \ \parent{v_1 - B_2} \cdot \tilde{b}_1.
\end{equation*}

Now, a new problem arises. Since $B_2$ is a random variable, it should not appear "in the wild". It should be contained in a certain aggregation like an expected value $\mathbb{E} \parent{\cdot}$, or a probability $\mathbb{P} \parent{\cdot}$. Hence we have
\begin{equation*}
    b_1^{*}
    = \text{argmax} \ \mathbb{E} \squared{v_1 - B_2} \cdot \tilde{b}_1.
\end{equation*}

We can compute
\begin{equation*}
    \mathbb{E} \squared{v_1 - B_2}
    = \int_{- \infty}^{\infty} \parent{v_1 - b_2} f_{B_2} \parent{b_2} \mathrm{d} b_2
\end{equation*}

where $f_{B_2} \parent{b_2}$ is the probability density function of $B_2$. We are in the second case of the enumeration made at the beginning of the proof. Hence $B_2$ will not go from $- \infty$ to $+ \infty$. Instead, $B_2$ will go from $0$ to $\tilde{b}_1$. Therefore, we have $B_2 \sim \text{Unif} \parent{0, \tilde{b}_1}$, we have
\begin{equation*}
    f_{B_2} \parent{b_2} = \dfrac{1}{\tilde{b}_1 - 0} = \dfrac{1}{\tilde{b}_1}.
\end{equation*}

As a reminder, if $X \sim \text{Unif} \parent{a, b}$, then
\begin{equation*}
f_{X}\parent{x} = 
    \begin{cases}
        \dfrac{1}{b-a}  & {\text{for }} x \in [a,b], \\
        0               & {\text{otherwise}}.
    \end{cases}
\end{equation*}


Writing adequate integration bounds, and replacing $f_{B_2} \parent{b_2}$, we find
\begin{equation*}
    \mathbb{E} \squared{v_1 - B_2}
    = \int_{0}^{\tilde{b}_1} \parent{v_1 - b_2} \cdot \dfrac{1}{\tilde{b}_1} \mathrm{d} b_2.
\end{equation*}

Putting everything together, and keeping in mind that $\tilde{b}_1$ does not depend on $b_2$, we obtain

\begin{equation*}
    b_1^{*}
    = \text{argmax} \ \dfrac{\tilde{b}_1}{\tilde{b}_1} \cdot \int_{0}^{\tilde{b}_1} \parent{v_1 - b_2} \mathrm{d} b_2
    = \text{argmax} \ g \parent{b_1}
\end{equation*}

where the function $g$ is defined like
\begin{equation*}
    g \parent{b_1}
    =
    \begin{cases}
       \int_{0}^{b_1} \parent{v_1 - b_2} \mathrm{d} b_2     & 0 \leq b_1 \leq 1, \\
       \int_{0}^{0} \parent{v_1 - b_2} \mathrm{d} b_2       & b_1 < 0, \\
       \int_{0}^{1} \parent{v_1 - b_2} \mathrm{d} b_2       & b_1 > 1. \\ 
     \end{cases}
\end{equation*}

Computing the integrals, we find $\int_{0}^{b_1} \parent{v_1 - b_2} \mathrm{d} b_2 = b_1 \cdot \parent{v_1 - \dfrac{b_1}{2}}$, hence we obtain
\begin{equation*}
    g \parent{b_1}
    =
    \begin{cases}
       b_{1} \cdot \parent{v_1 - \dfrac{b_1}{2}}    & 0 \leq b_1 \leq 1, \\
       0                                            & b_1 < 0, \\
       v_1 - \dfrac{1}{2}                           & b_1 > 1. \\ 
     \end{cases}
\end{equation*}

We can compute the differential of $g$ with respect to $b_1$. We find
\begin{equation*}
    g' \parent{b_1}
    =
    \begin{cases}
       v_1 - b_1  & 0 \leq b_1 \leq 1, \\
       0 & b_1 < 0, \\
       0 & b_1 > 1. \\ 
     \end{cases}
\end{equation*}

Hence we find the value of $b_1^{*}$ by cancelling the derivative. We find
\begin{equation*}
    g' \parent{b_1^{*}} = 0
    \Leftrightarrow 
    \begin{cases}
       v_1 = b_1^{*}  & 0 \leq b_1 \leq 1, \\
       0 & b_1 < 0, \\
       0 & b_1 > 1. \\
     \end{cases}
\end{equation*}

We know that $b_1 < 0$ and $b_1 > 1$ are absurd. Hence we only consider the case $0 \leq b_1 \leq 1$, and we conclude that $b_1^{*} = b_1 \parent{v_1} = v_1$. 

Finally, it is important to do a continuity check. 


\vspace{5mm}

Now let us show that $b_2 \parent{v_2} = v_2$ is a best response for player 2 to the strategy of player 1. For this, we simply use the symmetry of the problem. This concludes the proof.