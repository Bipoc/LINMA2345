\ifx \globalmark \undefined %% This is default.
	\documentclass[a4paper,12pt]{report}

%%% PACKAGES %%%

\usepackage[french, english] {babel}	% langue principale
\usepackage[ansinew]{inputenc}
\usepackage[T1]{fontenc}		% Police contenant les caractères français
\usepackage{lmodern}			% plus beau
\usepackage[a4paper]{geometry}
	\geometry{hscale=0.75,vscale=0.8,centering}

\usepackage[hidelinks]{hyperref} % pas de couleurs ici
%
\usepackage[natbibapa]{apacite}

%% Maths
\usepackage{amsfonts} % Equations etc
\usepackage{amsmath}
\usepackage{mathtools}
\usepackage{amssymb}

\usepackage{cleveref}

%% Itemizes
\usepackage{enumerate}
\usepackage{enumitem}

%% Dessins & Plots
\usepackage[pdftex]{graphicx} %Images dans le PDF
\usepackage{epstopdf}
\usepackage{color, xcolor}
%% Dates
\usepackage{datetime}
\newdateformat{monthyeardate}{\monthname[\THEMONTH] \THEYEAR}
\usepackage{chessboard}
\usepackage{booktabs,tabularx,multirow}
\usepackage[babel=true,kerning=true]{microtype}
%> Tikz
\usepackage{tikz}
\usepackage{hf-tikz}
\usetikzlibrary{
	calc,
	arrows,
	arrows.meta,
	automata,
	shapes,
	snakes,
	positioning,
	decorations,
	decorations.text,
	fit,
	matrix,
	mindmap
	}
	\tikzstyle{noeud-std}=[draw,fill=black,circle,inner sep=0pt,minimum size=7pt]% 7pt est la taille des cercles noirs
\usepackage{tkz-graph}
\newcommand{\tikzmark}[2]{\tikz[overlay,remember picture,baseline=(#1.base)] \node (#1) {#2};}
\newcommand{\Highlight}[1][submatrix]{%
    \tikz[overlay,remember picture]{
    \node[highlight,fit=(left.north west) (right.south east)] (#1) {};}
    }
\tikzset{%
  highlight/.style={rectangle,rounded corners,fill=ocre!50,draw,
    fill opacity=0.5,thick,inner sep=0pt}
}
\newcommand{\mytikzmark}[2]{\tikz[overlay,remember picture, baseline=(#1.base)] \node (#1) {#2};}


%% Meta
\usepackage{etoolbox}

% Debugging purposes: Catch when a reference is missing
\makeatletter
	\patchcmd{\@setref}{\bfseries ??}{{\color{red}[\texttt{\detokenize{ #3 }}]}}{}{%
	  \GenericWarning{}{Failed to patch \protect\@setref}}
	\patchcmd{\@citex}{\bfseries ?}{{\color{red}[\texttt{\detokenize{ #3 }}]}}{}{%{{\color{red}\texttt{\@citeb}}}{}{%
	  \GenericWarning{}{Failed to patch \protect\@citex}}
\makeatother

%% CUSTOM ENVIRONMENTS
% Packages
\usepackage{framed}
\usepackage{mdframed}
\usepackage[amsmath,thref, framed]{ntheorem}

%% Counters for theorem.
% The current choice is that all environments share a counter within chapters.
% It that regards, it becomes easy to navigate the document.

\newcounter{theo}[chapter]\setcounter{theo}{0}
\renewcommand{\thetheo}{\arabic{chapter}.\arabic{theo}}



% The following is adapted from https://texblog.org/2015/09/30/fancy-boxes-for-theorem-lemma-and-proof-with-mdframed/
% \newenvironment{name}[args]{begin_def}{end_def}
\newenvironment{generic_theo}[2][] % 2 arguments, and one is optional. The first argument will say if we have a Thm or somth, the second is the name of the thm.
	{% begin_def
		\refstepcounter{theo}%
		\ifstrempty{#2} % looking at the name of the thm
		{
			\mdfsetup{
				frametitle={
					\tikz[baseline=(current bounding box.east),outer sep=0pt]
					\node[anchor=east,rectangle,fill=white!20, draw = black!20, line width = 2pt]
					{\strut #1~\thetheo};}
			} % The #1 will say Theorem, or other.
		}
		{
			\mdfsetup{
				frametitle={
				\tikz[baseline=(current bounding box.east),outer sep=0pt]
				\node[anchor=east,rectangle,fill=white!20, draw = black!20, line width = 2pt]
				{\strut #1~\thetheo:~#2};}
			}
		}
		\mdfsetup{innertopmargin=10pt,linecolor=black!20,
					linewidth=2pt,topline=true,
					frametitleaboveskip=\dimexpr-\ht\strutbox\relax
				}
\begin{mdframed}[]\relax}{\end{mdframed}}


%% Proofs (slightly different)
\newenvironment{generic_proof}[2][] % 2 arguments, and one is optional. The first argument will say if we have a Thm or somth, the second is the name of the thm.
	{% begin_def
		\refstepcounter{theo}%
		\ifstrempty{#2} % looking at the name of the thm
		{
			\mdfsetup{
				frametitle={
					\tikz[baseline=(current bounding box.east),outer sep=0pt]
					\node[anchor=east,rectangle,fill=black!20]
					{\strut #1~\thetheo};}} % The #1 will say Theorem, or other.
		}
		{
			\mdfsetup{
				frametitle={
				\tikz[baseline=(current bounding box.east),outer sep=0pt]
				\node[anchor=east,rectangle,fill=black!20]
				{\strut #1~\thetheo:~#2};}}
		}
		\mdfsetup{innertopmargin=10pt,linecolor=black!20,
					linewidth=1pt,topline=true,
					frametitleaboveskip=\dimexpr-\ht\strutbox\relax
				}
\begin{mdframed}[]
}{\end{mdframed}}

%% Examples/Exercices
\newenvironment{generic_ex}[2][] % 2 arguments, and one is optional. The first argument will say if we have a Thm or somth, the second is the name of the thm.
	{% begin_def
		\refstepcounter{theo}%
		\ifstrempty{#2} % looking at the name of the thm
		{
			\mdfsetup{
				frametitle={
					\tikz[baseline=(current bounding box.east),outer sep=0pt]
					\node[anchor=east,rectangle,fill=black!20]
					{\strut #1~\thetheo};}} % The #1 will say Theorem, or other.
		}
		{
			\mdfsetup{
				frametitle={
				\tikz[baseline=(current bounding box.east),outer sep=0pt]
				\node[anchor=east,rectangle,fill=black!20]
				{\strut #1~\thetheo:~#2};}}
		}
		\mdfsetup{innertopmargin=10pt,linecolor=black!20,
					linewidth=1pt,topline=true,
					frametitleaboveskip=\dimexpr-\ht\strutbox\relax
				}
\begin{mdframed}[]
}{\end{mdframed}}



% The different environments that we need (Main stuff are highlighted.
\newenvironment{theorem}[1][]{\begin{generic_theo}[Theorem]{#1}}{\end{generic_theo}}
\newenvironment{lemma}[1][]{\begin{generic_theo}[Lemma]{#1}}{\end{generic_theo}}
\newenvironment{definition}[1][]{\begin{generic_theo}[Definition]{#1}}{\end{generic_theo}}
\newenvironment{notation}[1][]{\begin{generic_theo}[Notation]{#1}}{\end{generic_theo}}
\newenvironment{proposition}[1][]{\begin{generic_theo}[Proposition]{#1}}{\end{generic_theo}}
\newenvironment{procedure}[1][]{\begin{generic_theo}[Procedure]{#1}}{\end{generic_theo}}
\newenvironment{hypothese}[1][]{\begin{generic_theo}[Hypothesis]{#1}}{\end{generic_theo}}


% These ones, let's not overblow them
% The true reason why I'm doing this is that there may be floats in Exercices and Examples...
% You can't have  begin{figures} in mdframed env, it crashes.
%\newtheorem{notation}[theo]{Notation}
\newcounter{axiomc}[chapter]\setcounter{axiomc}{0}
\renewcommand{\theaxiomc}{\arabic{chapter}.\arabic{axiomc}}

\newtheorem{axiom}[axiomc]{Axiom}
\newtheorem{proof}[theo]{Proof}
\newtheorem{exercise}[theo]{Exercise}
\newtheorem{example}[theo]{Example}



%% CUSTOM Commands

% Identify TAs
\newcommand{\TAone}{Mahsa}
\newcommand{\TAtwo}{Beno\^it}

\newcommand{\reels}{\mathbb{R}}

\DeclareMathOperator*{\argmin}{arg\,min}
\DeclareMathOperator*{\argmax}{arg\,max}

% Nice brackets
\newcommand\parent[1]{\left(#1\right)}
\newcommand\abs[1]{\left\lvert#1\right\rvert}
\newcommand\norm[1]{\left\lVert#1\right\rVert}
\newcommand\bracket[1]{\left\{#1\right\}}
\newcommand\squared[1]{\left[#1\right]}

\DeclarePairedDelimiter{\floor}{\lfloor}{\rfloor}
\DeclarePairedDelimiter{\ceil}{\lceil}{\rceil}
\usepackage{eurosym}
\usepackage{siunitx}
\DeclareSIUnit{\EUR}{\text{\euro}}
\sisetup{
  per-mode = fraction,
  inter-unit-product = \ensuremath{{}\cdot{}},
}



\usepackage{textcomp}
\let\texteuro\euro





	\begin{document} %% Crashes if put after (one of the many mysteries of LaTeX?).
\else
\fi



\chapter{Repeated Games}
{\large{\itshape
``Insanity: Doing the same thing over and over again and expecting different results.''} --- Albert Einstein.\\
}
\label{chap:Rep}
{\small{\itshape
Chapter based on \cite[pages 308 - 331]{MyGTAO}, \cite[Section 6.1]{ShLeMSAG}.} and on
the Stanford/Coursera MOOC on Game Theory (for the section on learning in repeated games).\\
}

A repeated game is a strategic situation where a set of players play a same
game several times in a row. These situations occur quite often in real life,
and in fact, most of our everyday interactions can be seen as part of a
repeated game.\\
Interestingly, the behavior of a player in a repeated game may be different
than that of the same player when playing the game only once. For example,
while we know that the only Nash Equilibrium of the Prisoner's Dilemma game
is for both players to betray the other, cooperation may occur rationally
in the repeated game setting.

The chapter is split in three sections, presenting matters from the more
particular to the more general. Section~\ref{sec:FinRep} discusses finitely
repeated game, that can be studied directly with the analysis tools of
previous chapters. Section~\ref{sec:InfRep} discusses infinitely repeated
games. There, we start observing interesting phenomena, in particular
concerning the Prisoner's Dilemma. Finally, in Section~\ref{sec:GenMod}, we
present a more general and involved model for repeated games, and discuss
the notions of equilibria in more details.

\section{Finitely repeated games}
\label{sec:FinRep}
In a \emph{finitely repeated game}, players play the \emph{same game
$\Gamma$} repeatedly for a finite number of times $K$. The number $K$ is
known to the players at the beginning of the game. This situation is
represented in Figure~\ref{fig:repeatedFiniteNumberOfTimes}.

\begin{figure}[!ht]
    \centering
    \begin{tikzpicture}[->,>=stealth',shorten >=1pt,auto,node distance=2.5cm, semithick, scale = 1, transform shape ]
        \node	(start)								{Start};
        \node[state] 	(n1)    	[right of = start]		{Round 1};
        \node[state] 	(n2)		[right of = n1]			{Round 2};
        \node	(dots)		[right of = n2]			{$\cdots$};
        \node[state] 	(nk)		[right of = dots]		{Round K};
        \node	(stop)		[right of = nk]			{Stop};
        \path 	(start) edge node {}	(n1)
                (n1)	edge node {}	(n2)
                (n2)	edge node {}	(dots)
                (dots)	edge node {}	(nk)
                (nk) 	edge node {}	(stop);
    \end{tikzpicture}
    \caption{Finitely Repeated Games.}
    \label{fig:repeatedFiniteNumberOfTimes}
\end{figure}
After each round, the players collect their payoffs and then move on to
the next.

A finitely repeated game can be written as a classical game, where players
have to decide before the first round on a strategy to play at each round.
The next example illustrates this on the Prisoner's Dilemma.
\begin{example}[Twice Repeated Prisoner's Dilemma]
    \label{ch9:ex:PD1}
    Consider the Prisoner's Dilemma game $\Gamma$ below
    \begin{center}
    \begin{tabular}{c|cc}
        Prisoner's Dilemma & c & d \\
        \hline
        C & $1, 1$ & $-1, 2$ \\
        D & $2, -1$ & $0, 0$
    \end{tabular}
\end{center}
The only Nash Equilibrium is the situation where both players defect
$(D,d)$. However, it has been observed that, in many social interactions,
players have a natural tendency to cooperate.
Furthermore, it makes sense, intuitively, that if one is going to interact
with another on several occasions, it would be rational to show ourselves
as agreeable and cooperative in order to foster a prosperous relation.

Let us see what happens then when the game is repeated twice. The players
are now faced with the situation represented in Figure~\ref{fig:2PDtree}.
\begin{figure}[!ht]
    \centering
    \begin{tikzpicture}
    \node[noeud-std] (n1) {}
       [sibling distance=8.5cm]
          child {node[noeud-std] (n2-1c) {} % 1
          [sibling distance=4cm]
                 child{node[noeud-std] (n1-1c2c){} % 2
                    [sibling distance=2cm]
                    child{node[noeud-std] (n2-1c2c1c){}  % 1
                        [sibling distance = 1cm]
                             child[level distance=1.5cm]{node[noeud-std,fill=white] (p-1c2c1c2c){} } % 1
                             child[level distance=2.5cm]{node[noeud-std,fill=white] (p-1c2c1c2d){} } % 1
                         }
                    child{node[noeud-std] (n2-1c2c1d){}  % 1
                        [sibling distance = 1cm]
                            child[level distance=1.5cm]{node[noeud-std,fill=white] (p-1c2c1d2c){} } % 1
                            child[level distance=2.5cm]{node[noeud-std,fill=white] (p-1c2c1d2d){} } % 1
                        }
                }
                child{node[noeud-std] (n1-1c2d){} % 2
                    [sibling distance=2cm]
                    child{node[noeud-std] (n2-1c2d1c){}  % 1
                        [sibling distance = 1cm]
                             child[level distance=1.5cm]{node[noeud-std,fill=white] (p-1c2d1c2c){} } % 1
                             child[level distance=2.5cm]{node[noeud-std,fill=white] (p-1c2d1c2d){} } % 1
                         }
                    child{node[noeud-std] (n2-1c2d1d){}  % 1
                        [sibling distance = 1cm]
                            child[level distance=1.5cm]{node[noeud-std,fill=white] (p-1c2d1d2c){} } % 1
                            child[level distance=2.5cm]{node[noeud-std,fill=white] (p-1c2d1d2d){} } % 1
                        }
                }
       }
       child {node[noeud-std] (n2-1d) {} % 1
          [sibling distance=4cm]
                 child{node[noeud-std] (n1-1d2c){} % 2
                    [sibling distance=2cm]
                    child{node[noeud-std] (n2-1d2c1c){}  % 1
                        [sibling distance = 1cm]
                             child[level distance=1.5cm]{node[noeud-std,fill=white] (p-1d2c1c2c){} } % 1
                             child[level distance=2.5cm]{node[noeud-std,fill=white] (p-1d2c1c2d){} } % 1
                         }
                    child{node[noeud-std] (n2-1d2c1d){}  % 1
                        [sibling distance = 1cm]
                            child[level distance=1.5cm]{node[noeud-std,fill=white] (p-1d2c1d2c){} } % 1
                            child[level distance=2.5cm]{node[noeud-std,fill=white] (p-1d2c1d2d){} } % 1
                        }
                }
                child{node[noeud-std] (n1-1d2d){} % 2
                    [sibling distance=2cm]
                    child{node[noeud-std] (n2-1d2d1c){}  % 1
                        [sibling distance = 1cm]
                             child[level distance=1.5cm]{node[noeud-std,fill=white] (p-1d2d1c2c){} } % 1
                             child[level distance=2.5cm]{node[noeud-std,fill=white] (p-1d2d1c2d){} } % 1
                         }
                    child{node[noeud-std] (n2-1d2d1d){}  % 1
                        [sibling distance = 1cm]
                            child[level distance=1.5cm]{node[noeud-std,fill=white] (p-1d2d1d2c){} } % 1
                            child[level distance=2.5cm]{node[noeud-std,fill=white] (p-1d2d1d2d){} } % 1
                        }
                }
       }

    ;
    %
    \node[above=5pt] at (n1) {$1.a$};
    \node[above left] at ($(n1)!{0.5}!(n2-1c)$) {C};
    \node[above right] at ($(n1)!{0.5}!(n2-1d)$) {D};

    \node[above=5pt] at (n2-1c) {$2.b$};
    \node[above left] at ($(n2-1c)!{0.5}!(n1-1c2c)$) {c};
    \node[above right] at ($(n2-1c)!{0.5}!(n1-1c2d)$) {d};
    \node[above=5pt] at (n1-1c2c) {$1.c$};
    \node[above left] at ($(n1-1c2c)!{0.5}!(n2-1c2c1c)$) {C};
    \node[above right] at ($(n1-1c2c)!{0.5}!(n2-1c2c1d)$) {D};
    \node[above=5pt] at (n2-1c2c1d) {$2.d$};
    \node[above left] at ($(n2-1c2c1d)!{0.5}!(p-1c2c1d2c)$) {c};
    \node[above right] at ($(n2-1c2c1d)!{0.5}!(p-1c2c1d2d)$) {d};
    \node[above=5pt] at (n2-1c2c1c) {$2.d$};
    \node[above left] at ($(n2-1c2c1c)!{0.5}!(p-1c2c1c2c)$) {c};
    \node[above right] at ($(n2-1c2c1c)!{0.5}!(p-1c2c1c2d)$) {d};
    \node[above=5pt] at (n1-1c2d) {$1.e$};
    \node[above left] at ($(n1-1c2d)!{0.5}!(n2-1c2d1c)$) {C};
    \node[above right] at ($(n1-1c2d)!{0.5}!(n2-1c2d1d)$) {D};
    \node[above=5pt] at (n2-1c2d1d) {$2.f$};
    \node[above left] at ($(n2-1c2d1d)!{0.5}!(p-1c2d1d2c)$) {c};
    \node[above right] at ($(n2-1c2d1d)!{0.5}!(p-1c2d1d2d)$) {d};
    \node[above=5pt] at (n2-1c2d1c) {$2.f$};
    \node[above left] at ($(n2-1c2d1c)!{0.5}!(p-1c2d1c2c)$) {c};
    \node[above right] at ($(n2-1c2d1c)!{0.5}!(p-1c2d1c2d)$) {d};

    \node[above=5pt] at (n2-1d) {$2.b$};
    \node[above left] at ($(n2-1d)!{0.5}!(n1-1d2c)$) {c};
    \node[above right] at ($(n2-1d)!{0.5}!(n1-1d2d)$) {d};
    \node[above=5pt] at (n1-1d2c) {$1.g$};
    \node[above left] at ($(n1-1d2c)!{0.5}!(n2-1d2c1c)$) {C};
    \node[above right] at ($(n1-1d2c)!{0.5}!(n2-1d2c1d)$) {D};
    \node[above=5pt] at (n2-1d2c1d) {$2.h$};
    \node[above left] at ($(n2-1d2c1d)!{0.5}!(p-1d2c1d2c)$) {c};
    \node[above right] at ($(n2-1d2c1d)!{0.5}!(p-1d2c1d2d)$) {d};
    \node[above=5pt] at (n2-1d2c1c) {$2.h$};
    \node[above left] at ($(n2-1d2c1c)!{0.5}!(p-1d2c1c2c)$) {c};
    \node[above right] at ($(n2-1d2c1c)!{0.5}!(p-1d2c1c2d)$) {d};
    \node[above=5pt] at (n1-1d2d) {$1.i$};
    \node[above left] at ($(n1-1d2d)!{0.5}!(n2-1d2d1c)$) {C};
    \node[above right] at ($(n1-1d2d)!{0.5}!(n2-1d2d1d)$) {D};
    \node[above=5pt] at (n2-1d2d1d) {$2.j$};
    \node[above left] at ($(n2-1d2d1d)!{0.5}!(p-1d2d1d2c)$) {c};
    \node[above right] at ($(n2-1d2d1d)!{0.5}!(p-1d2d1d2d)$) {d};
    \node[above=5pt] at (n2-1d2d1c) {$2.j$};
    \node[above left] at ($(n2-1d2d1c)!{0.5}!(p-1d2d1c2c)$) {c};
    \node[above right] at ($(n2-1d2d1c)!{0.5}!(p-1d2d1c2d)$) {d};

    \path (n2-1d)  edge [dashed] node {} (n2-1c);
    \path (n2-1c2c1d)  edge [dashed] node {} (n2-1c2c1c);
    \path (n2-1c2d1d)  edge [dashed] node {} (n2-1c2d1c);
    \path (n2-1d2c1d)  edge [dashed] node {} (n2-1d2c1c);
    \path (n2-1d2d1d)  edge [dashed] node {} (n2-1d2d1c);

    \node[below = 5pt] at ($(p-1c2c1c2c)$) {2,2};
    \node[below = 5pt] at ($(p-1c2c1c2d)$) {0,3};
    \node[below = 5pt] at ($(p-1c2c1d2c)$) {3,0};
    \node[below = 5pt] at ($(p-1c2c1d2d)$) {1,1};


    \node[below = 5pt] at ($(p-1c2d1c2c)$) {0,3};
    \node[below = 5pt] at ($(p-1c2d1c2d)$) {-2,4};
    \node[below = 5pt] at ($(p-1c2d1d2c)$) {1,1};
    \node[below = 5pt] at ($(p-1c2d1d2d)$) {$-1$,2};


    \node[below = 5pt] at ($(p-1d2c1c2c)$) {3,0};
    \node[below = 5pt] at ($(p-1d2c1c2d)$) {0,3};
    \node[below = 5pt] at ($(p-1d2c1d2c)$) {4,-2};
    \node[below = 5pt] at ($(p-1d2c1d2d)$) {2,$-1$};


    \node[below = 5pt] at ($(p-1d2d1c2c)$) {1,1};
    \node[below = 5pt] at ($(p-1d2d1c2d)$) {$-1$,2};
    \node[below = 5pt] at ($(p-1d2d1d2c)$) {2,$-1$};
    \node[below = 5pt] at ($(p-1d2d1d2d)$) {0,0};

    \end{tikzpicture}
    \caption{The twice repeated Prisoner's Dilemma, in extensive form.
    The payoffs at the terminal nodes are the sum of those at both
    rounds.}
    \label{fig:2PDtree}
\end{figure}
The key to analyze repeated games is the \emph{backward analysis.} For this, we begin by studying how the players should act at the
second round. Clearly, they only want to maximize their contributions to
their own payoffs. Hence, in the second round, they should select
$([D],[d])$ (the Nash Equilibrium).
Knowing this, we can infer (by recurrence) that the players should pick
$([D],[d])$ as well in the first round.

In conclusion, when playing a finitely repeated version of the Prisoner's
Dilemma, players should always defect at every round.
We did the exercise for when the game is repeated twice, but this holds
for any fixed number $K$ of repetitions.
\end{example}

The study of repeated games calls for the definition of an appropriate
notion of \emph{strategy}. The following definition, though informal, has the advantage to represent our intuition and it is valid for both the finitely and infinitely repeated case.

\begin{definition}[Strategy in a repeated game (informal)]
In a repeated game with $N$ players, a strategy for player $i$ defines,
at each round $k = 1, 2, \ldots$ a (randomized) strategy to play for the
game at round $k$ given the information available at the beginning of
round $k$.
\end{definition}

Let us linger a bit longer on what constitutes \emph{information} for a
player in round $k$. A player should at least be aware of
\emph{the history of his moves and payoffs} so far.
The players may also know other players' moves or their payoffs depending
on the scenario.

\begin{definition}[Stationary Strategy (informal)]
In a repeated game, a \emph{stationary strategy} is a strategy that
depends only on the current round.
\end{definition}

\begin{proposition}[Equilibrium in Finitely Repeated Prisoner's Dilemma]
In a finitely repeated Prisoner's Dilemma, the unique equilibrium
strategy is stationary, where each player defect at every round.
\end{proposition}
This can be easily proven by using the \emph{backward analysis}, see Ch.\ \ref{chap:Seq}, Sect.\ \ref{ch4:sec:motivations}.\\
Remarkably, in \emph{infinitely repeated games}, the players will have
access to a richer array of actions and payoffs. They will not be
confined to playing Nash Equilibria at each round, but they will now be
able to learn and adapt their moves in accordance to the history of the
game. In particular, we can observe the rise of cooperation in the
Prisoner's Dilemma in the infinitely repeated setting.

\section{Infinitely Repeated Games}
\label{sec:InfRep}

In an infinitely repeated game, all the players involved know that they
will have to play the game an infinite amount of time.
At each round, the players gather their payoffs for the current round before
playing the next one (see Figure~\ref{fig:infiRepeated}).
\begin{figure}
    \centering
    \begin{tikzpicture}[->,>=stealth',shorten >=1pt,auto,node distance=2.5cm, semithick, scale = 1, transform shape ]
        \node	(start)								{Start};
        \node[state] 	(nk)    	[right of = start]		{Round k};
        \path 	(start) edge node {}	(nk)
                (nk)	edge [loop left, out = -30, in = 30, distance = 2cm] node[right] {$k := k +1$}	(nk);
    \end{tikzpicture}
    \caption{Infinitely Repeated Games.}
    \label{fig:infiRepeated}
\end{figure}

\begin{notation}[Time superscript]
Given a sequence $r$ in time, we use the superscript $r^{[t]}$ to denote
the element of the sequence $r$ at time $t$.
\end{notation}

\begin{definition}[Standard Repeated Game]
A \emph{standard repeated game} is an infinitely repeated game where a
same game in strategic form $\Gamma$ is repeated infinitely and where at each round $k$,
every player knows all the players past moves.
\end{definition}

When playing a repeated game, players needs to select some strategy to be
played instead of others. To do so, they need to be able to express
preferences, as prescribed by Decision Theory.
The following definitions provide two classical criteria for ranking
payoffs in repeated games.

\begin{definition}[Average and Discounted Payoffs]
Given a sequence of payoffs $r_i^{[1]}, r_i^{[2]}, \ldots$ for player $i$
in an infinitely repeated game,
\begin{itemize}
    \item the \emph{limit average payoff} of $i$ is
    \begin{equation}
        \lim_{k \rightarrow \infty} \frac{\sum_{j = 1}^k r_i^{[j]}}{k}.
        \label{eq:averagePayoffRep}
    \end{equation}
    Note that this limit may not exist (but more sophisticated
    limit definitions can be used to circumvent this),
    \item for a \emph{discount factor} $0 \leq \delta < 1$, the
    \emph{$\delta-$discounted payoff} is
    \begin{equation}
        \sum_{j = 1}^{\infty} \delta^{j-1} r_i^{[j]}.
        \label{eq:discountedPayoffRep}
    \end{equation}
    \item for a \emph{discount factor} $0 \leq \delta < 1$, the
    \emph{$\delta-$discounted average payoff} is
    \begin{equation}
        (1-\delta) \sum_{j = 1}^{\infty} \delta^{j-1} r_i^{[j]}.
        \label{eq:discountedAvgPayoffRep}
    \end{equation}
\end{itemize}
\end{definition}
Other criteria exist, see~\cite[page 315]{MyGTAO}.

The \emph{discounted payoff} can be interpreted in a number of ways.\\
The \emph{discount factor} $\delta$ is commonly understood as a measure
of the player's \emph{patience}: the closer to $0$, the more the player
aims at obtaining immediate payoffs, and the closer to $1$, the more the
player is willing to wait for future payoffs.\\
There is another way to interpret the discount factor, which is useful to model situations where the game will actually not be played an infinite number of time, but will end at some moment, \emph{in a probabilistic way}. Indeed, suppose that at every round, the game may terminate with probability $1-\delta$, or go on for at least one more round with probability $\delta$ (see Figure~\ref{fig:infiRepeatedDiscount}). The reader can observe that the equations for the expected payoff are exactly the same than in the discounted setting.  Indeed, if the game may stop at every round with probability $1-\delta$, in order to obtain the payoff $r_i^{(j)}$ the game must be repeated $j-1$ times, which occurs with probability $\delta^{j-1}$. The discounted payoff is then the
expected payoff of the game under this interpretation.
Finally, as the reader has guessed, there is a third interpretation/reason to use the discounted payoff:
it represents situation where inflation occurs, that is, the actual value of a gain at some time $t$ is preferred to the same gain later in time.

The \emph{discounted average payoff} is a normalized version of the
discounted payoff, where if $r_i^{[t]} = 1$ for all $t$,
$$(1-\delta) \sum_{j = 1}^{\infty} \delta^{j-1} 1 = 1.$$

\begin{figure}
    \centering
    \begin{tikzpicture}[->,>=stealth',shorten >=1pt,auto,node distance=7cm, semithick, scale = 1, transform shape ]
        \node (start) {Start};
        \node [state] (nk) [right of = start] {Round $k$};
        \node (stop) [right of = nk] {Stop};
        \path (start) edge node {} (nk)
        (nk) edge [loop above] node {Probability $\delta$, $k := k +1$}
        (nk) (nk) edge[] node{Probability $1-\delta$} (stop);
    \end{tikzpicture}
    \caption{Interpretation of discounted payoff: the game ends with
    probability $1 - \delta$.}
    \label{fig:infiRepeatedDiscount}
\end{figure}

\begin{definition}[Equilibrium for Repeated Games (informal)]
An equilibrium of a repeated game is a set of strategies (one per player)
where each strategy is a best response to the others, in that it maximizes
the selected payoff criterion given the other players' strategies.
\end{definition}

\begin{example}
\label{ch9:ex:trigger}
In an infinitely repeated version of the Prisoner's Dilemma, cooperation
can appear as a rational strategy to be played at each round.
Consider a discount factor $\delta$ and the game of
Example~\ref{ch9:ex:PD1}.
We define the strategy of the player as follows: each player is supposed
to start the game by cooperating. Then, if at any round a player defects,
the other reacts by defecting for the rest of the game.
The $\delta$-discounted payoff for the strategy is
$$\sum_{j = 1}^{\infty}(\delta)^{j-1} 1 = \frac{1}{1-\delta}$$
where players cooperate at every round.
If a player defects at the first round, he gets a payoff of $2$ and then
always gets a payoff of $0$ because the other player reacts by always
defecting. His total payoff is then simply $2$, independently of $\delta$.
Consequently, if
$$\frac{1}{1-\delta} \geq 2 \Leftrightarrow \delta \geq 0.5,$$
the  strategy is an equilibrium, leading to a cooperative outcome for
the game.
\end{example}

In the example above, the strategy leading to the cooperative equilibrium
infinitely punishes a player for uncooperative behaviors. This is called
the \textbf{grim} (or \textbf{trigger}) strategy. Other classical
strategies, leading to cooperative equilibria, are as follows
\begin{itemize}
	\item \textbf{Grim}: the grim strategy is that of Example \ref{ch9:ex:trigger}. We begin the game with cooperation and continue cooperating at each round unless the other player defects. From this point on, we defect. See the analysis in \cite[pp. 308-310]{MyGTAO} for a detailed analysis of this example.
    \item \textbf{tit-for-tat}: we begin by cooperating and then, at every
    round, we choose the move of our opponent at the last round,
    \item \textbf{getting-even}: we begin by cooperating, and continue
    cooperating \emph{unless} the opponent has defected more time that
    we did. If this is the case, we defect.
\end{itemize}

\subsection{Folk Theorems and Cooperation}
Infinitely repeated games offer opportunities for cooperative behaviors,
which can be linked to the theory of cooperative game theory.
The following is taken from~\cite[Theorem 6.1.5]{ShLeMSAG}.
\begin{theorem}[Folk Theorem]
\label{ch9:thm:folk1}
Consider any $N$-player normal-form game $\Gamma = (N,C,u)$, and a
payoff profile $r = (r_i)_{i \in N}$ with $r_i \in \reels$.
Assume that
\begin{itemize}
    \item the profile $r$ satisfies $r_i \geq v_i$, where
    $$v_i = \min_{c_{-i} \in C_{-i}} \max_{c_i \in C_i}
    u_i(c_{-i}, c_i),$$
    \item and there is a correlated strategy $\tau \in \Delta C$, where
    $\tau(c) \in \mathbb{Q}$ for each $c \in C$ where $\mathbb{Q}$ is the set
    of rational numbers, such that
    $$r_i = \sum_{c \in C} \tau(c) u_i(c).$$
\end{itemize}
Then $r$ is the payoff obtained by the players for some Nash Equilibrium
of the infinitely repeated game with \emph{limit average rewards}.
\end{theorem}

There is an interesting parallel here with Chapter~\ref{chap:Cor} and in
particular Section \ref{ch5:sec:cont}.
As a reminder, the topic there was finding rules under which a group of
players is willing to sign a contract, forcing each of them to follow the
recommendations of a mediator. The mediator is a proxy to represent
communication processes between the players. The following example
illustrates the results.

\begin{example}[Example~\ref{ch5:ex:minimax} revisited.]
Alice and Bob wish to hold an old promise: once a month, for the rest of
their lives (which we assume to be infinite for the exercise), they will
meet either at the cinema or at the ballet.
Alice prefers to go to the ballet, and Bob to the cinema.
Their payoffs and moves are summarized in the table below (Alice is player
1, Bob is Player 2).

\begin{center}
    \begin{tabular}{c | c  c}
          & c & b \\
        \hline
        C & $2, 3$ & $-1, -1$ \\
        B & $1, 1$ & $3, 2$
    \end{tabular}.
\end{center}
We computed in Example~\ref{ch5:ex:minimax} the minimax values
$v = (7/5, 7/5)$ for this game, and saw in
Example~\ref{example:AliceBobCinemaBargain} that a very good correlated
strategy here is to play $\tau([C],[c]) = 0.5$, $\tau([B],[b]) = 0.5$,
with expected payoffs $(2.5,2.5)$.

For the sake of the exercise, let us assume that Alice and Bob don't plan
ahead during these activities: they just enjoy the moment, and have no
other means of communication.
They decide their moves based on the history of the game (here, it is easy
to find out where was the other if they do not meet).

It is not difficult to construct strategies for Alice and Bob that grant
them an \emph{average payoff} of $2.5$ in the infinitely repeated game.

Let us consider the sequence of outcomes $a^{(t)} = ([C],[c])$ if $t$ is
even, and $a^{(t)} = ([B],[b])$ if $t$ is odd.
Consider the following strategy for each player: begin by playing with
$a^{(1)}$ at round 1, and then at round $k$
\begin{itemize}
    \item if the outcomes at all previous rounds are those of the
    sequence $a$, then play according to $a^{(k)}$,
    \item else, play the minimax strategy for the other player.
\end{itemize}
Clearly, if both player follow this strategy, they get an average payoff
of $2.5$. Moreover, if one deviates from this strategy, his expected
payoff drops to $7/5$, because the other plays is minimax for eternity.
Hence, the strategy is a Nash Equilibrium.
\end{example}




\section{General Model of Repeated Games}
\label{sec:GenMod}


So far, we have focused on finitely repeated games and standard repeated games. We now discuss a more general model, with the associated computational tools for formally defining strategies and equilibria.


To represent a repeated game, we need a number of elements. First, of course, we need a set of players $N$ and a set of actions $D = \times_{i \in N} D_i$ that the players can play at each round.\\ Additionally, we may consider that in each round, the game is in a different \emph{state}. For example we can understand the situation Figure \ref{fig:infiRepeatedDiscount} as one infinitely repeated game with \emph{two} states: the game is either running or stopped, and the finitely repeated game of Figure \ref{fig:repeatedFiniteNumberOfTimes} as infinitely repeated games with $K+1$ states: K rounds and the stop state.
In all generality, we let $\Theta$ be the set of \emph{states of the world}. \\
The actions of players lead to rewards at the end of each round, depending on the state of nature, through a utility function $u_i : D \times \Theta \rightarrow \reels$ for each $i \in N$. \\
As mentioned before, players make decision based on currently available information, gathered through past rounds. This is something we represent through a set of \emph{signals} $S = \times_{i \in N} S_i.$
Signals are flexible in their nature: the set of signal $S_i$ for player $i$ could contain $D$, meaning he obtain at each round information about the moves of all players on the past round; or it may contain $\Theta$, meaning the player has information on the state of the world, etc... This flexibility can be used for modeling purpose.\\
Additionally, we need to define how the game \emph{starts}. At the beginning of the game, each player must receive a signal (which can be as simple as "the game as started"), and a state of world has to be selected.
This is represented by an \emph{initial distribution}
$ q \in \Delta(S \times \Theta).$
Finally, as the game evolves, players will be fed signals and the state of the world will change. In all generality, we represent this by a \emph{transition function}
$ p : D \times \Theta \rightarrow \Delta(S \times \Theta).$

\begin{definition}[General Repeated Game]
A general repeated game is a structure of the form
$$\Gamma^r = (N, \Theta, (D_i, S_i, u_i)_{i \in N}, q, p), $$
where
\begin{itemize}
\item $N$ is the set of \emph{players},
\item $\Theta$ is the set of \emph{state of the worlds},
\item $D_i$ is the set of \emph{moves} available to player $i$,
\item $S_i$ is the set of \emph{signals} that player $i$ may receive,
\item $q \in \Delta(S \times \Theta)$ is the \emph{initial distribution},
\item $p : D \times \Theta \rightarrow \Delta(S \times \Theta) $ is the \emph{transition function}.
\end{itemize}
\end{definition}

\begin{definition}[Strategy]
Consider a repeated game $\Gamma^r = (N, \Theta, (D_i, S_i, u_i)_{i \in N}, q, p).$
The set of all \emph{pure strategies} is $C = \times_{i \in N} C_i$,
$$C_i = \{ c_i = (c_i^{[k]})_{k = 1}^{\infty} \mid \forall k , c_i^{[k]} : (S_i)^{\times k} \rightarrow D_i\}. $$
The set of all \emph{behavioral strategies} is $B = \times_{i \in N} B_i$,
$$B_i = \{ \sigma_i = (\sigma_i^{[k]})_{k = 1}^{\infty} \mid \forall k , \sigma_i^{[k]} : (S_i)^{\times k} \rightarrow \Delta D_i\}. $$
\end{definition}
In the above, the set $(S_i)^{\times k}$ is the $k$-fold Cartesian product of the set $S_i$.\\
The next concept will allow us to compute the payoffs of a given set of strategies.
\begin{definition}[Probability of playing move and reaching state]
Given a repeated game $\Gamma^r = (N, \Theta, (D_i, S_i, u_i)_{i \in N}, q, p)$, a behavioral strategy $\sigma \in B$,
moves $d \in D$ and a state of the world $\theta \in \Theta$, we let
$$ P^{[k]}(d,\theta \mid \sigma)$$
be the probability that, at round $k$, the state of the world is $\theta$ and the players' moves are $d$ if they play according to $\sigma$.
\end{definition}

These probabilities can be computed as follows. Let
$$Q^{[k]}(s,\theta \mid \sigma) $$
be the probability that, at time $k$, the players are in the state $\theta$ and have received the signal $s \in (S)^{\times k}$.
By definition, we have
$$ P^{[k]}(d, \theta \mid \sigma) = \sum_{s \in (S)^{\times k}} Q^{[k]}(s, \theta \mid \sigma) \prod_{i \in N} \sigma^{[k]}_i(d_i \mid s). $$
The probabilities $Q^{[k]}$ can be computed recursively:
\begin{equation*}
\begin{aligned}
& Q^{[1]}(s^{[1]}, \theta^{[1]} \mid \sigma)  = q(s^{[1]}, \theta^{[1]}), \\
& Q^{[k+1]}(s^{[1]} \ldots s^{[k+1]}, \theta^{[k+1]} \mid \sigma) =  \\
& \qquad \sum_{\theta^{[k]} \in \Theta} \sum_{d \in D} Q^{[k]}(s^{[1]} \ldots s^{[k]}, \theta^{[k]} \mid \sigma) \left( \prod_{i \in N} \sigma_i^{[k]}(d_i \mid s^{[1]} \ldots s^{[k]}) \right ) p(s^{[k+1]}, \theta^{[k+1]}|d,\theta^{[k]}).
\end{aligned}
\end{equation*}

The payoff of a player $i  \in N$ at round $k$ when following the strategies in $\sigma$ is therefore given by
$$ r_i^{[k]}(\sigma)  = \sum_{d \in D} \sum_{\theta \in \Theta} P^{[k]}(d, \theta|\sigma) u_i(d,\theta).$$

\begin{definition}[Equilibrium]
The behavioral strategy $\sigma=(\sigma_i,\sigma_{-i})$ is an equilibrium for
\begin{itemize}
\item the $\delta$ - discounted payoff criterion if
	$$\forall i \in N, \forall \hat{\sigma}_i \in B_i: \sum_{k = 1}^\infty \delta^{k-1} r_i^{[k]}(\sigma) \geq \sum_{k = 1}^\infty \delta^{k-1} r_i^{[k]}(\hat{\sigma}_i,\sigma_{-i});$$
\item the average payoff criterion if
	$$\forall i \in N, \forall \hat{\sigma}_i \in B_i: \lim_{k \rightarrow \infty}\frac{\sum_{j = 1}^k r_i^{[j]}(\sigma)}{k} \geq \lim_{k \rightarrow \infty}\frac{\sum_{j = 1}^k r_i^{[j]}(\hat{\sigma}_i,\sigma_{-i})}{k}.$$
\end{itemize}
\end{definition}

The formalism above highlights one of the big challenges of the analysis of repeated games: the set of behavioral strategies is infinite, and so checking if a strategy is an equilibrium can be very hard!
In practice one often focuses on strategies relying only on limited information (for example, use only the $\ell$ last signal received to define the strategy). This can be understood as a relaxation of the concept of rationality and intelligence in favor of pragmatism: in reality,  players may not have access to a perfect memory or be able to reflect perfectly about the game. Think for example about the following situation: what if rational players, instead of choosing a move in each round by themselves,  have to program a computer with a limited amount of memory and computational power to do so?
We refer to this as the \emph{bounded rationality} principle, which is discussed in e.g.  \cite{RuMBR}, \cite[Section 6.1.3]{ShLeMSAG}, \cite[Chapter 9]{OsRuACIG}.


\subsection{Games with complete state information and discounting}

We now narrow our scope to a particular (yet expressive) class of games.
\begin{definition}[Games with Complete state information]
A repeated game $\Gamma^r = (N, \Theta, (D_i, S_i, u_i)_{i \in N}, q, p) $ has \emph{complete state information}
if, at every round, every player knows the current state of nature. That is, there is a function $w_i : S_i \rightarrow \Theta$ known by each player such that
$$\forall s \in S, \forall \theta, \theta' \in \Theta, \forall d \in D:  p(s , \theta' | d, \theta) = 0 \text{ if } \theta' \neq w_i(s_i). $$
\end{definition}

\begin{definition}[Stationary strategy]
In a game with complete state information, a behavioral strategy is \emph{stationary} if the move of each player at each time \emph{depends only on the information state of the game at that time}.
That is, there is a function $\tau_{i}: \Theta \rightarrow \Delta(D_i)$ such that
$$\forall k, \forall s \in (S)^{\times k}: \sigma_i^{[k]}(\cdot \mid s) = \tau_i(\cdot \mid w_i(s_i^{[k]})). $$
\end{definition}

\begin{example}
Consider the Prisoner's Dilemma game of Example \ref{ch9:ex:PD1} in a setting where players make this decision based only on the last moves of their adversary.
Note that this goes against the idea that players are rational and intelligent, since they use only a part of the information at their disposal (bounded rationality).
Nevertheless, many intuitive strategies can be analyzed in such a setting. It is for example the case for the Tit-for-Tat strategy discussed in Section \ref{sec:InfRep}.\\
A natural way to model the situation is as a game with complete state information.
Here, we have 4 states, each one assigned to the way the player moved on the last round:
$$ \Theta = \{([C],[c]), \,  ([C],[d]), \, ([D],[c]), \, ([D], [d]) \}. $$
Regarding the signals, we let $S_i = \Theta $ and both players see the state of the world at every round.\\
For the initial distribution, we can assume that $q( ([C],[c]) ) = 1$ in the beginning. \\
The transition function is straightforward:
$$ p(s, \theta \mid d, \theta') = 1 \text{ if } d = \theta \text{ and } s = (\theta,\theta), $$
and it is equal to $0$ for any other $s, \theta$.\\
If both players play according to the Tit-for-Tat (stationary here) strategy, the execution of the game can be summarized at Figure \ref{fig:Tit-for-Tat}.
\begin{figure}[!ht]
\centering
\begin{tikzpicture}[->,>=stealth',shorten >=1pt,auto,node distance=3cm, semithick, scale = 1, transform shape ]
\node	(start)								{Start};
\node[state] 	(Cc)    	[right of = start]		{$([C],[c])$};
\node[state] 	(Cd)		[right of = Cc]			{$([C],[d])$};
\node[state]	(Dc)	    [below of = Cc]			{$([D],[c])$};
\node[state] 	(Dd)		[right of = Dc]			{$([D],[d])$};
\path 	(start) edge node {}	(Cc)
		(Cc)	edge[loop above] node {C}	(Cc)
				edge[dashed] node[left] {D}	(Dc)
		(Dc)	edge[bend left] node[right] {C}	(Cd)
				edge[dashed] node[below] {D}	(Dd)
		(Cd) 	edge[dashed] node[above] {C}	(Cc)
				edge[bend left] node[left] {D}	(Dc)
		(Dd)	edge[loop right] node {D}	(Dd)
				edge[dashed] node[right] {C}	(Cd);
\end{tikzpicture}
\caption{Prisonner's dilemma when playing Tit-for-Tat. Assuming Player 2 plays Tit-for-Tat, the plain arrows correspond to when Player 1 plays Tit-for-Tat as well, and the dashed arrows are what would happen should he decide to do the opposite.}
\label{fig:Tit-for-Tat}
\end{figure}
\label{ex:stationaryGame}
\end{example}

We now provide characterization of equilibrium in stationary strategies for games with complete state information and the
$\delta$-discounted average payoff criterion, see Eq.  \eqref{eq:discountedAvgPayoffRep}, recalled below:
given a sequence of payoffs $r^{[1]}, r^{[2]}, \ldots$ and $0 \leq \delta < 1$, the $\delta$-discounted average of the sequence is
$(1-\delta) \sum_{j = 1}^{\infty} \delta^{j-1}r^{[j-1]}$.

Observe that this satisfies the following recursion: letting
$$w(k) = (1-\delta) \sum_{j = 1}^{\infty} \delta^{j-1}r^{[j+k-1]},  $$
which is the average from time $k$, we have
\begin{equation}
 w(1)  =  (1-\delta)r^{[1]} + \delta w(2).
\label{eq:discaveg}
\end{equation}
We are now in position to analyze games. To do so, we first let
$$ \nu_i(\theta, \tau) $$
be the \emph{expected $\delta$-discounted average} of player $i$ when beginning from the state $\theta \in \Theta$ when everyone plays according to $\tau$.
The following holds from \eqref{eq:discaveg}:
\begin{equation}
\nu_i(\theta, \tau) = \sum_{d_i \in D_i} \tau_i(d_i | \theta) \sum_{d_{-i} \in D_{-i}} \tau_{-i}(d_{-i} | \theta) \left ( (1-\delta) u_i(d_i, \theta) + \delta \sum_{\theta' \in \Theta}  p(\theta' |d, \theta) \nu_i(\theta', \tau) \right ).
\label{theNuequation}
\end{equation}
The above highlights an interesting term:
$$Y_i(\tau, d_i, \nu_i, \theta, \delta) =  \sum_{d_{-i} \in D_{-i}} \tau_{-i}(d_{-i} | \theta) \left ( (1-\delta) u_i(d_i, \theta) + \delta \sum_{\theta' \in \Theta}  p(\theta' |d, \theta) \nu_i(\theta', \tau) \right ), $$
which can be interpreted as the gain of player $i$ if he played the move $d_i$ \emph{once} if at state $\theta$ while everyone else follows $\tau$ (himself included in the future).

We end this section with a characterization of stationary equilibria for games with complete state information when considering the discounted average payoff:
\begin{theorem}
\label{thm:stateq}
Given a repeated game with complete state information and bounded payoffs, and given a profile of stationary strategies $\tau$. If there exists a bounded vector
$$ \nu = (\nu_i(\theta))_{\theta \in \Theta, i \in N} $$
such that
\begin{align}
& \nu_i(\theta) = \sum_{d_i \in D_i} \tau_i(d_i | \theta) Y_i(\tau, d_i, \nu_i, \theta, \delta), \label{thmStufmathc1}\\
& \nu_i(\theta) = \max_{d_i \in D_i} Y_i(\tau, d_i, \nu_i, \theta, \delta) .\label{thmStufmathc2}
\end{align}
the strategy profile $\tau$ is an equilibrium of the repeated game. \\
In this equilibrium, $\nu_i(\theta)$ is the expected $\delta-$discounted average payoff for player $i$ in the repeated game if the initial state is $\theta$.
\end{theorem}


In general the equations of Theorem \ref{thm:stateq} can be quite hard to solve. In the following, we give an example applied to our Prisoner Dilemma game where we analyze the Tit-for-Tat and Grim strategies introduced in Section \ref{sec:InfRep}.

\begin{example}[Example \ref{ex:stationaryGame} continued]
Our goal is now to find the values $\nu_i(\theta)$ for the prisoner dilemma at Example \ref{ex:stationaryGame}.
We saw in Example \ref{ch9:ex:trigger} that the Grim strategy was an equilibrium for the game with the $\delta-$discounted payoff when $\delta \geq 0.5$.
Let us now consider the Tit-for-Tat strategies for both players and we focus on Player 1.
For a given $0 \leq \delta < 1$, applying \eqref{theNuequation} to solve \eqref{thmStufmathc1} (it's the same...), we have the following:
$$
\begin{aligned}
\nu_1([C],[c])
& = (1-\delta) \cdot u_1([C],[c]) + \delta \cdot \nu_1([C],[c]) \\
& = (1-\delta) \cdot 1 + \delta \cdot \nu_1([C],[c]), \\
\nu_1([D],[c])
& = (1-\delta) \cdot u_1([C],[d]) + \delta \cdot \nu_1([C],[d]) \\ % The students presenting this course in 2019 got confused thinking that -1 is u_2([D], [c]) so explicitly writing u_1([C],[d]) should help
& = (1-\delta) \cdot (-1) + \delta \cdot \nu_1([C],[d]), \\
\nu_1([C],[d])
& = (1-\delta) \cdot u_1([D],[c]) + \delta \cdot \nu_1([D],[c]) \\
& = (1-\delta) \cdot 2 + \delta \cdot \nu_1([D],[c]), \\
\nu_1([D],[d])
& = (1-\delta) \cdot u_1([D],[d]) + \delta \cdot \nu_1([D],[d]) \\
& = (1-\delta) \cdot 0 + \delta \cdot \nu_1([D],[d]). \\
\end{aligned}
$$
The solutions are
$$
\begin{aligned}
& \nu_1([C],[c]) =  1 , \\
& \nu_1([D],[c]) = \frac{2\delta - 1}{1+\delta}, \\
& \nu_1([C],[d]) = \frac{2- \delta}{1+\delta}, \\
& \nu_1([D],[d]) = 0. \\
\end{aligned}
$$
Then, for each state, we need to verify \eqref{thmStufmathc2}. In our present case, this boils down to checking, at any state $\theta \in \Theta$, if we would not have an higher payoff by deviating from the prescribed action at this state. Hence, we would be at equilibrium whenever the following inequalities are satisfied.
$$
\begin{aligned}
& \nu_1([C],[c])  \geq (1-\delta) \cdot 2 + \delta\nu_1([D],[c]) = \nu_1([C],[d]) , \text{ doing $D$ instead of $C$,}\\
& \nu_1([D],[c])  \geq (1-\delta) \cdot 0 + \delta \nu_1([D],[d]) = 0,  \text{ doing $D$ instead of $C$ , }\\
& \nu_1([C],[d])  \geq (1-\delta) \cdot 1 + \delta \nu_1([C],[c]) = 1, \text{ doing $C$ instead of $D$, } \\
& \nu_1([D],[d])  \geq (1-\delta) \cdot (-1) + \delta \nu_1([C],[d])  = \nu_1([D],[c]). \text{ doing $C$ instead of $D$. } \\
\end{aligned}
$$
So we see that, for the equations to hold, we need to have exactly $\delta = 0.5$.
If it is not the case, then:
\begin{itemize}
\item For $\delta < 0.5$, the two first inequalities are invalid, but not the two second. This means that we should always do $D$.
Hence, we revert to our original behavior when playing the non-repeated Prisoner's Dilemma. This fits the interpretation of a low $\delta$ being associated to the search for an immediate reward.
\item For $\delta > 0.5$, the two second inequalities are invalid, but not the two first.
This means that if we know the second player is going for Tit-for-Tat, we should always cooperate when $\delta$ is big enough.
\end{itemize}



\end{example}

\section{Learning in repeated games}
In repeated games, \textit{learning} strategies can be defined as follows.

\begin{definition}[Learning strategy]
    A learning strategy is a strategy that uses the information collected by a
    player during the previous rounds to improve its behavior in future rounds.
\end{definition}

This corresponds to the way we see learning in our everyday life: using
past inputs to improve itself in the future.
In addition, learning strategies are supposed to draw interesting inferences
or use accumulated experience in an interesting way. This will be illustrated
with the convergence of ``fictitious play'' below.

There exists a lot of learning strategies. Two will be discussed in this
section \emph{fictitious play} and \emph{no-regret playing}.

\subsection{Fictitious play}
Fictitious play is an algorithm introduced by \textsc{Brown} at the RAND
corporation in 1949. Originally, the objective of this algorithm was to
provide an explanation for the origin of Nash equilibria as well a heuristic
to compute Nash equilibria. One can imagine this algorithm as a player
playing against itself multiple times, trying to devise the best strategy
(i.e. a Nash equilibrium) for when the time to play against a real opponent
will come.

More rigorously, fictitious play is an instance of \textit{model-based}
learning where learners maintain beliefs about opponent's behaviors. The
way the opponent's model is build is really simple and relies on a single
assumption: the opponent is playing a mixed strategy given by the empirical
distribution of its past moves.

\begin{procedure}[Fictitious play]
    \emph{Repeat:}
    \begin{enumerate}
        \item Play a best response to the assessed strategy of the opponent (if
        first round, play an arbitrary strategy)
	    \item Observe the opponent's actual play and update beliefs accordingly
    \end{enumerate}
    \label{chap9:fictplayproc}
\end{procedure}

Note that this algorithm is often given with an additional beliefs initialization
step. As a consequence, step 1 always consists in playing a best reponse (and
no longer an arbitrary strategy at the first round). However, this version does
not correspond to the original algorithm proposed by \textsc{Brown}.

\begin{example}[Fictitious play]
\label{ex:fictitiousPlay}
Let's consider the normal form game given in table~\ref{chap9:tab-fic-play-ex} and
two players using the fictitious play algorithm.

\begin{table}
    \centering
    \begin{tabular}{c|cc}
        & l & r \\
        \hline
        U & 3,3 & 0,0 \\
        D & 4,0 & 1,1 \\
    \end{tabular}
    \caption{Normal form game used in the Fictitious play example. The only
    Nash Equilibrium of this game is (D, r).}
    \label{chap9:tab-fic-play-ex}
\end{table}

Let's first define some notations
\begin{itemize}
        \item $\nu_i^t$: belief of player $i$ at turn $t$,
        \item $\mu_i^t$: expected play of player $i$ by the other players at turn $t$,
        \item $\sigma_i^t$: strategy of player $i$ at turn $t+1$.
\end{itemize}

At the first round, each player plays an arbitrary strategy.
Let's assume then that Player 1 plays U and Player 2 plays l initially.\\

At the second round the belief vectors of each player are the following:
$\nu_1^2=(1,0), \nu_2^2=(1,0)$. The expected move of the two players are
thus the following: Player 1 expects Player 2 to play $\mu_1^2=l$ and player
2 expects Player 1 to play $\mu_2^2=U$. The best response strategy of each
player is thus going to be the following: Player 1 is going to play
$\sigma_1^2=D$ and Player 2 is going to play $\sigma_2^2=l$.\\

At the third round the belief vectors are $\nu_1^3=(2,0)$ and $\nu_2^3=(1,1)$.
The expected strategy of each player is thus $\mu_1^3=l$ and
$\mu_2^3=\frac{1}{2}U+\frac{1}{2}D$. The best response strategy for each
player is thus $\sigma_1^3=D$ and $\sigma_2^3=l$: in fact for Player 2 the
expected payoff if  he plays $l$ is going to be equal to
$\frac{1}{2}\cdot3+\frac{1}{2}\cdot0=1.5$ and the payoff for playing right is
$\frac{1}{2}\cdot0+\frac{1}{2}\cdot1=0.5$, thus playing $l$ is the best response
against the supposed mixed strategy of the opponent.\\

At the fourth round the belief vectors are $\nu_1^4=(3,0)$ and $\nu_2^4=(1,2)$.
The expected strategy of each player is thus $\mu_1^4=l$ and
$\mu_2^4=\frac{1}{3}U+\frac{2}{3}D$. The best response strategy for each
player is thus $\sigma_1^4=D$ and $\sigma_2^4=l$: the expected payoff for
Player 2 is equal to $\frac{1}{3}\cdot3+\frac{2}{3}\cdot0=1$ if he plays l and
$\frac{1}{3}\cdot0+\frac{2}{3}\cdot1=\frac{2}{3}$ if he plays right.\\

At the fifth round the belief vectors are $\nu_1^5=(4,0)$ and $\nu_2^5=(1,3)$.
The expected strategy of each player is thus $\mu_1^5=l$ and
$\mu_2^5=\frac{1}{4}U+\frac{3}{4}D$. The best response strategy for each
player is thus $\sigma_1^5=D$ and $\sigma_2^5$ can be either equal to $l$
or to $r$, in fact, the expected payoffs for Player 2 are the same:
$\frac{1}{4}\cdot3+\frac{3}{4}\cdot0=0.75$ if he plays l and
$\frac{1}{4}\cdot0+\frac{3}{4}\cdot1=0.75$ if he plays right.\\

Finally at the sixth round, the belief vectors are $\nu_1^6=(5,0)$ and
$\nu_2^6=(1,4)$. The expected strategy of each player is thus $\mu_1^6=l$
and $\mu_2^6=\frac{1}{5}U+\frac{4}{5}D$. The best response strategy for
each player is thus $\sigma_1^4=D$ and $\sigma_2^4=r$: in fact,  the
expected payoff for Player 2 is equal to $\frac{1}{5}\cdot3+\frac{4}{5}\cdot0=0.6$
if he plays l and $\frac{1}{5}\cdot0+\frac{4}{5}\cdot1=0.8$ if he plays right.
Since the strategy $(D,r)$ is a Nash equilibrium (and a strongly dominating
strategy) of the game, the strategy of each player won't change anymore.\\
\end{example}

This example suggests that fictitious play might converge. In particular,
in might converge to a Nash's Equilibrium. Two questions remain, however:
does it always converge? Does it always converge to a Nash's Equilibrium?
The two following theorems answer these two questions.

\begin{theorem}
    \label{thm:fpconvergence}
    If the empirical distribution of each player's strategies converges in
    fictitious play, then \emph{it converges to a Nash equilibrium}.
\end{theorem}

\begin{theorem}
    \label{thm:fpconvergencecond}
    Each of the following are a sufficient conditions for the empirical
    frequencies of play to converge in fictitious play\footnote{Additionnal
    conditions that are out of the scope of this course exist but are not
    mentionned here.}
    \begin{itemize}
        \item the game is zero-sum;
        \item the game is solvable by iterated elimination of strictly dominated strategies
    \end{itemize}
\end{theorem}

\subsection{No-regret playing}
The second algorithm we are going to study in this section is \emph{no-regret
playing}. To be able to define the concept of \emph{no-regret playing}, we
first need to define the concept of \emph{regret}.

\begin{definition}[Regret]
    Let $\alpha^t$ be the average per-period reward the agent received up to time
    $t$. Let $\alpha^t(d)$ be the average per-period reward the agent would have
    received by playing strategy $d$. The regret an agent experiences at
    time $t$ for not having played $d$ is defined as
    $$R^t(d) = \alpha^t(d) - \alpha^t.$$
\end{definition}

Once we have defined the concept of regret, we can define \emph{no-regret playing}:

\begin{definition}[No-regret playing]
    A learning rule exhibits no regret if it guarantees with high probability that
    the agent will not experience any positive regret asymptotically, for all strategy $d$
    $$P([\liminf_{t\rightarrow \infty} R^t(d)] \leq 0) = 1.$$
\end{definition}

Thus, No-regret is an enviable property for any learning strategy.
But how to achieve it? Is it possible at all?
%We can see in this case that no-regret playing is a way more general concept
%than fictitious play: in fact in this case it is a whole new category of
%algorithms.
One example of a no-regret playing algorithm is \emph{regret
matching}.

The procedure for the regret matching algorithm is given below.

\begin{procedure}[Regret matching]
    \emph{Repeat:\\}
	\quad At each time step each action is chosen with probability proportional to its regret:\\
			$$\sigma_i^{t+1}(d)= \frac{R^t(d)}{\sum_{d'\in D_i} R^t(d')}$$
    \label{chap9:regretmatchproc}
\end{procedure}



\ifx \globalmark \undefined %% This is default.
\bibliographystyle{apalike}
\bibliography{../gametheorybibliography}
	\end{document}
\else

\fi
